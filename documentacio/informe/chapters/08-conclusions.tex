\section{Conclusions}
\label{sec:conclusions}

\vspace{0.5cm}


\subsection{Assoliment d'objectius}

En aquesta pràctica hem abordat amb èxit un problema real de planificació de rutes d'abastiment de combustible utilitzant tècniques de búsqueda local. El desenvolupament ha cobert tots els aspectes fonamentals de la resolució de problemes mitjançant Intel·ligència Artificial. Tots els objectius plantejats a la Secció \ref{sec:introduction} han estat assolits satisfactòriament:

\begin{enumerate}
    \item \textbf{Modelatge del problema}: Hem modelat el problema com un problema de busqueda local i definit una representació de l'espai d'estats que permet explorar totes les solucions: Secció \ref{sec:problem}\ i \ref{sec:state} \
    
    \item \textbf{Disseny d'operadors}: Hem desenvolupat un conjunt complet d'operadors que cobreixen tot l'espai de solucions: Secció  \ref{sec:operadors} \
    
    \item \textbf{Estratègies d'inicialització}: Hem comparat diferents aproximacions i escollit entre elles mitjançant l'experimentació: Secció \ref{sec:initial} \
    
    \item \textbf{Funció heurística}: Hem dissenyat una heurística que equilibra múltiples objectius del problema: Secció \ref{sec:heuristic} \
    
    \item \textbf{Experimentació rigorosa}: Hem realitzat els experiments sistemàticament i validat les nostres dessicions de disseny: Secció  \ref{sec:experiments} \
    
    \item \textbf{Comparació d'algoritmes}: Hem demostrat les diferències entre Hill Climbing i Simulated Annealing: Secció \ref{sec:experiments} \
    
\end{enumerate}

\vspace{0.5cm}


\subsection{Conclusions principals}
Les principals conclusions de la pràctica mostren que les tècniques de cerca local són una eina molt efectiva per abordar problemes reals de planificació de rutes. El model de representació utilitzat, basat en els viatges de cada camió, ha resultat eficient i flexible, i el conjunt d’operadors dissenyats permet explorar adequadament l’espai de solucions. També s’ha vist que la manera d’inicialitzar la solució influeix fortament en la velocitat de convergència i en la qualitat final dels resultats.

Pel que fa a la funció heurística, la combinació de beneficis, penalitzacions i costos de desplaçament s’ha demostrat equilibrada i capaç d’adaptar-se als diferents escenaris del problema. Els experiments han confirmat que els tres components són necessaris per mantenir un bon compromís entre ingressos i eficiència. 

En relació amb l'escalabilitat del problema, hem comprovat que Hill Climbing és molt més lent que Simulated Annealing, i el marge de benefici és molt just per justificar el seu ús: l'algorisme de SA aconsegueix aproximadament els mateixos beneficis i el seu cost temporal és molt més elevat. Si hem de tractar el problema amb escenaris grans, hem d'utilitzar SA per sobre de Hill Climbing degut al cost en temps tan exagerat.

A nivell de problema, s’ha observat que la restricció més limitant és el nombre de viatges per camió, ja que el nombre de punts d'abastiment i el cost per quilòmetre no han estat gaire restrictius.

En conjunt, la pràctica ha servit per entendre millor la relació entre modelatge, heurístiques i estratègies de cerca, i per comprovar que amb una bona representació, una heurística adequada i una experimentació sistemàtica és possible obtenir solucions de qualitat en temps molt raonables. A més, l’experiència ha reforçat la importància del treball en equip, la planificació constant i la documentació com a factors clau per a l’èxit del projecte. 

\vspace{0.5cm}


\subsection{Valoració personal}

\vspace{0.5cm}


\subsubsection{Dificultats trobades}
Ens va costar començar la pràctica, sobretot en la representació de l'estat. Ens va costar sobretot perquè no sabíem per on començar. També ens ha costat processar i emmagatzemar els resultats dels experiments per tal de tenir gràfiques adequades i trobar la manera d'executar alguns experiments que han trigat molt temps d'execució. Utilitzar Git com a eina de control de versions i gestió de codi, així com Latex per fer la documentació, no ha estat tan fàcil com ens pensàvem.

\vspace{0.5cm}


\subsubsection{Valoració del treball en equip}

El treball en equip ha estat fonamental per a l'èxit d'aquesta pràctica. Cada membre ha assumit responsabilitats clares i ens hem dividit la feina creant tasques diferenciades i independents. Hem aprofitat les classes de laboratori per fer reunions setmanals per sincronitzar el treball i discutir sobre el millor disseny de la pràctica. \\

Ens hem ajudat amb dubtes sempre que hem pogut i l'ajuda del Carles durant les hores de laboratori a on no arribem nosaltes també ha estat crítica. 
Seguir la planificació proposada a l'enunciat de la pràctica ens ha anat bé per anar al dia i tenir un progrés constant.

\vspace{0.5cm}


\subsection{Millores de la implementació actual}
Algunes millores que hem considerat i no hem implementat són:

\begin{enumerate}
    \item \textbf{Operadors més sofisticats}:
    \begin{itemize}
        \item Operador de reorganització completa d'un camió
        \item Operador d'intercanvi de múltiples peticions
        \item Operadors sensibles al context (segons estat actual)
    \end{itemize}

    \newpage
    
    \item \textbf{Heurística adaptativa}:
    \begin{itemize}
        \item Ajustar ponderacions
        \item Considerar la fase de la búsqueda (inici vs final)
        \item Incorporar informació històrica de la búsqueda
    \end{itemize}
\end{enumerate}


