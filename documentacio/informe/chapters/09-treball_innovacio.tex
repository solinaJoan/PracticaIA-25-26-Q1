\clearpage
%%%%%%%%%%%%%%%%%%%%%%%%%%%%%%%%%%%%%%%%%%%%%%%%%%%%%%%%%%%%%%%%%%%%%%%%%%%%%%%%
\section{Part II. Treball d'innovació}
%%%%%%%%%%%%%%%%%%%%%%%%%%%%%%%%%%%%%%%%%%%%%%%%%%%%%%%%%%%%%%%%%%%%%%%%%%%%%%%%


\subsection{Tema del treball}
\text{Adobe Firefly i la innovació Generative Fill a Photoshop}

\subsection{Breu descripció del tema}
Adobe Firefly és una plataforma d’intel·ligència artificial generativa creada per Adobe que permet generar i editar contingut visual mitjançant instruccions en llenguatge natural. La seva funció de Generative Fill a Photoshop permet afegir, eliminar o modificar elements d’una imatge de manera automàtica i realista. Aquesta innovació transforma el procés creatiu, fent-lo més ràpid, accessible i potent per a tot tipus d’usuaris.

\subsection{Repartiment del treball entre els membres del grup}

Hem dividit el treball en tres tasques diferencades que després repartirem i posarem en comú:

\begin{itemize}
    \item Encarregat de la cerca d’informació i redacció dels apartats relacionats amb la introducció, la descripció del producte o servei i el context de la innovació dins d’Adobe.
    \item Encarregat d'investigar i desenvolupar els apartats tècnics: tècniques d’IA utilitzades (models de difusió), adaptació a l’ecosistema Adobe i la naturalesa de la innovació (tipus i comparació amb altres solucions).
    \item Responsable dels apartats d’impacte a l’empresa, impacte en els usuaris i la societat, conclusions i de la compilació i revisió general del document.
    
\end{itemize}

Fins ara només hem triat el tema i dividit les tasques, encara no hem començat amb l'informe ni la cerca d'informació. Encara no tenim bibliografia ni ens hem trobat amb gaires dificultats

