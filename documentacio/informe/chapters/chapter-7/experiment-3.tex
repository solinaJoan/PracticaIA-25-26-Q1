
\subsection{Experiment 3: Ajust de paràmetres del Simulated Annealing}

\vspace{0.75cm}

\subsubsection{Objectiu}
Trobar els paràmetres òptims per a Simulated Annealing en el nostre problema.

\subsubsection{Resultats}

Els resultats mostren que, per a un nombre baix d’iteracions (1000 o 5000), totes les configuracions aconsegueixen beneficis similars i lleugerament inferiors als de Hill Climbing, la qual cosa indica que el nombre d’iteracions no és suficient perquè l’algorisme explori prou l’espai de cerca. Quan s’augmenta fins a 10000 iteracions, s’observa una millora més clara en el benefici obtingut, especialment per a la combinació amb $k = 25$ i $\lambda = 0.01$.

\vspace{0.5cm}

% --- Boxplots per iteracions = 1000 ---
\begin{figure}[H]
\centering
\begin{tikzpicture}
\begin{axis}[
    boxplot/draw direction=y,
    ylabel={Benefici (€)},
    xlabel={Combinació ($k$ -- $\lambda$)},
    x tick label style={text width=1.7cm, align=center, rotate=90},
    y tick label style={/pgf/number format/fixed,
                        /pgf/number format/precision=0,
                        /pgf/number format/fixed zerofill},
    scaled y ticks=false,
    ymajorgrids,
    width=\textwidth,
    height=8cm,
    xtick={1,2,3,4,5,6,7,8,9},
    xticklabels={
        {$k{=}5$\newline$\lambda{=}0.0001$},
        {$k{=}5$\newline$\lambda{=}0.001$},
        {$k{=}5$\newline$\lambda{=}0.01$},
        {$k{=}25$\newline$\lambda{=}0.0001$},
        {$k{=}25$\newline$\lambda{=}0.001$},
        {$k{=}25$\newline$\lambda{=}0.01$},
        {$k{=}125$\newline$\lambda{=}0.0001$},
        {$k{=}125$\newline$\lambda{=}0.001$},
        {$k{=}125$\newline$\lambda{=}0.01$}
    },
    ymin=92500, ymax=97000
]

% experiment-3-comp-param-sa-1000-5-1.0e-4.txt
\addplot+[boxplot prepared={median=94792.0, upper quartile=95230.0, lower quartile=93929.0, upper whisker=96456.0, lower whisker=93188.0}] coordinates {};
% experiment-3-comp-param-sa-1000-5-0.001.txt
\addplot+[boxplot prepared={median=94896.0, upper quartile=95439.0, lower quartile=94013.0, upper whisker=96436.0, lower whisker=93160.0}] coordinates {};
% experiment-3-comp-param-sa-1000-5-0.01.txt
\addplot+[boxplot prepared={median=94842.0, upper quartile=95239.0, lower quartile=94020.0, upper whisker=96432.0, lower whisker=93228.0}] coordinates {};
% experiment-3-comp-param-sa-1000-25-1.0e-4.txt
\addplot+[boxplot prepared={median=94772.0, upper quartile=95150.0, lower quartile=93834.0, upper whisker=96288.0, lower whisker=93088.0}] coordinates {};
% experiment-3-comp-param-sa-1000-25-0.001.txt
\addplot+[boxplot prepared={median=94766.0, upper quartile=95171.0, lower quartile=93841.0, upper whisker=96460.0, lower whisker=93228.0}] coordinates {};
% experiment-3-comp-param-sa-1000-25-0.01.txt
\addplot+[boxplot prepared={median=94848.0, upper quartile=95371.0, lower quartile=93822.0, upper whisker=96312.0, lower whisker=93148.0}] coordinates {};
% experiment-3-comp-param-sa-1000-125-1.0e-4.txt
\addplot+[boxplot prepared={median=94732.0, upper quartile=95108.0, lower quartile=93834.0, upper whisker=96324.0, lower whisker=93036.0}] coordinates {};
% experiment-3-comp-param-sa-1000-125-0.001.txt
\addplot+[boxplot prepared={median=94764.0, upper quartile=95109.0, lower quartile=93834.0, upper whisker=96288.0, lower whisker=93036.0}] coordinates {};
% experiment-3-comp-param-sa-1000-125-0.01.txt
\addplot+[boxplot prepared={median=94732.0, upper quartile=95108.0, lower quartile=93822.0, upper whisker=96288.0, lower whisker=93036.0}] coordinates {};


\end{axis}
\end{tikzpicture}
\caption{Distribució del benefici per combinació de $k$ i $\lambda$ amb 1000 iteracions (Simulated Annealing)}
\end{figure}


\vspace{0.5cm}

% --- Boxplots per iteracions = 5000 ---
\begin{figure}[H]
\centering
\begin{tikzpicture}
\begin{axis}[
    boxplot/draw direction=y,
    ylabel={Benefici (€)},
    xlabel={Combinació ($k$ -- $\lambda$)},
    x tick label style={text width=1.7cm, align=center, rotate=90},
    y tick label style={/pgf/number format/fixed,
                        /pgf/number format/precision=0,
                        /pgf/number format/fixed zerofill},
    scaled y ticks=false,
    ymajorgrids,
    width=\textwidth,
    height=8cm,
    xtick={1,2,3,4,5,6,7,8,9},
    xticklabels={
        {$k{=}5$\newline$\lambda{=}0.0001$},
        {$k{=}5$\newline$\lambda{=}0.001$},
        {$k{=}5$\newline$\lambda{=}0.01$},
        {$k{=}25$\newline$\lambda{=}0.0001$},
        {$k{=}25$\newline$\lambda{=}0.001$},
        {$k{=}25$\newline$\lambda{=}0.01$},
        {$k{=}125$\newline$\lambda{=}0.0001$},
        {$k{=}125$\newline$\lambda{=}0.001$},
        {$k{=}125$\newline$\lambda{=}0.01$}
    },
    ymin=92500, ymax=97000
]

% experiment-3-comp-param-sa-5000-5-1.0e-4.txt
\addplot+[boxplot prepared={median=94994.0, upper quartile=95506.0, lower quartile=94144.0, upper whisker=96664.0, lower whisker=93400.0}] coordinates {};
% experiment-3-comp-param-sa-5000-5-0.001.txt
\addplot+[boxplot prepared={median=94946.0, upper quartile=95539.0, lower quartile=94235.0, upper whisker=96600.0, lower whisker=93436.0}] coordinates {};
% experiment-3-comp-param-sa-5000-5-0.01.txt
\addplot+[boxplot prepared={median=95046.0, upper quartile=95542.0, lower quartile=94099.0, upper whisker=96680.0, lower whisker=93452.0}] coordinates {};

% experiment-3-comp-param-sa-5000-25-1.0e-4.txt
\addplot+[boxplot prepared={median=94786.0, upper quartile=95338.0, lower quartile=93810.0, upper whisker=96288.0, lower whisker=93252.0}] coordinates {};
% experiment-3-comp-param-sa-5000-25-0.001.txt
\addplot+[boxplot prepared={median=94860.0, upper quartile=95553.0, lower quartile=93982.0, upper whisker=96688.0, lower whisker=93568.0}] coordinates {};
% experiment-3-comp-param-sa-5000-25-0.01.txt
\addplot+[boxplot prepared={median=94972.0, upper quartile=95443.0, lower quartile=93975.0, upper whisker=96596.0, lower whisker=93488.0}] coordinates {};

% experiment-3-comp-param-sa-5000-125-1.0e-4.txt
\addplot+[boxplot prepared={median=94732.0, upper quartile=95111.0, lower quartile=93834.0, upper whisker=96300.0, lower whisker=93036.0}] coordinates {};
% experiment-3-comp-param-sa-5000-125-0.001.txt
\addplot+[boxplot prepared={median=94732.0, upper quartile=95108.0, lower quartile=93834.0, upper whisker=96288.0, lower whisker=93156.0}] coordinates {};
% experiment-3-comp-param-sa-5000-125-0.01.txt
\addplot+[boxplot prepared={median=94716.0, upper quartile=95108.0, lower quartile=93888.0, upper whisker=96288.0, lower whisker=93036.0}] coordinates {};



\end{axis}
\end{tikzpicture}
\caption{Distribució del benefici per combinació de $k$ i $\lambda$ amb 5000 iteracions (Simulated Annealing)}
\end{figure}


\vspace{0.5cm}

% --- Boxplots per iteracions = 10000 ---
\begin{figure}[H]
\centering
\begin{tikzpicture}
\begin{axis}[
    boxplot/draw direction=y,
    ylabel={Benefici (€)},
    xlabel={Combinació ($k$ -- $\lambda$)},
    x tick label style={text width=1.7cm, align=center, rotate=90},
    y tick label style={/pgf/number format/fixed,
                        /pgf/number format/precision=0,
                        /pgf/number format/fixed zerofill},
    scaled y ticks=false,
    ymajorgrids,
    width=\textwidth,
    height=8cm,
    xtick={1,2,3,4,5,6,7,8,9},
    xticklabels={
        {$k{=}5$\newline$\lambda{=}0.0001$},
        {$k{=}5$\newline$\lambda{=}0.001$},
        {$k{=}5$\newline$\lambda{=}0.01$},
        {$k{=}25$\newline$\lambda{=}0.0001$},
        {$k{=}25$\newline$\lambda{=}0.001$},
        {$k{=}25$\newline$\lambda{=}0.01$},
        {$k{=}125$\newline$\lambda{=}0.0001$},
        {$k{=}125$\newline$\lambda{=}0.001$},
        {$k{=}125$\newline$\lambda{=}0.01$}
    },
    ymin=92500, ymax=97000
]

% experiment-3-comp-param-sa-10000-5-1.0e-4.txt
\addplot+[boxplot prepared={median=95240.0, upper quartile=95625.0, lower quartile=94174.0, upper whisker=96768.0, lower whisker=93532.0}] coordinates {};
% experiment-3-comp-param-sa-10000-5-0.001.txt
\addplot+[boxplot prepared={median=95138.0, upper quartile=95673.0, lower quartile=94284.0, upper whisker=96772.0, lower whisker=93644.0}] coordinates {};
% experiment-3-comp-param-sa-10000-5-0.01.txt
\addplot+[boxplot prepared={median=95188.0, upper quartile=95800.0, lower quartile=94178.0, upper whisker=96716.0, lower whisker=93648.0}] coordinates {};

% experiment-3-comp-param-sa-10000-25-1.0e-4.txt
\addplot+[boxplot prepared={median=95082.0, upper quartile=95352.0, lower quartile=94316.0, upper whisker=96220.0, lower whisker=93732.0}] coordinates {};
% experiment-3-comp-param-sa-10000-25-0.001.txt
\addplot+[boxplot prepared={median=95268.0, upper quartile=95602.0, lower quartile=94299.0, upper whisker=96704.0, lower whisker=93944.0}] coordinates {};
% experiment-3-comp-param-sa-10000-25-0.01.txt
\addplot+[boxplot prepared={median=95292.0, upper quartile=95713.0, lower quartile=94146.0, upper whisker=96816.0, lower whisker=93828.0}] coordinates {};

% experiment-3-comp-param-sa-10000-125-1.0e-4.txt
\addplot+[boxplot prepared={median=94732.0, upper quartile=95108.0, lower quartile=93855.0, upper whisker=96288.0, lower whisker=93036.0}] coordinates {};
% experiment-3-comp-param-sa-10000-125-0.001.txt
\addplot+[boxplot prepared={median=94792.0, upper quartile=95495.0, lower quartile=94035.0, upper whisker=96460.0, lower whisker=93888.0}] coordinates {};
% experiment-3-comp-param-sa-10000-125-0.01.txt
\addplot+[boxplot prepared={median=94800.0, upper quartile=95367.0, lower quartile=94203.0, upper whisker=96288.0, lower whisker=93516.0}] coordinates {};


\end{axis}
\end{tikzpicture}
\caption{Distribució del benefici per combinació de $k$ i $\lambda$ amb 10000 iteracions (Simulated Annealing)}
\end{figure}


\vspace{0.5cm}

Pel que fa al temps d’execució, totes les combinacions mostren valors similars dins de cada nivell d’iteracions, de manera que la diferència de rendiment es deu principalment a la qualitat de les solucions trobades. Així doncs, el conjunt de paràmetres ($10000$ iteracions, $k = 25$, $\lambda = 0.01$) és el que proporciona els millors beneficis econòmics de manera consistent i, per molt que aquest increment de benefici vingui acompanyat amb un augment del cost temporal, aquest cost no es inabordable i és per això que s’ha seleccionat com a configuració òptima. 

\vspace{0.5cm}

% --- Boxplots per iteracions = 1000 ---
\begin{figure}[H]
\centering
\begin{tikzpicture}
\begin{axis}[
    boxplot/draw direction=y,
    ylabel={Temps (ms)},
    xlabel={Combinació ($k$ -- $\lambda$)},
    x tick label style={text width=1.7cm, align=center, rotate=90},
    y tick label style={/pgf/number format/fixed,
                        /pgf/number format/precision=0,
                        /pgf/number format/fixed zerofill},
    scaled y ticks=false,
    ymajorgrids,
    width=\textwidth,
    height=8cm,
    xtick={1,2,3,4,5,6,7,8,9},
    xticklabels={
        {$k{=}5$\newline$\lambda{=}0.0001$},
        {$k{=}5$\newline$\lambda{=}0.001$},
        {$k{=}5$\newline$\lambda{=}0.01$},
        {$k{=}25$\newline$\lambda{=}0.0001$},
        {$k{=}25$\newline$\lambda{=}0.001$},
        {$k{=}25$\newline$\lambda{=}0.01$},
        {$k{=}125$\newline$\lambda{=}0.0001$},
        {$k{=}125$\newline$\lambda{=}0.001$},
        {$k{=}125$\newline$\lambda{=}0.01$}
    },
    ymin=0, ymax=150
]

% experiment-3-comp-param-sa-1000-5-1.0e-4.txt
\addplot+[boxplot prepared={median=12.0, upper quartile=12.0, lower quartile=12.0, upper whisker=13, lower whisker=11}] coordinates {};
% experiment-3-comp-param-sa-1000-5-0.001.txt
\addplot+[boxplot prepared={median=12.0, upper quartile=13.75, lower quartile=12.0, upper whisker=15, lower whisker=12}] coordinates {};
% experiment-3-comp-param-sa-1000-5-0.01.txt
\addplot+[boxplot prepared={median=16.0, upper quartile=29.5, lower quartile=15.0, upper whisker=44, lower whisker=13}] coordinates {};

% experiment-3-comp-param-sa-1000-25-1.0e-4.txt
\addplot+[boxplot prepared={median=12.0, upper quartile=12.0, lower quartile=12.0, upper whisker=13, lower whisker=11}] coordinates {};
% experiment-3-comp-param-sa-1000-25-0.001.txt
\addplot+[boxplot prepared={median=12.0, upper quartile=12.75, lower quartile=12.0, upper whisker=14, lower whisker=11}] coordinates {};
% experiment-3-comp-param-sa-1000-25-0.01.txt
\addplot+[boxplot prepared={median=12.0, upper quartile=12.0, lower quartile=12.0, upper whisker=13, lower whisker=11}] coordinates {};

% experiment-3-comp-param-sa-1000-125-1.0e-4.txt
\addplot+[boxplot prepared={median=12.5, upper quartile=13.0, lower quartile=12.0, upper whisker=14, lower whisker=11}] coordinates {};
% experiment-3-comp-param-sa-1000-125-0.001.txt
\addplot+[boxplot prepared={median=12.0, upper quartile=12.75, lower quartile=12.0, upper whisker=15, lower whisker=11}] coordinates {};
% experiment-3-comp-param-sa-1000-125-0.01.txt
\addplot+[boxplot prepared={median=12.0, upper quartile=12.0, lower quartile=11.0, upper whisker=13, lower whisker=11}] coordinates {};



\end{axis}
\end{tikzpicture}
\caption{Distribució del temps d'execució per combinació de $k$ i $\lambda$ amb 1000 iteracions (Simulated Annealing)}
\end{figure}


\vspace{0.5cm}

% --- Boxplots per iteracions = 1000 ---
\begin{figure}[H]
\centering
\begin{tikzpicture}
\begin{axis}[
    boxplot/draw direction=y,
    ylabel={Temps (ms)},
    xlabel={Combinació ($k$ -- $\lambda$)},
    x tick label style={text width=1.7cm, align=center, rotate=90},
    y tick label style={/pgf/number format/fixed,
                        /pgf/number format/precision=0,
                        /pgf/number format/fixed zerofill},
    scaled y ticks=false,
    ymajorgrids,
    width=\textwidth,
    height=8cm,
    xtick={1,2,3,4,5,6,7,8,9},
    xticklabels={
        {$k{=}5$\newline$\lambda{=}0.0001$},
        {$k{=}5$\newline$\lambda{=}0.001$},
        {$k{=}5$\newline$\lambda{=}0.01$},
        {$k{=}25$\newline$\lambda{=}0.0001$},
        {$k{=}25$\newline$\lambda{=}0.001$},
        {$k{=}25$\newline$\lambda{=}0.01$},
        {$k{=}125$\newline$\lambda{=}0.0001$},
        {$k{=}125$\newline$\lambda{=}0.001$},
        {$k{=}125$\newline$\lambda{=}0.01$}
    },
    ymin=0, ymax=150
]

% experiment-3-comp-param-sa-5000-5-1.0e-4.txt
\addplot+[boxplot prepared={median=58.0, upper quartile=59.0, lower quartile=57.25, upper whisker=60, lower whisker=55}] coordinates {};
% experiment-3-comp-param-sa-5000-5-0.001.txt
\addplot+[boxplot prepared={median=58.0, upper quartile=59.75, lower quartile=57.0, upper whisker=61, lower whisker=54}] coordinates {};
% experiment-3-comp-param-sa-5000-5-0.01.txt
\addplot+[boxplot prepared={median=58.0, upper quartile=60.0, lower quartile=57.25, upper whisker=62, lower whisker=54}] coordinates {};

% experiment-3-comp-param-sa-5000-25-1.0e-4.txt
\addplot+[boxplot prepared={median=58.5, upper quartile=59.0, lower quartile=57.0, upper whisker=61, lower whisker=52}] coordinates {};
% experiment-3-comp-param-sa-5000-25-0.001.txt
\addplot+[boxplot prepared={median=58.0, upper quartile=59.0, lower quartile=56.25, upper whisker=63, lower whisker=53}] coordinates {};
% experiment-3-comp-param-sa-5000-25-0.01.txt
\addplot+[boxplot prepared={median=57.5, upper quartile=59.5, lower quartile=56.25, upper whisker=60, lower whisker=52}] coordinates {};

% experiment-3-comp-param-sa-5000-125-1.0e-4.txt
\addplot+[boxplot prepared={median=61.5, upper quartile=62.0, lower quartile=60.0, upper whisker=64, lower whisker=54}] coordinates {};
% experiment-3-comp-param-sa-5000-125-0.001.txt
\addplot+[boxplot prepared={median=59.5, upper quartile=61.5, lower quartile=58.25, upper whisker=64, lower whisker=54}] coordinates {};
% experiment-3-comp-param-sa-5000-125-0.01.txt
\addplot+[boxplot prepared={median=59.0, upper quartile=59.75, lower quartile=56.5, upper whisker=62, lower whisker=53}] coordinates {};

\end{axis}
\end{tikzpicture}
\caption{Distribució del temps d'execució per combinació de $k$ i $\lambda$ amb 1000 iteracions (Simulated Annealing)}
\end{figure}


\vspace{0.5cm}

% --- Boxplots per iteracions = 1000 ---
\begin{figure}[H]
\centering
\begin{tikzpicture}
\begin{axis}[
    boxplot/draw direction=y,
    ylabel={Temps (ms)},
    xlabel={Combinació ($k$ -- $\lambda$)},
    x tick label style={text width=1.7cm, align=center, rotate=90},
    y tick label style={/pgf/number format/fixed,
                        /pgf/number format/precision=0,
                        /pgf/number format/fixed zerofill},
    scaled y ticks=false,
    ymajorgrids,
    width=\textwidth,
    height=8cm,
    xtick={1,2,3,4,5,6,7,8,9},
    xticklabels={
        {$k{=}5$\newline$\lambda{=}0.0001$},
        {$k{=}5$\newline$\lambda{=}0.001$},
        {$k{=}5$\newline$\lambda{=}0.01$},
        {$k{=}25$\newline$\lambda{=}0.0001$},
        {$k{=}25$\newline$\lambda{=}0.001$},
        {$k{=}25$\newline$\lambda{=}0.01$},
        {$k{=}125$\newline$\lambda{=}0.0001$},
        {$k{=}125$\newline$\lambda{=}0.001$},
        {$k{=}125$\newline$\lambda{=}0.01$}
    },
    ymin=0, ymax=150
]

% experiment-3-comp-param-sa-10000-5-1.0e-4.txt
\addplot+[boxplot prepared={median=114.5, upper quartile=118.25, lower quartile=112.5, upper whisker=120, lower whisker=106}] coordinates {};
% experiment-3-comp-param-sa-10000-5-0.001.txt
\addplot+[boxplot prepared={median=116.0, upper quartile=117.5, lower quartile=112.25, upper whisker=120, lower whisker=107}] coordinates {};
% experiment-3-comp-param-sa-10000-5-0.01.txt
\addplot+[boxplot prepared={median=114.5, upper quartile=117.75, lower quartile=112.5, upper whisker=119, lower whisker=107}] coordinates {};


% experiment-3-comp-param-sa-10000-25-1.0e-4.txt
\addplot+[boxplot prepared={median=116.0, upper quartile=118.75, lower quartile=114.0, upper whisker=121, lower whisker=109}] coordinates {};
% experiment-3-comp-param-sa-10000-25-0.001.txt
\addplot+[boxplot prepared={median=114.5, upper quartile=116.5, lower quartile=112.5, upper whisker=120, lower whisker=106}] coordinates {};
% experiment-3-comp-param-sa-10000-25-0.01.txt
\addplot+[boxplot prepared={median=115.5, upper quartile=117.0, lower quartile=114.0, upper whisker=121, lower whisker=109}] coordinates {};


% experiment-3-comp-param-sa-10000-125-1.0e-4.txt
\addplot+[boxplot prepared={median=120.0, upper quartile=125.0, lower quartile=118.0, upper whisker=133, lower whisker=112}] coordinates {};
% experiment-3-comp-param-sa-10000-125-0.001.txt
\addplot+[boxplot prepared={median=117.5, upper quartile=119.5, lower quartile=114.5, upper whisker=122, lower whisker=111}] coordinates {};
% experiment-3-comp-param-sa-10000-125-0.01.txt
\addplot+[boxplot prepared={median=116.5, upper quartile=119.75, lower quartile=114.25, upper whisker=121, lower whisker=108}] coordinates {};


\end{axis}
\end{tikzpicture}
\caption{Distribució del temps d'execució per combinació de $k$ i $\lambda$ amb 1000 iteracions (Simulated Annealing)}
\end{figure}

