\subsection{Experiment 3: Ajust de paràmetres del Simulated Annealing}

\vspace{0.75cm}

\subsubsection{Objectiu}
Trobar els paràmetres òptims per a l'algorisme Simulated Annealing en el nostre problema de distribució de combustible.

\subsubsection{Configuració experimental}

Per aquest experiment s'han explorat sistemàticament diferents combinacions dels paràmetres clau de Simulated Annealing:

\begin{itemize}
    \item \textbf{Iteracions:} Nombre total d'iteracions de l'algorisme. S'han provat 1000, 5000 i 10000 iteracions.
    \item \textbf{$k$:} Factor d'escala de la temperatura inicial. S'han provat valors de 5, 25 i 125.
    \item \textbf{$\lambda$:} Taxa de refredament exponencial. S'han provat valors de 0,0001, 0,001 i 0,01.
\end{itemize}

Això resulta en 27 combinacions diferents, cadascuna executada 10 vegades per garantir la validesa estadística dels resultats. S'ha utilitzat el conjunt d'operadors només moviments identificat com a òptim en l'experiment 1.

\subsubsection{Anàlisi dels resultats}

\paragraph{Efecte del nombre d'iteracions amb 1000 iteracions}

\vspace{0.5cm}
% --- Boxplots per iteracions = 1000 ---
\begin{figure}[H]
\centering
\begin{tikzpicture}
\begin{axis}[
    boxplot/draw direction=y,
    ylabel={Benefici (€)},
    xlabel={Combinació ($k$ -- $\lambda$)},
    x tick label style={text width=1.7cm, align=center, rotate=90},
    y tick label style={/pgf/number format/fixed,
                        /pgf/number format/precision=0,
                        /pgf/number format/fixed zerofill},
    scaled y ticks=false,
    ymajorgrids,
    width=\textwidth,
    height=8cm,
    xtick={1,2,3,4,5,6,7,8,9},
    xticklabels={
        {$k{=}5$\newline$\lambda{=}0.0001$},
        {$k{=}5$\newline$\lambda{=}0.001$},
        {$k{=}5$\newline$\lambda{=}0.01$},
        {$k{=}25$\newline$\lambda{=}0.0001$},
        {$k{=}25$\newline$\lambda{=}0.001$},
        {$k{=}25$\newline$\lambda{=}0.01$},
        {$k{=}125$\newline$\lambda{=}0.0001$},
        {$k{=}125$\newline$\lambda{=}0.001$},
        {$k{=}125$\newline$\lambda{=}0.01$}
    },
    ymin=92500, ymax=97000
]

% experiment-3-comp-param-sa-1000-5-1.0e-4.txt
\addplot+[boxplot prepared={median=94792.0, upper quartile=95230.0, lower quartile=93929.0, upper whisker=96456.0, lower whisker=93188.0}] coordinates {};
% experiment-3-comp-param-sa-1000-5-0.001.txt
\addplot+[boxplot prepared={median=94896.0, upper quartile=95439.0, lower quartile=94013.0, upper whisker=96436.0, lower whisker=93160.0}] coordinates {};
% experiment-3-comp-param-sa-1000-5-0.01.txt
\addplot+[boxplot prepared={median=94842.0, upper quartile=95239.0, lower quartile=94020.0, upper whisker=96432.0, lower whisker=93228.0}] coordinates {};
% experiment-3-comp-param-sa-1000-25-1.0e-4.txt
\addplot+[boxplot prepared={median=94772.0, upper quartile=95150.0, lower quartile=93834.0, upper whisker=96288.0, lower whisker=93088.0}] coordinates {};
% experiment-3-comp-param-sa-1000-25-0.001.txt
\addplot+[boxplot prepared={median=94766.0, upper quartile=95171.0, lower quartile=93841.0, upper whisker=96460.0, lower whisker=93228.0}] coordinates {};
% experiment-3-comp-param-sa-1000-25-0.01.txt
\addplot+[boxplot prepared={median=94848.0, upper quartile=95371.0, lower quartile=93822.0, upper whisker=96312.0, lower whisker=93148.0}] coordinates {};
% experiment-3-comp-param-sa-1000-125-1.0e-4.txt
\addplot+[boxplot prepared={median=94732.0, upper quartile=95108.0, lower quartile=93834.0, upper whisker=96324.0, lower whisker=93036.0}] coordinates {};
% experiment-3-comp-param-sa-1000-125-0.001.txt
\addplot+[boxplot prepared={median=94764.0, upper quartile=95109.0, lower quartile=93834.0, upper whisker=96288.0, lower whisker=93036.0}] coordinates {};
% experiment-3-comp-param-sa-1000-125-0.01.txt
\addplot+[boxplot prepared={median=94732.0, upper quartile=95108.0, lower quartile=93822.0, upper whisker=96288.0, lower whisker=93036.0}] coordinates {};


\end{axis}
\end{tikzpicture}
\caption{Distribució del benefici per combinació de $k$ i $\lambda$ amb 1000 iteracions (Simulated Annealing)}
\end{figure}

\vspace{0.5cm}

Amb només 1000 iteracions, totes les combinacions de paràmetres generen resultats molt similars, amb una mediana al voltant dels 94.700-94.900€. Aquesta homogeneïtat indica que l'algorisme no té temps suficient per explorar adequadament l'espai de cerca, independentment dels valors de $k$ i $\lambda$ escollits. Els beneficis obtinguts són lleugerament inferiors als assolits per Hill Climbing en l'experiment 1 (95.076€).

\vspace{0.5cm}
% --- Boxplots per iteracions = 1000 ---
\begin{figure}[H]
\centering
\begin{tikzpicture}
\begin{axis}[
    boxplot/draw direction=y,
    ylabel={Temps (ms)},
    xlabel={Combinació ($k$ -- $\lambda$)},
    x tick label style={text width=1.7cm, align=center, rotate=90},
    y tick label style={/pgf/number format/fixed,
                        /pgf/number format/precision=0,
                        /pgf/number format/fixed zerofill},
    scaled y ticks=false,
    ymajorgrids,
    width=\textwidth,
    height=8cm,
    xtick={1,2,3,4,5,6,7,8,9},
    xticklabels={
        {$k{=}5$\newline$\lambda{=}0.0001$},
        {$k{=}5$\newline$\lambda{=}0.001$},
        {$k{=}5$\newline$\lambda{=}0.01$},
        {$k{=}25$\newline$\lambda{=}0.0001$},
        {$k{=}25$\newline$\lambda{=}0.001$},
        {$k{=}25$\newline$\lambda{=}0.01$},
        {$k{=}125$\newline$\lambda{=}0.0001$},
        {$k{=}125$\newline$\lambda{=}0.001$},
        {$k{=}125$\newline$\lambda{=}0.01$}
    },
    ymin=0, ymax=150
]

% experiment-3-comp-param-sa-1000-5-1.0e-4.txt
\addplot+[boxplot prepared={median=12.0, upper quartile=12.0, lower quartile=12.0, upper whisker=13, lower whisker=11}] coordinates {};
% experiment-3-comp-param-sa-1000-5-0.001.txt
\addplot+[boxplot prepared={median=12.0, upper quartile=13.75, lower quartile=12.0, upper whisker=15, lower whisker=12}] coordinates {};
% experiment-3-comp-param-sa-1000-5-0.01.txt
\addplot+[boxplot prepared={median=16.0, upper quartile=29.5, lower quartile=15.0, upper whisker=44, lower whisker=13}] coordinates {};

% experiment-3-comp-param-sa-1000-25-1.0e-4.txt
\addplot+[boxplot prepared={median=12.0, upper quartile=12.0, lower quartile=12.0, upper whisker=13, lower whisker=11}] coordinates {};
% experiment-3-comp-param-sa-1000-25-0.001.txt
\addplot+[boxplot prepared={median=12.0, upper quartile=12.75, lower quartile=12.0, upper whisker=14, lower whisker=11}] coordinates {};
% experiment-3-comp-param-sa-1000-25-0.01.txt
\addplot+[boxplot prepared={median=12.0, upper quartile=12.0, lower quartile=12.0, upper whisker=13, lower whisker=11}] coordinates {};

% experiment-3-comp-param-sa-1000-125-1.0e-4.txt
\addplot+[boxplot prepared={median=12.5, upper quartile=13.0, lower quartile=12.0, upper whisker=14, lower whisker=11}] coordinates {};
% experiment-3-comp-param-sa-1000-125-0.001.txt
\addplot+[boxplot prepared={median=12.0, upper quartile=12.75, lower quartile=12.0, upper whisker=15, lower whisker=11}] coordinates {};
% experiment-3-comp-param-sa-1000-125-0.01.txt
\addplot+[boxplot prepared={median=12.0, upper quartile=12.0, lower quartile=11.0, upper whisker=13, lower whisker=11}] coordinates {};



\end{axis}
\end{tikzpicture}
\caption{Distribució del temps d'execució per combinació de $k$ i $\lambda$ amb 1000 iteracions (Simulated Annealing)}
\end{figure}

\vspace{0.5cm}

El temps d'execució és extremadament baix per a totes les combinacions, amb una mediana al voltant dels 12 ms. Destaca només la combinació $k=5, \lambda=0.01$ amb temps lleugerament superiors, però l'impacte computacional és negligible en tots els casos.

\paragraph{Efecte del nombre d'iteracions amb 5000 iteracions}

\vspace{0.5cm}
% --- Boxplots per iteracions = 5000 ---
\begin{figure}[H]
\centering
\begin{tikzpicture}
\begin{axis}[
    boxplot/draw direction=y,
    ylabel={Benefici (€)},
    xlabel={Combinació ($k$ -- $\lambda$)},
    x tick label style={text width=1.7cm, align=center, rotate=90},
    y tick label style={/pgf/number format/fixed,
                        /pgf/number format/precision=0,
                        /pgf/number format/fixed zerofill},
    scaled y ticks=false,
    ymajorgrids,
    width=\textwidth,
    height=8cm,
    xtick={1,2,3,4,5,6,7,8,9},
    xticklabels={
        {$k{=}5$\newline$\lambda{=}0.0001$},
        {$k{=}5$\newline$\lambda{=}0.001$},
        {$k{=}5$\newline$\lambda{=}0.01$},
        {$k{=}25$\newline$\lambda{=}0.0001$},
        {$k{=}25$\newline$\lambda{=}0.001$},
        {$k{=}25$\newline$\lambda{=}0.01$},
        {$k{=}125$\newline$\lambda{=}0.0001$},
        {$k{=}125$\newline$\lambda{=}0.001$},
        {$k{=}125$\newline$\lambda{=}0.01$}
    },
    ymin=92500, ymax=97000
]

% experiment-3-comp-param-sa-5000-5-1.0e-4.txt
\addplot+[boxplot prepared={median=94994.0, upper quartile=95506.0, lower quartile=94144.0, upper whisker=96664.0, lower whisker=93400.0}] coordinates {};
% experiment-3-comp-param-sa-5000-5-0.001.txt
\addplot+[boxplot prepared={median=94946.0, upper quartile=95539.0, lower quartile=94235.0, upper whisker=96600.0, lower whisker=93436.0}] coordinates {};
% experiment-3-comp-param-sa-5000-5-0.01.txt
\addplot+[boxplot prepared={median=95046.0, upper quartile=95542.0, lower quartile=94099.0, upper whisker=96680.0, lower whisker=93452.0}] coordinates {};

% experiment-3-comp-param-sa-5000-25-1.0e-4.txt
\addplot+[boxplot prepared={median=94786.0, upper quartile=95338.0, lower quartile=93810.0, upper whisker=96288.0, lower whisker=93252.0}] coordinates {};
% experiment-3-comp-param-sa-5000-25-0.001.txt
\addplot+[boxplot prepared={median=94860.0, upper quartile=95553.0, lower quartile=93982.0, upper whisker=96688.0, lower whisker=93568.0}] coordinates {};
% experiment-3-comp-param-sa-5000-25-0.01.txt
\addplot+[boxplot prepared={median=94972.0, upper quartile=95443.0, lower quartile=93975.0, upper whisker=96596.0, lower whisker=93488.0}] coordinates {};

% experiment-3-comp-param-sa-5000-125-1.0e-4.txt
\addplot+[boxplot prepared={median=94732.0, upper quartile=95111.0, lower quartile=93834.0, upper whisker=96300.0, lower whisker=93036.0}] coordinates {};
% experiment-3-comp-param-sa-5000-125-0.001.txt
\addplot+[boxplot prepared={median=94732.0, upper quartile=95108.0, lower quartile=93834.0, upper whisker=96288.0, lower whisker=93156.0}] coordinates {};
% experiment-3-comp-param-sa-5000-125-0.01.txt
\addplot+[boxplot prepared={median=94716.0, upper quartile=95108.0, lower quartile=93888.0, upper whisker=96288.0, lower whisker=93036.0}] coordinates {};



\end{axis}
\end{tikzpicture}
\caption{Distribució del benefici per combinació de $k$ i $\lambda$ amb 5000 iteracions (Simulated Annealing)}
\end{figure}

\vspace{0.5cm}

Amb 5000 iteracions, comencen a aparèixer diferències més clares entre les combinacions. Les configuracions amb valors baixos de $k$ (5) tendeixen a obtenir beneficis lleugerament superiors (al voltant dels 95.000€), mentre que valors alts de $k$ (125) mantenen resultats similars al cas de 1000 iteracions. Això suggereix que temperatures inicials massa altes impedeixen que l'algorisme convergeixi adequadament dins del pressupost d'iteracions disponible.

\vspace{0.5cm}
% --- Boxplots per iteracions = 1000 ---
\begin{figure}[H]
\centering
\begin{tikzpicture}
\begin{axis}[
    boxplot/draw direction=y,
    ylabel={Temps (ms)},
    xlabel={Combinació ($k$ -- $\lambda$)},
    x tick label style={text width=1.7cm, align=center, rotate=90},
    y tick label style={/pgf/number format/fixed,
                        /pgf/number format/precision=0,
                        /pgf/number format/fixed zerofill},
    scaled y ticks=false,
    ymajorgrids,
    width=\textwidth,
    height=8cm,
    xtick={1,2,3,4,5,6,7,8,9},
    xticklabels={
        {$k{=}5$\newline$\lambda{=}0.0001$},
        {$k{=}5$\newline$\lambda{=}0.001$},
        {$k{=}5$\newline$\lambda{=}0.01$},
        {$k{=}25$\newline$\lambda{=}0.0001$},
        {$k{=}25$\newline$\lambda{=}0.001$},
        {$k{=}25$\newline$\lambda{=}0.01$},
        {$k{=}125$\newline$\lambda{=}0.0001$},
        {$k{=}125$\newline$\lambda{=}0.001$},
        {$k{=}125$\newline$\lambda{=}0.01$}
    },
    ymin=0, ymax=150
]

% experiment-3-comp-param-sa-5000-5-1.0e-4.txt
\addplot+[boxplot prepared={median=58.0, upper quartile=59.0, lower quartile=57.25, upper whisker=60, lower whisker=55}] coordinates {};
% experiment-3-comp-param-sa-5000-5-0.001.txt
\addplot+[boxplot prepared={median=58.0, upper quartile=59.75, lower quartile=57.0, upper whisker=61, lower whisker=54}] coordinates {};
% experiment-3-comp-param-sa-5000-5-0.01.txt
\addplot+[boxplot prepared={median=58.0, upper quartile=60.0, lower quartile=57.25, upper whisker=62, lower whisker=54}] coordinates {};

% experiment-3-comp-param-sa-5000-25-1.0e-4.txt
\addplot+[boxplot prepared={median=58.5, upper quartile=59.0, lower quartile=57.0, upper whisker=61, lower whisker=52}] coordinates {};
% experiment-3-comp-param-sa-5000-25-0.001.txt
\addplot+[boxplot prepared={median=58.0, upper quartile=59.0, lower quartile=56.25, upper whisker=63, lower whisker=53}] coordinates {};
% experiment-3-comp-param-sa-5000-25-0.01.txt
\addplot+[boxplot prepared={median=57.5, upper quartile=59.5, lower quartile=56.25, upper whisker=60, lower whisker=52}] coordinates {};

% experiment-3-comp-param-sa-5000-125-1.0e-4.txt
\addplot+[boxplot prepared={median=61.5, upper quartile=62.0, lower quartile=60.0, upper whisker=64, lower whisker=54}] coordinates {};
% experiment-3-comp-param-sa-5000-125-0.001.txt
\addplot+[boxplot prepared={median=59.5, upper quartile=61.5, lower quartile=58.25, upper whisker=64, lower whisker=54}] coordinates {};
% experiment-3-comp-param-sa-5000-125-0.01.txt
\addplot+[boxplot prepared={median=59.0, upper quartile=59.75, lower quartile=56.5, upper whisker=62, lower whisker=53}] coordinates {};

\end{axis}
\end{tikzpicture}
\caption{Distribució del temps d'execució per combinació de $k$ i $\lambda$ amb 1000 iteracions (Simulated Annealing)}
\end{figure}

\vspace{0.5cm}

El temps d'execució es manté notablement consistent entre totes les combinacions, situant-se al voltant dels 57-62 ms. Aquesta uniformitat indica que el cost computacional depèn principalment del nombre d'iteracions i no dels paràmetres específics de temperatura.

\paragraph{Efecte del nombre d'iteracions amb 10000 iteracions}

\vspace{0.5cm}
% --- Boxplots per iteracions = 10000 ---
\begin{figure}[H]
\centering
\begin{tikzpicture}
\begin{axis}[
    boxplot/draw direction=y,
    ylabel={Benefici (€)},
    xlabel={Combinació ($k$ -- $\lambda$)},
    x tick label style={text width=1.7cm, align=center, rotate=90},
    y tick label style={/pgf/number format/fixed,
                        /pgf/number format/precision=0,
                        /pgf/number format/fixed zerofill},
    scaled y ticks=false,
    ymajorgrids,
    width=\textwidth,
    height=8cm,
    xtick={1,2,3,4,5,6,7,8,9},
    xticklabels={
        {$k{=}5$\newline$\lambda{=}0.0001$},
        {$k{=}5$\newline$\lambda{=}0.001$},
        {$k{=}5$\newline$\lambda{=}0.01$},
        {$k{=}25$\newline$\lambda{=}0.0001$},
        {$k{=}25$\newline$\lambda{=}0.001$},
        {$k{=}25$\newline$\lambda{=}0.01$},
        {$k{=}125$\newline$\lambda{=}0.0001$},
        {$k{=}125$\newline$\lambda{=}0.001$},
        {$k{=}125$\newline$\lambda{=}0.01$}
    },
    ymin=92500, ymax=97000
]

% experiment-3-comp-param-sa-10000-5-1.0e-4.txt
\addplot+[boxplot prepared={median=95240.0, upper quartile=95625.0, lower quartile=94174.0, upper whisker=96768.0, lower whisker=93532.0}] coordinates {};
% experiment-3-comp-param-sa-10000-5-0.001.txt
\addplot+[boxplot prepared={median=95138.0, upper quartile=95673.0, lower quartile=94284.0, upper whisker=96772.0, lower whisker=93644.0}] coordinates {};
% experiment-3-comp-param-sa-10000-5-0.01.txt
\addplot+[boxplot prepared={median=95188.0, upper quartile=95800.0, lower quartile=94178.0, upper whisker=96716.0, lower whisker=93648.0}] coordinates {};

% experiment-3-comp-param-sa-10000-25-1.0e-4.txt
\addplot+[boxplot prepared={median=95082.0, upper quartile=95352.0, lower quartile=94316.0, upper whisker=96220.0, lower whisker=93732.0}] coordinates {};
% experiment-3-comp-param-sa-10000-25-0.001.txt
\addplot+[boxplot prepared={median=95268.0, upper quartile=95602.0, lower quartile=94299.0, upper whisker=96704.0, lower whisker=93944.0}] coordinates {};
% experiment-3-comp-param-sa-10000-25-0.01.txt
\addplot+[boxplot prepared={median=95292.0, upper quartile=95713.0, lower quartile=94146.0, upper whisker=96816.0, lower whisker=93828.0}] coordinates {};

% experiment-3-comp-param-sa-10000-125-1.0e-4.txt
\addplot+[boxplot prepared={median=94732.0, upper quartile=95108.0, lower quartile=93855.0, upper whisker=96288.0, lower whisker=93036.0}] coordinates {};
% experiment-3-comp-param-sa-10000-125-0.001.txt
\addplot+[boxplot prepared={median=94792.0, upper quartile=95495.0, lower quartile=94035.0, upper whisker=96460.0, lower whisker=93888.0}] coordinates {};
% experiment-3-comp-param-sa-10000-125-0.01.txt
\addplot+[boxplot prepared={median=94800.0, upper quartile=95367.0, lower quartile=94203.0, upper whisker=96288.0, lower whisker=93516.0}] coordinates {};


\end{axis}
\end{tikzpicture}
\caption{Distribució del benefici per combinació de $k$ i $\lambda$ amb 10000 iteracions (Simulated Annealing)}
\end{figure}

\vspace{0.5cm}

Amb 10000 iteracions, s'observa una millora clara i consistent en els beneficis obtinguts. Les millors combinacions són aquelles amb $k=5$ i $k=25$, independentment del valor de $\lambda$, assolint medianes al voltant dels 95.200-95.300€. Destaca especialment la combinació $k=25, \lambda=0.01$, que obté una mediana de 95.292€, superant lleugerament els resultats de Hill Climbing.

Les combinacions amb $k=125$ segueixen mostrant resultats inferiors, confirmant que temperatures inicials excessivament altes no són adequades per aquest problema, probablement perquè l'exploració inicial és massa aleatòria i impedeix una convergència eficient.

\vspace{0.5cm}
% --- Boxplots per iteracions = 1000 ---
\begin{figure}[H]
\centering
\begin{tikzpicture}
\begin{axis}[
    boxplot/draw direction=y,
    ylabel={Temps (ms)},
    xlabel={Combinació ($k$ -- $\lambda$)},
    x tick label style={text width=1.7cm, align=center, rotate=90},
    y tick label style={/pgf/number format/fixed,
                        /pgf/number format/precision=0,
                        /pgf/number format/fixed zerofill},
    scaled y ticks=false,
    ymajorgrids,
    width=\textwidth,
    height=8cm,
    xtick={1,2,3,4,5,6,7,8,9},
    xticklabels={
        {$k{=}5$\newline$\lambda{=}0.0001$},
        {$k{=}5$\newline$\lambda{=}0.001$},
        {$k{=}5$\newline$\lambda{=}0.01$},
        {$k{=}25$\newline$\lambda{=}0.0001$},
        {$k{=}25$\newline$\lambda{=}0.001$},
        {$k{=}25$\newline$\lambda{=}0.01$},
        {$k{=}125$\newline$\lambda{=}0.0001$},
        {$k{=}125$\newline$\lambda{=}0.001$},
        {$k{=}125$\newline$\lambda{=}0.01$}
    },
    ymin=0, ymax=150
]

% experiment-3-comp-param-sa-10000-5-1.0e-4.txt
\addplot+[boxplot prepared={median=114.5, upper quartile=118.25, lower quartile=112.5, upper whisker=120, lower whisker=106}] coordinates {};
% experiment-3-comp-param-sa-10000-5-0.001.txt
\addplot+[boxplot prepared={median=116.0, upper quartile=117.5, lower quartile=112.25, upper whisker=120, lower whisker=107}] coordinates {};
% experiment-3-comp-param-sa-10000-5-0.01.txt
\addplot+[boxplot prepared={median=114.5, upper quartile=117.75, lower quartile=112.5, upper whisker=119, lower whisker=107}] coordinates {};


% experiment-3-comp-param-sa-10000-25-1.0e-4.txt
\addplot+[boxplot prepared={median=116.0, upper quartile=118.75, lower quartile=114.0, upper whisker=121, lower whisker=109}] coordinates {};
% experiment-3-comp-param-sa-10000-25-0.001.txt
\addplot+[boxplot prepared={median=114.5, upper quartile=116.5, lower quartile=112.5, upper whisker=120, lower whisker=106}] coordinates {};
% experiment-3-comp-param-sa-10000-25-0.01.txt
\addplot+[boxplot prepared={median=115.5, upper quartile=117.0, lower quartile=114.0, upper whisker=121, lower whisker=109}] coordinates {};


% experiment-3-comp-param-sa-10000-125-1.0e-4.txt
\addplot+[boxplot prepared={median=120.0, upper quartile=125.0, lower quartile=118.0, upper whisker=133, lower whisker=112}] coordinates {};
% experiment-3-comp-param-sa-10000-125-0.001.txt
\addplot+[boxplot prepared={median=117.5, upper quartile=119.5, lower quartile=114.5, upper whisker=122, lower whisker=111}] coordinates {};
% experiment-3-comp-param-sa-10000-125-0.01.txt
\addplot+[boxplot prepared={median=116.5, upper quartile=119.75, lower quartile=114.25, upper whisker=121, lower whisker=108}] coordinates {};


\end{axis}
\end{tikzpicture}
\caption{Distribució del temps d'execució per combinació de $k$ i $\lambda$ amb 1000 iteracions (Simulated Annealing)}
\end{figure}

\vspace{0.5cm}

El temps d'execució amb 10000 iteracions se situa uniformement entre 114-120 ms per a totes les combinacions. Aquest cost temporal és aproximadament el doble que amb 5000 iteracions, reflectint la relació lineal esperada entre nombre d'iteracions i temps de càlcul.

\subsubsection{Conclusions}

Els resultats d'aquest experiment demostren que el nombre d'iteracions és el factor més crític per a l'èxit de Simulated Annealing en aquest problema. Amb 1000 o 5000 iteracions, l'algorisme no té temps suficient per explorar adequadament l'espai de cerca, mentre que amb 10000 iteracions s'observen millores consistents.

Pel que fa als paràmetres de temperatura, valors baixos de $k$ (5 o 25) són preferibles, ja que permeten una exploració inicial controlada seguida d'una convergència efectiva. El paràmetre $\lambda$ mostra menys impacte, amb tots els valors provats oferint resultats similars quan el nombre d'iteracions és suficient.

La configuració òptima identificada és \textbf{10000 iteracions, $k=25$, $\lambda=0.01$}, que proporciona els millors beneficis econòmics de manera consistent (mediana de 95.292€). Tot i que aquest increment de benefici respecte a Hill Climbing és modest (aproximadament un 0,2\%), el cost temporal addicional (115 ms versus 412 ms per Hill Climbing) és perfectament assumible i fa que aquesta configuració sigui adequada per a aplicacions on es prioritza la qualitat de la solució sobre la velocitat d'execució.