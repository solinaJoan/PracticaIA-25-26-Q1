
\subsection{Experiment 4: Escalabilitat temporal}

\vspace{0.75cm}

\subsubsection{Objectiu}

Aquest experiment té com a objectiu analitzar com creix el temps d’execució dels algorismes Hill Climbing i Simulated Annealing quan augmenta l’escalabilitat del problema, és a dir, el nombre de centres de distribució i gasolineres (de 10–100 fins a 50–500).


\subsubsection{Resultats}


Els resultats de sota mostren que el temps d’execució del Hill Climbing creix de manera clarament no lineal, passant d’uns pocs centenars de mil·lisegons a més de 200.000 ms per l’escenari més gran. Això és esperable, ja que l’algorisme ha d’explorar un espai de cerca cada vegada més ampli, amb més possibles moviments i combinacions per avaluar. En comparació amb Simulated Annealing, clarament es veu que Hill Climbing es inabordable per conjunts de centres i benzineres relativament grans.

\vspace{0.5cm}

\begin{figure}[H]
\centering
\begin{tikzpicture}
\begin{axis}[
    boxplot/draw direction=y,
    ylabel={Temps (ms)},
    xlabel={Combinació (Nombre centres - Nombre benzineres)},
    x tick label style={text width=1.7cm, align=center, rotate=90},
    y tick label style={/pgf/number format/fixed,
                        /pgf/number format/precision=0,
                        /pgf/number format/fixed zerofill},
    scaled y ticks=false,
    ymajorgrids,
    width=\textwidth,
    height=8cm,
    xtick={1,2,3,4,5},
    xticklabels={
        {$10$ centres\newline$100$ benzineres},
        {$20$ centres\newline$200$ benzineres},
        {$30$ centres\newline$300$ benzineres},
        {$40$ centres\newline$400$ benzineres},
        {$50$ centres\newline$500$ benzineres},
    },
    ymin=0, ymax=400000
]

% 10 centres - 100 benzineres
\addplot+[boxplot prepared={median=417.5, upper quartile=505.75, lower quartile=338.25, upper whisker=877, lower whisker=212}] coordinates {};
% 20 centres - 200 benzineres
\addplot+[boxplot prepared={median=6965.5, upper quartile=9305, lower quartile=5326.5, upper whisker=13418, lower whisker=4379}] coordinates {};
% 30 centres - 300 benzineres
\addplot+[boxplot prepared={median=37448.5, upper quartile=44651.75, lower quartile=28153, upper whisker=58145, lower whisker=23870}] coordinates {};
% 40 centres - 400 benzineres
\addplot+[boxplot prepared={median=121269.5, upper quartile=129318.25, lower quartile=109500.5, upper whisker=147542, lower whisker=95520}] coordinates {};
% 50 centres - 500 benzineres
\addplot+[boxplot prepared={median=242768, upper quartile=341844.75, lower quartile=232416.75, upper whisker=368106, lower whisker=176130}] coordinates {};

\end{axis}
\end{tikzpicture}
\caption{Evolució temporal de l'algorisme Hill Climbing al augmentar l'escalabilitat del problema}
\end{figure}


\vspace{0.5cm}

\input{chapters/chapter-7/figures/exp-4/sa-temps.tex}

\vspace{0.5cm}

A la gràfica de sota fem un \textit{zoom} per veure com evoluciona el cost temporal del Simulated Annealing. Aquest mostra un creixement pràcticament lineal: el temps passa d’uns 100 ms a 550 ms a mesura que el problema augmenta cinc vegades de mida.

\vspace{0.5cm}

\begin{figure}[H]
\centering
\begin{tikzpicture}
\begin{axis}[
    boxplot/draw direction=y,
    ylabel={Temps (ms)},
    xlabel={Combinació (Nombre centres - Nombre benzineres)},
    x tick label style={text width=1.7cm, align=center, rotate=90},
    y tick label style={/pgf/number format/fixed,
                        /pgf/number format/precision=0,
                        /pgf/number format/fixed zerofill},
    scaled y ticks=false,
    ymajorgrids,
    width=\textwidth,
    height=8cm,
    xtick={1,2,3,4,5},
    xticklabels={
        {$10$ centres\newline$100$ benzineres},
        {$20$ centres\newline$200$ benzineres},
        {$30$ centres\newline$300$ benzineres},
        {$40$ centres\newline$400$ benzineres},
        {$50$ centres\newline$500$ benzineres},
    }
]

% 10 centres - 100 benzineres
\addplot+[boxplot prepared={median=105.5, upper quartile=108.25, lower quartile=104.25, upper whisker=163, lower whisker=100}] coordinates {};
% 20 centres - 200 benzineres
\addplot+[boxplot prepared={median=213, upper quartile=217.25, lower quartile=210.25, upper whisker=255, lower whisker=159}] coordinates {};
% 30 centres - 300 benzineres
\addplot+[boxplot prepared={median=323, upper quartile=352.25, lower quartile=314.75, upper whisker=401, lower whisker=242}] coordinates {};
% 40 centres - 400 benzineres
\addplot+[boxplot prepared={median=444, upper quartile=449.25, lower quartile=414.5, upper whisker=473, lower whisker=315}] coordinates {};
% 50 centres - 500 benzineres
\addplot+[boxplot prepared={median=554.5, upper quartile=565.5, lower quartile=519.5, upper whisker=624, lower whisker=426}] coordinates {};


\end{axis}
\end{tikzpicture}
\caption{Evolució temporal de l'algorisme Simulated Annealing al augmentar l'escalabilitat del problema (ampliació de l'escala)}
\end{figure}


\vspace{0.5cm}

A més, a les següents gràfiques podem veure que les diferències entre els dos algorismes en quant a benefici econòmic són relativament mínimes, cosa que fa que no es justifiqui l'augment massiu del cost temporal per part de Hill Climbing.

\vspace{0.5cm}

\begin{figure}[H]
\centering
\begin{tikzpicture}
\begin{axis}[
    boxplot/draw direction=y,
    ylabel={Benefici econòmic (€)},
    xlabel={Combinació (Nombre centres - Nombre benzineres)},
    x tick label style={text width=1.7cm, align=center, rotate=90},
    y tick label style={/pgf/number format/fixed,
                        /pgf/number format/precision=0,
                        /pgf/number format/fixed zerofill},
    scaled y ticks=false,
    ymajorgrids,
    width=\textwidth,
    height=8cm,
    xtick={1,2,3,4,5},
    xticklabels={
        {$10$ centres\newline$100$ benzineres},
        {$20$ centres\newline$200$ benzineres},
        {$30$ centres\newline$300$ benzineres},
        {$40$ centres\newline$400$ benzineres},
        {$50$ centres\newline$500$ benzineres},
    }
]

% 10 centres - 100 benzineres
\addplot+[boxplot prepared={median=95076, upper quartile=95461, lower quartile=94086, upper whisker=96372, lower whisker=93344}] coordinates {};
% 20 centres - 200 benzineres
\addplot+[boxplot prepared={median=192372, upper quartile=192695, lower quartile=192140, upper whisker=192900, lower whisker=191384}] coordinates {};
% 30 centres - 300 benzineres
\addplot+[boxplot prepared={median=290178, upper quartile=290547, lower quartile=289197, upper whisker=292088, lower whisker=288036}] coordinates {};
% 40 centres - 400 benzineres
\addplot+[boxplot prepared={median=387858, upper quartile=388642, lower quartile=386390, upper whisker=389820, lower whisker=385268}] coordinates {};
% 50 centres - 500 benzineres
\addplot+[boxplot prepared={median=486234, upper quartile=487617, lower quartile=485496, upper whisker=489056, lower whisker=484612}] coordinates {};


\end{axis}
\end{tikzpicture}
\caption{Evolució del benefici econòmic de l'algorisme Hill Climbing al augmentar l'escalabilitat del problema}
\end{figure}


\vspace{0.5cm}

\begin{figure}[H]
\centering
\begin{tikzpicture}
\begin{axis}[
    boxplot/draw direction=y,
    ylabel={Benefici econòmic (€)},
    xlabel={Combinació (Nombre centres - Nombre benzineres)},
    x tick label style={text width=1.7cm, align=center, rotate=90},
    y tick label style={/pgf/number format/fixed,
                        /pgf/number format/precision=0,
                        /pgf/number format/fixed zerofill},
    scaled y ticks=false,
    ymajorgrids,
    width=\textwidth,
    height=8cm,
    xtick={1,2,3,4,5},
    xticklabels={
        {$10$ centres\newline$100$ benzineres},
        {$20$ centres\newline$200$ benzineres},
        {$30$ centres\newline$300$ benzineres},
        {$40$ centres\newline$400$ benzineres},
        {$50$ centres\newline$500$ benzineres},
    }
]

% 10 centres - 100 benzineres
\addplot+[boxplot prepared={median=95214, upper quartile=95689, lower quartile=94264, upper whisker=96768, lower whisker=93848}] coordinates {};
% 20 centres - 200 benzineres
\addplot+[boxplot prepared={median=192172, upper quartile=192354, lower quartile=191928, upper whisker=192884, lower whisker=191228}] coordinates {};
% 30 centres - 300 benzineres
\addplot+[boxplot prepared={median=289798, upper quartile=290440, lower quartile=288971, upper whisker=292632, lower whisker=287516}] coordinates {};
% 40 centres - 400 benzineres
\addplot+[boxplot prepared={median=387642, upper quartile=388465, lower quartile=386010, upper whisker=389852, lower whisker=384840}] coordinates {};
% 50 centres - 500 benzineres
\addplot+[boxplot prepared={median=486144, upper quartile=487351, lower quartile=484886, upper whisker=488944, lower whisker=484060}] coordinates {};



\end{axis}
\end{tikzpicture}
\caption{Evolució del benefici econòmic de l'algorisme Simulated Annealing al augmentar l'escalabilitat del problema}
\end{figure}
