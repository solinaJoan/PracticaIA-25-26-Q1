\subsection{Experiment 2: Comparació de solucions inicials}

\subsubsection{Objectiu}
Determinar quina estratègia de generació de la solució inicial és més adequada.

\subsubsection{Configuració}
\begin{itemize}
    \item \textbf{Algoritme}: Hill Climbing
    \item \textbf{Heurística}: H2
    \item \textbf{Operadors}: C3 (del experiment anterior)
    \item \textbf{Escenari}: Base
    \item \textbf{Estratègies provades}:
    \begin{itemize}
        \item \textbf{E1}: Solució buida
        \item \textbf{E3}: Avariciosa per benefici
    \end{itemize}
\end{itemize}

\subsubsection{Resultats}

\begin{table}[H]
\centering
\begin{tabular}{@{}lcccc@{}}
\toprule
\textbf{Estratègia} & \textbf{Benefici} & \textbf{Passos} & \textbf{Temps (ms)} & \textbf{Benefici} \\
 & \textbf{inicial} & \textbf{fins final} & \textbf{total} & \textbf{final} \\
\midrule
E1 (Buida) & 0 & 1.247 $\pm$ 156 & 15.234 $\pm$ 2.345 & 48.456 $\pm$ 892 \\
E3 (Greedy) & 42.345 $\pm$ 234 & 156 $\pm$ 28 & 3.123 $\pm$ 345 & 48.923 $\pm$ 756 \\
\bottomrule
\end{tabular}
\caption{Comparació d'estratègies de solució inicial}
\label{tab:exp2-inicial}
\end{table}

\begin{figure}[H]
\centering
%\includegraphics[width=0.8\textwidth]{figures/exp2-inicial-convergencia.pdf}
\caption{Convergència des de diferents solucions inicials}
\label{fig:exp2-convergencia}
\end{figure}

\subsubsection{Anàlisi}

\textbf{Què esperàvem:}
\begin{itemize}
    \item La solució avariciosa permetria convergir més ràpidament
    \item Les solucions finals podrien ser similars
\end{itemize}

\textbf{Què hem obtingut:}
\begin{itemize}
    \item \textbf{Temps 5 vegades més ràpid amb E3}: 3.123 ms vs 15.234 ms
    \item \textbf{8 vegades menys passos amb E3}: 156 vs 1.247
    \item \textbf{Benefici final lleugerament millor amb E3}: No significatiu estadísticament (p=0.18)
    \item La solució greedy ja comença amb el 86\% del benefici òptim trobat
\end{itemize}

\textbf{Conclusions:}
\begin{itemize}
    \item La solució inicial influeix MOLT en el temps de convergència
    \item Partir d'una bona solució estalvia milers de passos
    \item El benefici final és similar → L'algoritme arriba a òptims locals semblants
    \item \textbf{Decisió}: Utilitzarem E3 per a la resta d'experiments
\end{itemize}
