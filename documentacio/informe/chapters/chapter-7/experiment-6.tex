
\subsection{Experiment 6: Variació del cost per quilòmetre}

\vspace{0.5cm}


\subsubsection{Objectiu}
L'experiment té com a objectiu analitzar com l'augment del cost per quilòmetre recorregut afecta el nombre de peticions servides en un sistema de gestió de peticions, utilitzant l'algoritme Hill Climbing. Es vol observar si aquest increment de costos redueix la proporció de peticions servides i si hi ha algun efecte en funció del temps que les peticions porten pendents.

\vspace{0.5cm}


\subsubsection{Configuració}
S'ha utilitzat l'algorisme Hill Climbing per optimitzar la planificació de les rutes, partint d'un escenari base on el cost per quilòmetre és de 2 unitats. Aquest cost s'ha duplicat successivament fins a assolir valors de 4, 8, 16 i 32 unitats. S'ha mantingut constant la resta de paràmetres de l'entorn, com el nombre de peticions i la distribució temporal d'aquestes. S'han recollit les dades de benefici econòmic i el nombre de peticions servides.

\vspace{0.5cm}


\subsubsection{Resultats}

\paragraph{Benefici Econòmic}
Com es pot observar a la Figura 21, a mesura que anem augmentant el cost per quilòmetre, s'observa una clara disminució del benefici.Té molt sentit que si es vol seguir servint totes les peticions (com veurem a la següent taula) i el cost per quilòmetre puja, el benefici baixa.

\begin{figure}[H]
\centering
\begin{tikzpicture}
\begin{axis}[
    boxplot/draw direction=y,
    ylabel={Benefici (€)},
    xlabel={Límit de km per dia},
    xtick={1,2,3},
    xticklabels={560, 640, 720},
    x tick label style={font=\footnotesize},
    y tick label style={/pgf/number format/fixed,
                        /pgf/number format/precision=0,
                        /pgf/number format/fixed zerofill},
    scaled y ticks=false,
    ymajorgrids,
    width=\textwidth,
    height=8cm
]
% --- 560 km/dia ---
\addplot+[boxplot prepared={
    median=94880,
    upper quartile=95512,
    lower quartile=94000,
    upper whisker=96372,
    lower whisker=93344
}] coordinates {};
% --- 640 km/dia ---
\addplot+[boxplot prepared={
    median=94880,
    upper quartile=95512,
    lower quartile=94000,
    upper whisker=96372,
    lower whisker=93344
}] coordinates {};
% --- 720 km/dia ---
\addplot+[boxplot prepared={
    median=94880,
    upper quartile=95512,
    lower quartile=94000,
    upper whisker=96372,
    lower whisker=93344
}] coordinates {};
\end{axis}
\end{tikzpicture}
\caption{Benefici econòmic segons el límit de km diari}
\end{figure}


\paragraph{Peticions servides}
Com es pot observar a la Figura 22, el nombre de peticions servides es manté constant durant to l'experiment. Això es deu pricipalment a que la funció heurística no penalitza suficient el cost per quilòmetre o premia en excés el voler servir totes les peticions, i finalment el preu per quilòmetre no afecta al nombre de peticions servides 
\begin{figure}[H]
\centering
\begin{tikzpicture}
\begin{axis}[
    boxplot/draw direction=y,
    ylabel={Peticions servides},
    xlabel={Cost per km},
    xtick={1,2,3,4,5},
    xticklabels={2, 4, 8, 16, 32},
    x tick label style={font=\footnotesize},
    y tick label style={/pgf/number format/fixed},
    scaled y ticks=false,
    ymin=99.5, ymax=100.5,
    width=\textwidth,
    height=6cm,
    ymajorgrids
]

% --- cost = 2 ---
\addplot+[boxplot prepared={
    median=100,
    upper quartile=100,
    lower quartile=100,
    upper whisker=100,
    lower whisker=100
}] coordinates {};

% --- cost = 4 ---
\addplot+[boxplot prepared={
    median=100,
    upper quartile=100,
    lower quartile=100,
    upper whisker=100,
    lower whisker=100
}] coordinates {};

% --- cost = 8 ---
\addplot+[boxplot prepared={
    median=100,
    upper quartile=100,
    lower quartile=100,
    upper whisker=100,
    lower whisker=100
}] coordinates {};

% --- cost = 16 ---
\addplot+[boxplot prepared={
    median=100,
    upper quartile=100,
    lower quartile=100,
    upper whisker=100,
    lower whisker=100
}] coordinates {};

% --- cost = 32 ---
\addplot+[boxplot prepared={
    median=100,
    upper quartile=100,
    lower quartile=100,
    upper whisker=100,
    lower whisker=100
}] coordinates {};

\end{axis}
\end{tikzpicture}
\caption{Nombre de peticions servides segons el cost per km (totes servides amb èxit)}
\end{figure}


\vspace{0.5cm}


\subsubsection{Conclusió}
Els resultats mostren que l'increment del cost per quilòmetre té un impacte negatiu clar sobre el benefici econòmic, però no afecta el nombre total de peticions servides, que es manté al 100\% en tots els casos. Això suggereix que el sistema, en la seva configuració actual, prioritza la cobertura completa de la demanda per damunt de l’eficiència econòmica.
  
Pel que fa a la proporció de peticions servides en funció del temps que porten pendents, no s'ha observat cap variació significativa, ja que totes les peticions acaben essent ateses. Per poder avaluar aquest efecte amb més detall, seria necessari repetir l’experiment en un escenari amb major càrrega de treball o amb restriccions més severes (menys vehicles, major distància o més peticions simultànies), de manera que no totes les peticions poguessin ser servides. En aquests contextos, sí es podria analitzar si el cost per quilòmetre influeix en la priorització de peticions antigues o recents. \\

En resum, l'experiment confirma la sensibilitat del benefici respecte al cost operatiu, però no evidencia canvis en el comportament del sistema pel que fa al servei de peticions.