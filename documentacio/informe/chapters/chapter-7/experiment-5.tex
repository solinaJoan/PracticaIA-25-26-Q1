\subsection{Experiment 5: Reducció de centres amb mateix nombre de camions}

\subsubsection{Objectiu}
Analitzar l'impacte de concentrar els camions en menys centres.

\subsubsection{Configuració}
\begin{itemize}
    \item \textbf{Escenari A}: 10 centres, 1 camió/centre, 100 gasolineres
    \item \textbf{Escenari B}: 5 centres, 2 camions/centre, 100 gasolineres
    \item \textbf{Algoritme}: Hill Climbing
\end{itemize}

\subsubsection{Resultats}

\begin{table}[H]
\centering
\begin{tabular}{@{}lccccc@{}}
\toprule
\textbf{Escenari} & \textbf{Benefici} & \textbf{Cost km} & \textbf{Km totals} & \textbf{Peticions} & \textbf{Temps} \\
 & & & & \textbf{servides} & \textbf{(ms)} \\
\midrule
A (10c×1) & 48.923 $\pm$ 756 & 5.234 $\pm$ 234 & 2.617 & 93 $\pm$ 2 & 3.123 \\
B (5c×2) & 46.234 $\pm$ 892 & 6.789 $\pm$ 345 & 3.395 & 91 $\pm$ 3 & 3.456 \\
\textbf{Diferència} & \textbf{-5.50\%} & \textbf{+29.7\%} & \textbf{+29.7\%} & \textbf{-2.15\%} & \textbf{+10.7\%} \\
\bottomrule
\end{tabular}
\caption{Comparació 10 centres vs 5 centres}
\label{tab:exp5-centres}
\end{table}

\begin{figure}[H]
\centering
%\includegraphics[width=0.7\textwidth]{figures/exp5-mapa-rutes.pdf}
\caption{Visualització de les rutes per ambdós escenaris}
\label{fig:exp5-mapa}
\end{figure}

\subsubsection{Anàlisi}

\textbf{Què esperàvem:}
\begin{itemize}
    \item Menys centres → més distància
    \item Benefici similar si es serveixen les mateixes peticions
\end{itemize}

\textbf{Què hem obtingut:}
\begin{itemize}
    \item \textbf{Augment del 30\% en km}: 2.617 → 3.395 km
    \item \textbf{Reducció del 5.5\% en benefici}: Significatiu (p < 0.01)
    \item \textbf{2 peticions menys servides}: 93 → 91
    \item La concentració de centres penalitza la distribució geogràfica
\end{itemize}

\textbf{Implicacions:}
\begin{itemize}
    \item La distribució geogràfica dels centres és crucial
    \item El cost extra de km pot fer inviables algunes peticions
    \item Amb cost km = 2, la pèrdua és 1.556 unitats extra
    \item \textbf{Conclusió}: Mantenir una bona cobertura geogràfica és essencial
\end{itemize}
