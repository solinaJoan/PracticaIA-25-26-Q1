\subsection{Experiment 5: Reducció de centres amb mateix nombre de camions}

\subsubsection{Objectiu}
Analitzar l'impacte de concentrar els camions en menys centres, mantenint el nombre total de camions constant (10 camions en total), per tal d'avaluar els efectes en l'eficiència operativa, el benefici econòmic i la capacitat de resposta del servei.


\subsubsection{Configuració}
S'han comparat dos escenaris diferents amb el mateix nombre total de camions (10) però amb diferent distribució:
\begin{itemize}
    \item \textbf{Escenari A}: 10 centres amb 1 camió per centre
    \item \textbf{Escenari B}: 5 centres amb 2 camions per centre
\end{itemize}

L'experiment s'ha executat 10 vegades per a cada escenari per garantir la significació estadística dels resultats. S'han analitzat tres mètriques principals: benefici econòmic, quilòmetres recorreguts i percentatge de peticions servides.

\subsubsection{Resultats i Anàlisi}

\paragraph{Benefici Econòmic}

Com es pot observar a la Figura 11, l'escenari amb 10 centres presenta un benefici econòmic lleugerament superior (mitjana: 95 mil€) respecte a l'escenari amb 5 centres (mitjana: 94mil€). La diferència és estadísticament significativa, amb una dispersió menor en el cas dels 10 centres, indicant major consistència en els resultats.
\begin{figure}[H]
\centering
\begin{tikzpicture}
\begin{axis}[
    boxplot/draw direction=y,
    ylabel={Benefici (€)},
    xlabel={Límit de km per dia},
    xtick={1,2,3},
    xticklabels={560, 640, 720},
    x tick label style={font=\footnotesize},
    y tick label style={/pgf/number format/fixed,
                        /pgf/number format/precision=0,
                        /pgf/number format/fixed zerofill},
    scaled y ticks=false,
    ymajorgrids,
    width=\textwidth,
    height=8cm
]
% --- 560 km/dia ---
\addplot+[boxplot prepared={
    median=94880,
    upper quartile=95512,
    lower quartile=94000,
    upper whisker=96372,
    lower whisker=93344
}] coordinates {};
% --- 640 km/dia ---
\addplot+[boxplot prepared={
    median=94880,
    upper quartile=95512,
    lower quartile=94000,
    upper whisker=96372,
    lower whisker=93344
}] coordinates {};
% --- 720 km/dia ---
\addplot+[boxplot prepared={
    median=94880,
    upper quartile=95512,
    lower quartile=94000,
    upper whisker=96372,
    lower whisker=93344
}] coordinates {};
\end{axis}
\end{tikzpicture}
\caption{Benefici econòmic segons el límit de km diari}
\end{figure}


\paragraph{Quilòmetres Recorreguts}

La Figura 12 mostra una diferència notable en la distància total recorreguda. L'escenari amb 5 centres presenta un major recorregut mitjà (2.900 km) respecte als 10 centres (2.600 km). Això representa un increment del 12.5\% en la distància recorreguda quan es concentren els camions en menys centres.
\begin{figure}[H]
\centering
\begin{tikzpicture}
\begin{axis}[
    boxplot/draw direction=y,
    ylabel={Km recorreguts},
    xlabel={Escenari},
    xtick={1,2},
    xticklabels={10 centres (1 camió/centre), 5 centres (2 camions/centre)},
    x tick label style={text width=4cm, align=center, font=\footnotesize},
    y tick label style={/pgf/number format/fixed,
                        /pgf/number format/precision=0},
    scaled y ticks=false,
    width=0.8\textwidth,
    height=8cm,
    ymajorgrids
]
\addplot+[
    boxplot prepared={
        median=2610,
        upper quartile=2744,
        lower quartile=2244,
        upper whisker=3118,
        lower whisker=1874
    },
] coordinates {};
\addplot+[
    boxplot prepared={
        median=2936,
        upper quartile=3180,
        lower quartile=2738,
        upper whisker=4200,
        lower whisker=2578
    },
] coordinates {};
\end{axis}
\end{tikzpicture}
\caption{Comparació dels km recorreguts entre ambdós escenaris}
\end{figure}


\paragraph{Peticions Servides}
Com es veu a la Figura 13, ambdós escenaris serveixen el 100\% de les peticions en totes les execucions, demostrant que la capacitat de resposta no es veu afectada per la distribució dels camions i centres, i igualment totes les peticions son servides
\begin{figure}[H]
\centering
\begin{tikzpicture}
\begin{axis}[
    boxplot/draw direction=y,
    ylabel={Peticions servides},
    xlabel={Escenari},
    xtick={1,2},
    xticklabels={10 centres (1 camió/centre), 5 centres (2 camions/centre)},
    x tick label style={text width=4cm, align=center, font=\footnotesize},
    y tick label style={/pgf/number format/fixed},
    scaled y ticks=false,
    ymin=99.5, ymax=100.5,  % totes són 100
    width=0.8\textwidth,
    height=6cm,
    ymajorgrids
]
\addplot+[
    boxplot prepared={
        median=100,
        upper quartile=100,
        lower quartile=100,
        upper whisker=100,
        lower whisker=100
    },
] coordinates {};
\addplot+[
    boxplot prepared={
        median=100,
        upper quartile=100,
        lower quartile=100,
        upper whisker=100,
        lower whisker=100
    },
] coordinates {};
\end{axis}
\end{tikzpicture}
\caption{Comparació de peticions servides entre ambdós escenaris}
\end{figure}


\subsubsection{Anàlisi}
Els resultats indiquen que la distribució amb més centres (encara que amb menys camions per centre) és més eficient. La reducció de centres implica que els camions han de realitzar trajectes més llargs per cobrir tot el territori, incrementant significativament els quilòmetres recorreguts i, conseqüentment, reduint el benefici econòmic degut als majors costos operatius.

Això suggereix que, en aquest context específic, una distribució més descentralitzada dels recursos és preferible, ja que permet reduir les distàncies de desplaçament mantenint la mateixa capacitat de resposta. La proximitat dels centres als punts de servei sembla ser un factor crític en l'eficiència del sistema.

\subsubsection{Conclusió}
La concentració de camions en menys centres, tot i mantenir el mateix nombre total de vehicles, resulta en un increment dels costos operatius i una reducció del benefici econòmic, sense millorar la capacitat de servei. Per tant, en aquest escenari concret, la distribució amb 10 centres és més eficient que la distribució amb 5 centres.

