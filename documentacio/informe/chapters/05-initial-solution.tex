\section{Estratègies de generació de la solució inicial}
\label{sec:initial}

\subsection{Imporsdfasdfasftància de la solució inicial}

La solució inicial té un impacte significatiu en la búsqueda local:

\begin{itemize}
    \item \textbf{Temps de convergència}: Una bona solució inicial redueix el nombre de passos necessaris
    \item \textbf{Qualitat de la solució final}: Pot influir en quin òptim local s'arriba
    \item \textbf{Cost de generació}: Cal balancejar qualitat vs temps de càlcul
\end{itemize}

\subsection{Estratègies proposades}

\subsubsection{Estratègia 1: Solució buida}

\paragraph{Descripció:}
No s'assigna cap petició inicialment. Tots els camions comencen sense viatges.

\begin{lstlisting}[caption={Implementació de solució buida}, label={lst:empty-init}]
public PracticaBoard() {
    assignacions = new ArrayList<>();
    for (int i = 0; i < NUM_CAMIONS; i++) {
        assignacions.add(new ArrayList<>());
    }
    peticionsServides = new HashSet<>();
    beneficiTotal = calcularBenefici();
}
\end{lstlisting}

\paragraph{Avantatges:}
\begin{itemize}
    \item Temps de generació: $O(1)$ - instantani
    \item Implementació trivial
    \item Sempre compleix les restriccions
\end{itemize}

\paragraph{Inconvenients:}
\begin{itemize}
    \item Benefici inicial: 0 (pèssim)
    \item Requereix molts passos per arribar a una bona solució
    \item Pot quedar atrapada en òptims locals primerencs
\end{itemize}

\subsubsection{Estratègia 2: Assignació aleatòria}

\paragraph{Descripció:}
S'assignen peticions aleatòriament als camions fins que no es poden afegir més sense violar restriccions.

\begin{algorithm}[H]
\caption{Generació aleatòria de solució inicial}
\begin{algorithmic}[1]
\State $peticions \gets$ barrejar\_aleatoriament(totes\_peticions)
\For{cada $p$ in $peticions$}
    \State $camio \gets$ aleatori(0, NUM\_CAMIONS-1)
    \State $nou\_estat \gets$ afegir\_peticio($p$, $camio$)
    \If{$nou\_estat.esValid()$}
        \State estat $\gets$ $nou\_estat$
    \EndIf
\EndFor
\State \Return estat
\end{algorithmic}
\end{algorithm}

\paragraph{Avantatges:}
\begin{itemize}
    \item Temps de generació: $O(p)$ - ràpid
    \item Genera solucions diferents en cada execució
    \item Sol complir restriccions (amb validació)
\end{itemize}

\paragraph{Inconvenients:}
\begin{itemize}
    \item Qualitat variable segons l'aleatorietat
    \item No garanteix aprofitar bé la capacitat dels camions
    \item Pot deixar moltes peticions sense servir
\end{itemize}

\subsubsection{Estratègia 3: Assignació avariciosa per benefici}

\paragraph{Descripció:}
S'assignen les peticions en ordre de benefici decreixent, assignant cada petició al camió més proper que pugui servir-la.

\begin{algorithm}[H]
\caption{Generació avariciosa per benefici}
\begin{algorithmic}[1]
\State $peticions \gets$ ordenar\_per\_benefici\_decreixent()
\For{cada $p$ in $peticions$}
    \State $millor\_camio \gets$ NULL
    \State $min\_dist \gets \infty$
    \For{cada $camio$ in $camions$}
        \State $dist \gets$ distancia($camio$, gasolinera\_de($p$))
        \If{$dist < min\_dist$ AND pot\_afegir($camio$, $p$)}
            \State $millor\_camio \gets camio$
            \State $min\_dist \gets dist$
        \EndIf
    \EndFor
    \If{$millor\_camio \neq$ NULL}
        \State afegir\_peticio($p$, $millor\_camio$)
    \EndIf
\EndFor
\end{algorithmic}
\end{algorithm}

\paragraph{Càlcul del benefici d'una petició:}
\begin{lstlisting}[caption={Càlcul del benefici}, label={lst:benefici-peticio}]
private double calcularBeneficiPeticio(Peticio p) {
    int dies = p.getDies();
    double preu;
    
    if (dies == 0) {
        preu = 1000 * 1.02;
    } else {
        preu = 1000 * (1 - Math.pow(2, dies) / 100.0);
    }
    return preu;
}
\end{lstlisting}

\paragraph{Avantatges:}
\begin{itemize}
    \item Prioritza les peticions més rendibles
    \item Minimitza distàncies (menys cost)
    \item Sol generar solucions acceptables
    \item Temps de generació: $O(p \times n \times \log p)$ - raonable
\end{itemize}

\paragraph{Inconvenients:}
\begin{itemize}
    \item Pot deixar camions infrautilitzats
    \item No considera l'agrupació òptima en viatges
    \item Sempre genera la mateixa solució (determinista)
\end{itemize}

\subsubsection{Estratègia 4: Assignació avariciosa per proximitat geogràfica}

\paragraph{Descripció:}
Agrupa les peticions per zones geogràfiques i assigna cada zona al camió més proper.

\begin{algorithm}[H]
\caption{Generació avariciosa per proximitat}
\begin{algorithmic}[1]
\State $clusters \gets$ clustering\_geografic(gasolineres, NUM\_CAMIONS)
\For{cada $cluster$ in $clusters$}
    \State $camio \gets$ camio\_mes\_proper($cluster$)
    \State $peticions \gets$ peticions\_del\_cluster($cluster$)
    \State ordenar\_per\_ruta\_optima($peticions$)
    \For{cada $p$ in $peticions$}
        \If{pot\_afegir($camio$, $p$)}
            \State afegir\_peticio($p$, $camio$)
        \EndIf
    \EndFor
\EndFor
\end{algorithmic}
\end{algorithm}

\paragraph{Avantatges:}
\begin{itemize}
    \item Minimitza distàncies totals
    \item Aprofita millor l'agrupació geogràfica
    \item Genera viatges més eficients
\end{itemize}

\paragraph{Inconvenients:}
\begin{itemize}
    \item Complexitat de càlcul més alta: $O(p^2)$
    \item Requereix algoritme de clustering
    \item No considera el benefici de les peticions
\end{itemize}

\subsection{Comparació teòrica de les estratègies}

\begin{table}[H]
\centering
\begin{tabular}{@{}lcccc@{}}
\toprule
\textbf{Estratègia} & \textbf{Complexitat} & \textbf{Benefici} & \textbf{Distància} & \textbf{Peticions} \\
 & \textbf{temporal} & \textbf{inicial} & \textbf{inicial} & \textbf{servides} \\
\midrule
Buida & $O(1)$ & Mínim & Mínima & 0 \\
Aleatòria & $O(p)$ & Variable & Variable & Variable \\
Per benefici & $O(p \times n \times \log p)$ & Alt & Baixa & Alta \\
Per proximitat & $O(p^2)$ & Mitjà & Mínima & Mitjana-Alta \\
\bottomrule
\end{tabular}
\caption{Comparació teòrica de les estratègies}
\label{tab:estrategies-comp}
\end{table}

\subsection{Estratègies escollides per experimentació}

Hem escollit experimentar amb les següents estratègies:

\begin{enumerate}
    \item \textbf{Estratègia buida} (E1): Com a línia base
    \item \textbf{Estratègia avariciosa per benefici} (E3): Com a solució elaborada
\end{enumerate}

\textbf{Justificació:}
\begin{itemize}
    \item E1 representa el pitjor cas i és útil per veure la capacitat de millora de l'algoritme
    \item E3 equilibra qualitat i temps de generació
    \item Ambdues són deterministes, facilitant la comparació
    \item L'aleatòria (E2) no és reproducible
    \item La geogràfica (E4) és massa costosa per la millora marginal esperada
\end{itemize}

\subsection{Implementació de la solució escollida}

\begin{lstlisting}[caption={Implementació de la solució avariciosa per benefici}, label={lst:greedy-impl}]
public static PracticaBoard generarSolucioGreedy() {
    PracticaBoard board = new PracticaBoard();
    
    // Ordenar peticions per benefici decreixent
    List<Map.Entry<Integer, Peticio>> peticionsOrdenades = 
        new ArrayList<>(peticions.entrySet());
    peticionsOrdenades.sort((a, b) -> 
        Double.compare(
            calcularBeneficiPeticio(b.getValue()),
            calcularBeneficiPeticio(a.getValue())
        )
    );
    
    // Assignar cada petició al millor camió
    for (Map.Entry<Integer, Peticio> entry : peticionsOrdenades) {
        int idPeticio = entry.getKey();
        Peticio peticio = entry.getValue();
        
        int millorCamio = trobarMillorCamio(board, peticio);
        
        if (millorCamio != -1) {
            board.afegirPeticio(idPeticio, millorCamio);
        }
    }
    
    return board;
}

private static int trobarMillorCamio(PracticaBoard board, 
                                      Peticio peticio) {
    int millorCamio = -1;
    double minDistancia = Double.MAX_VALUE;
    
    for (int i = 0; i < NUM_CAMIONS; i++) {
        double dist = calcularDistanciaAfegir(board, i, peticio);
        
        if (dist < minDistancia && 
            board.potAfegirPeticio(i, peticio)) {
            millorCamio = i;
            minDistancia = dist;
        }
    }
    
    return millorCamio;
}
\end{lstlisting}

\subsection{Expectatives per a l'experimentació}

\textbf{Hipòtesi 1:} La solució avariciosa per benefici (E3) permetrà convergir més ràpidament que la solució buida (E1).

\textbf{Hipòtesi 2:} La qualitat de la solució final serà similar independentment de la solució inicial, però el nombre de passos diferirà significativament.

\textbf{Hipòtesi 3:} Per a Simulated Annealing, la influència de la solució inicial serà menor que per a Hill Climbing, ja que pot escapar d'òptims locals.

Aquestes hipòtesis es validaran en la secció d'experimentació (Secció \ref{sec:experiments}).