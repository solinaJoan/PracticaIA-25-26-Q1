\section{Estratègies de generació de la solució inicial}
\label{sec:initial}

\subsection{Importància de la solució inicial}

La qualitat de la solució inicial té un impacte significatiu en el rendiment dels algoritmes de cerca local. Una solució inicial de qualitat pot reduir el temps de convergència i millorar la qualitat de la solució final obtinguda. És per això que hem explorat diverses estratègies per generar la solució inicial, avaluant el compromís entre la qualitat de la solució generada i el cost computacional d'obtenir-la.

\subsection{Estratègies Considerades}

\subsection{Assignació Greedy per Benefici}

L'estratègia greedy per benefici consisteix en assignar iterativament les peticions més prometedores a cada camió, prioritzant aquelles que maximitzen el benefici mentre es respecten les restriccions del problema.

\subsubsection{Descripció de l'algorisme}

L'algorisme segueix un procediment constructiu que assigna peticions als camions de manera iterativa. Per cada camió disponible, es construeixen viatges de fins a dues peticions cadascun, respectant el límit màxim de viatges per dia. El procés es divideix en dues fases:

\begin{enumerate}
    \item \textbf{Selecció de la primera petició:} Es busca la petició no assignada que proporciona el màxim benefici considerant la distància des del centre de distribució del camió.
    
    \item \textbf{Selecció de la segona petició:} Un cop assignada la primera petició, es busca una segona petició compatible que minimitzi la distància addicional respecte a la primera, maximitzant així l'eficiència del viatge.
\end{enumerate}

Aquest procés es repeteix fins que no es poden crear més viatges per al camió actual o s'ha assolit el límit de viatges diari. El criteri de selecció considera tant el benefici econòmic com la proximitat geogràfica entre peticions.

\begin{algorithm}[H]
\caption{Generació de Solució Inicial Greedy}
\begin{algorithmic}[1]
\Procedure{GenerarSolucióGreedy}{}
    \For{cada camió $i$ en la flota}
        \State $centro \gets$ centre de distribució del camió $i$
        \For{$j = 1$ to MAX\_VIATGES\_DIA}
            \State $viatge \gets$ nou viatge buit
            \State $p_1 \gets$ TrobarMillorPetició($centro$, null)
            \If{$p_1 \neq$ null}
                \State Afegir $p_1$ al viatge
                \State Eliminar $p_1$ de peticions no assignades
                \State $p_2 \gets$ TrobarMillorPetició($centro$, $p_1$)
                \If{$p_2 \neq$ null}
                    \State Afegir $p_2$ al viatge
                    \State Eliminar $p_2$ de peticions no assignades
                \EndIf
                \State Afegir viatge a $viatgesPerCamio[i]$
            \Else
                \State \textbf{break} 
            \EndIf
        \EndFor
    \EndFor
\EndProcedure
\end{algorithmic}
\end{algorithm}

\subsubsection{Justificació de l'estratègia}

L'estratègia greedy per benefici presenta diversos avantatges que justifiquen la seva elecció:

\begin{itemize}
    \item \textbf{Qualitat de la solució:} Aquest enfocament garanteix que cada decisió d'assignació considera el benefici de la petició, evitant solucions trivials o de baixa qualitat. La incorporació de la proximitat geogràfica en la selecció de la segona petició del viatge introdueix una heurística que tendeix a minimitzar els costos de transport, resultant en solucions que balancentegen benefici i eficiència operativa.
    
    \item \textbf{Eficiència computacional:} La complexitat temporal és $O(C \cdot V \cdot P)$, on $C$ és el nombre de camions, $V$ el nombre màxim de viatges per camió, i $P$ el nombre de peticions. En la pràctica, aquesta complexitat es redueix a $O(C \cdot P)$ perquè $V$ és una constant petita i les peticions s'eliminen de la llista de candidates.
    
    \item \textbf{Reproducibilitat:} Aquesta estratègia genera sempre la mateixa solució per a una instància donada, facilitant la comparació d'experiments i la depuració del codi.
    
    \item \textbf{Bona base per a cerca local:} Les solucions generades ja incorporen decisions raonables sobre l'agrupació de peticions, proporcionant un punt de partida sòlid per als operadors de cerca local.
\end{itemize}

\subsection{Altres Estratègies Descartades}

Vam considerar també la solució inicial buida la qual consisteix en iniciar sense cap assignació de peticions als camions, deixant que els operadors de cerca local construeixin la solució des de zero. 
Però la vam acabar descartent com bé s'explica a l'apartat d'experiments.
