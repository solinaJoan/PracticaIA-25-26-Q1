\subsection{Configuració}
\begin{itemize}
    \item \textbf{Centres}: 10
    \item \textbf{Camions per centre}: 1
    \item \textbf{Gasolineres}: 100
    \item \textbf{Semilla}: 1234 (centres i gasolineres)
    \item \textbf{Algoritme}: Simulated Annealing amb paràmetres optimitzats
    \item \textbf{Operadors}: C3 (complet)
    \item \textbf{Solució inicial}: E3 (Greedy)
    \item \textbf{Heurística}: H2
\end{itemize}

\subsubsection{Resultats}

\begin{table}[H]
\centering
\begin{tabular}{@{}lc@{}}
\toprule
\textbf{Mètrica} & \textbf{Valor} \\
\midrule
Benefici de peticions servides & 96.847 \\
Cost de quilòmetres & -5.234 \\
Penalització peticions no servides & -1.456 \\
\midrule
\textbf{Benefici net final} & \textbf{50.157} \\
\midrule
Temps d'execució & 7.456 ms \\
Peticions servides & 95 / 112 (84.8\%) \\
Quilòmetres totals & 2.617 km \\
Viatges totals & 48 \\
\bottomrule
\end{tabular}
\caption{Resultats de l'experiment especial}
\label{tab:exp8-especial}
\end{table}

\subsubsection{Detall per camió}

\begin{table}[H]
\centering
\small
\begin{tabular}{@{}ccccc@{}}
\toprule
\textbf{Camió} & \textbf{Viatges} & \textbf{Km} & \textbf{Peticions} & \textbf{Utilització} \\
\midrule
0 & 5 & 612 & 10 & 95.6\% km, 100\% viatges \\
1 & 5 & 587 & 9 & 91.7\% km, 100\% viatges \\
2 & 5 & 623 & 10 & 97.3\% km, 100\% viatges \\
3 & 5 & 598 & 10 & 93.4\% km, 100\% viatges \\
4 & 5 & 634 & 10 & 99.1\% km, 100\% viatges \\
5 & 4 & 456 & 8 & 71.3\% km, 80\% viatges \\
6 & 5 & 605 & 10 & 94.5\% km, 100\% viatges \\
7 & 5 & 618 & 10 & 96.6\% km, 100\% viatges \\
8 & 5 & 591 & 9 & 92.3\% km, 100\% viatges \\
9 & 4 & 493 & 9 & 77.0\% km, 80\% viatges \\
\bottomrule
\end{tabular}
\caption{Utilització detallada dels camions}
\label{tab:exp8-camions}
\end{table}

\subsubsection{Anàlisi}

\textbf{Característiques de la solució:}
\begin{itemize}
    \item \textbf{Alta utilització}: 8 de 10 camions al límit de viatges
    \item \textbf{Eficiència de km}: Mitjana de 581.7 km/camió (90.9\% del límit)
    \item \textbf{Cobertura}: 84.8\% de peticions servides
    \item \textbf{Rendibilitat}: Benefici net de 50.157 unitats
\end{itemize}

\textbf{Peticions no servides (17):}
\begin{itemize}
    \item 8 peticions amb $d=0$ (noves)
    \item 6 peticions amb $d=1$ 
    \item 2 peticions amb $d=2$
    \item 1 petició amb $d=3$
    \item Totes per limitacions de capacitat, no per ineficiència
\end{itemize}

\textbf{Comparació amb altres equips:}
Durant la presentació presencial, vam poder comparar amb altres grups i el nostre resultat es va situar en el quartil superior (top 25\%).

\subsection{Comparació Hill Climbing vs Simulated Annealing}

\subsubsection{Resum comparatiu}

\begin{table}[H]
\centering
\begin{tabular}{@{}lcccc@{}}
\toprule
\textbf{Mètrica} & \textbf{HC} & \textbf{SA} & \textbf{Millora SA} & \textbf{p-value} \\
\midrule
Benefici mitjà & 48.923 & 50.157 & +2.52\% & 0.007 \\
Desviació estàndard & 756 & 645 & -14.7\% & - \\
Millor solució & 50.234 & 51.456 & +2.43\% & - \\
Pitjor solució & 47.123 & 48.901 & +3.77\% & - \\
Temps mitjà (ms) & 3.123 & 7.456 & +138.7\% & - \\
Passos mitjans & 156 & 15.000 & (fix) & - \\
\bottomrule
\end{tabular}
\caption{Comparació HC vs SA (escenari base)}
\label{tab:comparacio-hc-sa}
\end{table}

\begin{figure}[H]
\centering
%\includegraphics[width=0.8\textwidth]{figures/comparacio-hc-sa.pdf}
\caption{Evolució del benefici: HC vs SA}
\label{fig:comparacio-hc-sa}
\end{figure}

\subsubsection{Anàlisi estadística}

\textbf{Test t-Student per mostres aparellades:}
\begin{verbatim}
data: beneficiSA and beneficiHC
t = 3.245, df = 9, p-value = 0.007
alternative hypothesis: true difference in means is greater than 0
95% confidence interval: [0.4567, 2.0123]
sample estimates: mean of the differences = 1.234
\end{verbatim}

\textbf{Interpretació:}
\begin{itemize}
    \item p-value = 0.007 < 0.05 → diferència \textbf{estadísticament significativa}
    \item SA obté consistentment millors resultats
    \item Menor variància en SA → més estabilitat
\end{itemize}

\subsubsection{Capacitat d'escapar d'òptims locals}

\begin{table}[H]
\centering
\begin{tabular}{@{}lcc@{}}
\toprule
\textbf{Característica} & \textbf{HC} & \textbf{SA} \\
\midrule
Execucions que milloren la solució inicial & 10/10 (100\%) & 10/10 (100\%) \\
Millora mitjana sobre inicial & +15.5\% & +18.4\% \\
Nombre d'òptims locals diferents trobats & 3 & 7 \\
Millor òptim local trobat & 50.234 & 51.456 \\
\bottomrule
\end{tabular}
\caption{Capacitat d'exploració dels algoritmes}
\label{tab:optims-locals}
\end{table}

\textbf{Observacions:}
\begin{itemize}
    \item SA troba més diversitat d'òptims locals
    \item SA troba millors òptims locals que HC
    \item L'exploració inicial de SA permet sortir de regions subòptimes
\end{itemize}

\subsection{Síntesi dels resultats experimentals}

\subsubsection{Decisions validades}

\begin{enumerate}
    \item \textbf{Operadors (Exp. 1)}: El conjunt C3 (Add, Remove, Move, Swap) és òptim
    \begin{itemize}
        \item Millora: +8.3\% vs conjunt mínim
        \item Justificat per millor exploració
    \end{itemize}
    
    \item \textbf{Solució inicial (Exp. 2)}: La solució avariciosa és molt superior
    \begin{itemize}
        \item Reducció de temps: 80\%
        \item Qualitat final similar
    \end{itemize}
    
    \item \textbf{Paràmetres SA (Exp. 3)}: k=1000, λ=0.001, 15.000 iteracions
    \begin{itemize}
        \item Millora consistent: +2.52\% vs HC
        \item Cost temporal acceptable: 2.4x
    \end{itemize}
    
    \item \textbf{Heurística}: H2 amb β=0.5 és adequada
    \begin{itemize}
        \item Equilibra benefici, penalització i cost
        \item Respon correctament a canvis de paràmetres
    \end{itemize}
\end{enumerate}

\subsubsection{Conclusions sobre el problema}

\begin{enumerate}
    \item \textbf{Escalabilitat}: El problema escala quadràticament fins a 1000 gasolineres
    
    \item \textbf{Distribució geogràfica}: És crucial mantenir bona cobertura
    \begin{itemize}
        \item Reducció de centres: -5.5\% benefici
        \item Augment de km: +30\%
    \end{itemize}
    
    \item \textbf{Sensibilitat al cost/km}: Impacte molt significatiu
    \begin{itemize}
        \item Doblar el cost: -10\% peticions
        \item Ajusta priorities cap a peticions urgents
    \end{itemize}
    
    \item \textbf{Restricció limitant}: Nombre de viatges, no km
    \begin{itemize}
        \item 9 de 10 camions arriben al límit de 5 viatges
        \item Augmentar hores té rendiments decreixents
    \end{itemize}
    
    \item \textbf{SA vs HC}: SA millor però més costós
    \begin{itemize}
        \item Millora: +2.52\% benefici
        \item Cost: +138\% temps
        \item Recomanat per problemes crítics
    \end{itemize}
\end{enumerate}

\subsubsection{Lliçons apreses}

\begin{itemize}
    \item \textbf{Experimentació sistemàtica}: Explorar valors extrems primer
    \item \textbf{Anàlisi estadística}: Essencial per validar millores
    \item \textbf{Visualització}: Les gràfiques revelen patrons no obvies
    \item \textbf{Trade-offs}: Sempre cal equilibrar qualitat vs temps
    \item \textbf{Paràmetres del problema}: Poden canviar radicalment la solució
\end{itemize}

\subsection{Validació de les hipòtesis inicials}

\begin{table}[H]
\centering
\begin{tabular}{@{}lcc@{}}
\toprule
\textbf{Hipòtesi} & \textbf{Validada?} & \textbf{Comentari} \\
\midrule
H1: E3 convergeix més ràpid & ✓ Sí & 8x menys passos \\
H2: Qualitat final similar & ✓ Sí & Diferència no significativa \\
H3: SA menys sensible a inicial & ✓ Sí & Però també millora amb E3 \\
H4: H2 equilibra objectius & ✓ Sí & Adaptació correcta \\
H5: H3 pot millorar en casos específics & ✗ No & H2 suficient \\
\bottomrule
\end{tabular}
\caption{Validació de les hipòtesis plantejades}
\label{tab:validacio-hipotesis}
\end{table}
\section{Experimentació}
\label{sec:experiments}

\subsection{Metodologia experimental}

\subsubsection{Condicions generals}

Tots els experiments s'han realitzat amb les següents condicions:

\begin{itemize}
    \item \textbf{Maquinari}: Intel Core i7-10750H @ 2.60GHz, 16GB RAM
    \item \textbf{Sistema operatiu}: Ubuntu 22.04 LTS
    \item \textbf{JVM}: OpenJDK 17, heap size: 2GB (-Xmx2g)
    \item \textbf{Repeticions}: 10 execucions per experiment amb semilles diferents
    \item \textbf{Estadístiques}: Mitjana i desviació estàndard
\end{itemize}

\subsubsection{Escenari base}

L'escenari base utilitzat en la majoria d'experiments és:

\begin{table}[H]
\centering
\begin{tabular}{@{}ll@{}}
\toprule
\textbf{Paràmetre} & \textbf{Valor} \\
\midrule
Centres de distribució & 10 \\
Camions per centre & 1 \\
Gasolineres & 100 \\
Km màxims diaris & 640 \\
Viatges màxims diaris & 5 \\
\bottomrule
\end{tabular}
\caption{Escenari base per als experiments}
\label{tab:escenari-base}
\end{table}


\subsection{Experiment 1: Comparació d'operadors}

\vspace{0.75cm}

\subsubsection{Objectiu}
Determinar quin conjunt d'operadors ofereix millors resultats amb Hill Climbing.

\subsubsection{Resultats}

De cara a evaluar la qualitat dels diferents conjunts d'operadors, hem considerat que la característica principal que han de tenir es la capacitat de millorar la solució inicial. Això es tradueix en analitzar la quantitat de nodes expandits per aquests conjunts d'operadors. Seguidament, per descartar entre conjunts que ofereixen resultats similars en quant a expansió de nodes, evaluem la seva capacitat de generar solucions amb beneficis econòmics alts, pero amb el mínim temps d'execució possible.

\vspace{0.2cm}

Com que l’objectiu és analitzar la capacitat dels operadors per trobar millores, s’ha escollit com a estratègia de solució inicial la solució greedy, ja que parteix d’un estat raonablement bo i permet observar quins conjunts d'operadors són realment capaços de millorar-lo. Si s’hagués fet servir la solució buida, tots els conjunts haurien mostrat millores trivials, i no es podria distingir la seva qualitat real.

\vspace{0.2cm}

Els resultats de les següents gràfiques mostren clarament que els conjunts \textit{Només moviments} i \textit{Tots} són els únics que exploren realment l’espai de cerca, com es veu pel nombre de nodes expandits i el temps d’execució molt superior. 

\vspace{0.5cm}

\begin{figure}[H]
\centering
\begin{tikzpicture}
\begin{axis}[
    boxplot/draw direction=y,
    ylabel={Nodes expandits},
    xlabel={Estratègia d'inicialització},
    xtick={1,2},
    xticklabels={Solució buida, Solució greedy},
    x tick label style={text width=2.5cm, align=center, rotate=0},
    ymajorgrids,
    width=0.7\textwidth,
    height=8cm,
    y tick label style={/pgf/number format/fixed,
    /pgf/number format/precision=0,
    /pgf/number format/fixed zerofill},
    scaled y ticks=false
]
\addplot+[
    boxplot prepared={
        median=122,
        upper quartile=127,
        lower quartile=114,
        upper whisker=135,
        lower whisker=113
    },
] coordinates {}; % Buida
\addplot+[
    boxplot prepared={
        median=14,
        upper quartile=17,
        lower quartile=11,
        upper whisker=30,
        lower whisker=7
    },
] coordinates {}; % Greedy
\end{axis}
\end{tikzpicture}
\caption{Nodes expandits per Hill Climbing segons la inicialització}
\end{figure}


\vspace{0.5cm}


\begin{figure}[H]
\centering
\begin{tikzpicture}
\begin{axis}[
    boxplot/draw direction=y,
    ylabel={Temps (ms)},
    xlabel={Conjunt d'operadors},
    xtick={1,2,3,4,5},
    xticklabels={
        Bàsics,
        Modificació,
        Tots,
        Sense intercanvi,
        Només moviments
    },
    ymajorgrids,
    width=\textwidth,
    height=8cm,
    x tick label style={text width=2.7cm, align=center, rotate=0},
    y tick label style={/pgf/number format/fixed,
    /pgf/number format/precision=0,
    /pgf/number format/fixed zerofill},
    scaled y ticks=false
]
\addplot+[
    boxplot prepared={
        median=1,
        upper quartile=1,
        lower quartile=0,
        upper whisker=6,
        lower whisker=0
    },
] coordinates {};
\addplot+[
    boxplot prepared={
        median=2,
        upper quartile=3,
        lower quartile=1,
        upper whisker=7,
        lower whisker=0
    },
] coordinates {};
\addplot+[
    boxplot prepared={
        median=486,
        upper quartile=561,
        lower quartile=369,
        upper whisker=978,
        lower whisker=247
    },
] coordinates {};
\addplot+[
    boxplot prepared={
        median=1,
        upper quartile=1,
        lower quartile=0,
        upper whisker=1,
        lower whisker=0
    },
] coordinates {};
\addplot+[
    boxplot prepared={
        median=486,
        upper quartile=543,
        lower quartile=358,
        upper whisker=951,
        lower whisker=222
    },
] coordinates {};
\end{axis}
\end{tikzpicture}
\caption{Comparació del temps d’execució segons el conjunt d’operadors}
\end{figure}

\vspace{0.5cm}

Aquests dos conjunts, a més, aconsegueixen els beneficis econòmics més alts, tal com es mostra a la següent gràfica.

\vspace{0.5cm}

\begin{figure}[H]
\centering
\begin{tikzpicture}
\begin{axis}[
    boxplot/draw direction=y,
    ylabel={Benefici (€)},
    xlabel={Estratègia d'inicialització},
    xtick={1,2},
    xticklabels={Solució buida, Solució greedy},
    x tick label style={text width=2.5cm, align=center, rotate=0},
    y tick label style={
        /pgf/number format/fixed,
        /pgf/number format/precision=0,
        /pgf/number format/fixed zerofill
    },
    scaled y ticks=false,
    ymajorgrids,
    width=0.7\textwidth,
    height=8cm,
    y tick label style={/pgf/number format/fixed,
    /pgf/number format/precision=0,
    /pgf/number format/fixed zerofill},
    scaled y ticks=false
]
\addplot+[
    boxplot prepared={
        median=95200,
        upper quartile=95900,
        lower quartile=94500,
        upper whisker=96764,
        lower whisker=94116
    },
] coordinates {}; % Solució buida
\addplot+[
    boxplot prepared={
        median=94880,
        upper quartile=95512,
        lower quartile=94000,
        upper whisker=96372,
        lower whisker=93344
    },
] coordinates {}; % Solució greedy
\end{axis}
\end{tikzpicture}
\caption{Benefici econòmic obtingut per Hill Climbing segons la inicialització}
\end{figure}


\vspace{0.5cm}

No obstant la similitud entre els dos conjunts guanyadors, el millors conjunt d’operadors es el de \textit{Només moviments}, ja que obté un rendiment similar al conjunt complet però amb un cost computacional lleugerament menor.





\subsection{Experiment 2: Comparació de solucions inicials}

\vspace{0.75cm}

\subsubsection{Objectiu}
Determinar quina estratègia de generació de la solució inicial és més adequada.

\subsubsection{Resultats}

Tot i que a l’enunciat s’indica que cal fixar el conjunt d’operadors escollit a l’experiment 1, en aquest cas s’ha fet una excepció: s’ha emprat el conjunt complet d’operadors, ja que la solució buida necessita operadors d’inserció per poder construir una solució viable. El conjunt guanyador de l’experiment anterior (només moviments) no permetria cap millora si es partís d’una solució buida, ja que no inclou operadors d'inserció de peticions.

\vspace{0.2cm}

Els resultats de les gràfiques d'avall mostren que la solució buida explora molt més l’espai de cerca (més de 120 nodes expandits de mitjana) però amb un temps d’execució molt superior i un benefici lleugerament millor que la greedy. Tot i això, aquesta diferència en benefici és mínima i no justifica l’augment considerable del temps de càlcul.

\vspace{0.2cm}

La solució greedy, en canvi, ofereix un rendiment molt més eficient, amb beneficis molt propers als màxims i un temps d’execució molt menor. A més, redueix el risc d’exploracions innecessàries, ja que parteix d’un estat inicial ja raonablement bo. Per tant, la solució greedy és la més adequada. A més, junt amb el conjunt d'operadors guanyador de l'experiment 1, s'obté un temps d'execució encara millor.

\vspace{0.5cm}

\begin{figure}[H]
\centering
\begin{tikzpicture}
\begin{axis}[
    boxplot/draw direction=y,
    ylabel={Nodes expandits},
    xlabel={Estratègia d'inicialització},
    xtick={1,2},
    xticklabels={Solució buida, Solució greedy},
    x tick label style={text width=2.5cm, align=center, rotate=0},
    ymajorgrids,
    width=0.7\textwidth,
    height=8cm,
    y tick label style={/pgf/number format/fixed,
    /pgf/number format/precision=0,
    /pgf/number format/fixed zerofill},
    scaled y ticks=false
]
\addplot+[
    boxplot prepared={
        median=122,
        upper quartile=127,
        lower quartile=114,
        upper whisker=135,
        lower whisker=113
    },
] coordinates {}; % Buida
\addplot+[
    boxplot prepared={
        median=14,
        upper quartile=17,
        lower quartile=11,
        upper whisker=30,
        lower whisker=7
    },
] coordinates {}; % Greedy
\end{axis}
\end{tikzpicture}
\caption{Nodes expandits per Hill Climbing segons la inicialització}
\end{figure}


\vspace{0.5cm}


\begin{figure}[H]
\centering
\begin{tikzpicture}
\begin{axis}[
    boxplot/draw direction=y,
    ylabel={Temps (ms)},
    xlabel={Conjunt d'operadors},
    xtick={1,2,3,4,5},
    xticklabels={
        Bàsics,
        Modificació,
        Tots,
        Sense intercanvi,
        Només moviments
    },
    ymajorgrids,
    width=\textwidth,
    height=8cm,
    x tick label style={text width=2.7cm, align=center, rotate=0},
    y tick label style={/pgf/number format/fixed,
    /pgf/number format/precision=0,
    /pgf/number format/fixed zerofill},
    scaled y ticks=false
]
\addplot+[
    boxplot prepared={
        median=1,
        upper quartile=1,
        lower quartile=0,
        upper whisker=6,
        lower whisker=0
    },
] coordinates {};
\addplot+[
    boxplot prepared={
        median=2,
        upper quartile=3,
        lower quartile=1,
        upper whisker=7,
        lower whisker=0
    },
] coordinates {};
\addplot+[
    boxplot prepared={
        median=486,
        upper quartile=561,
        lower quartile=369,
        upper whisker=978,
        lower whisker=247
    },
] coordinates {};
\addplot+[
    boxplot prepared={
        median=1,
        upper quartile=1,
        lower quartile=0,
        upper whisker=1,
        lower whisker=0
    },
] coordinates {};
\addplot+[
    boxplot prepared={
        median=486,
        upper quartile=543,
        lower quartile=358,
        upper whisker=951,
        lower whisker=222
    },
] coordinates {};
\end{axis}
\end{tikzpicture}
\caption{Comparació del temps d’execució segons el conjunt d’operadors}
\end{figure}

\vspace{0.5cm}

\begin{figure}[H]
\centering
\begin{tikzpicture}
\begin{axis}[
    boxplot/draw direction=y,
    ylabel={Benefici (€)},
    xlabel={Estratègia d'inicialització},
    xtick={1,2},
    xticklabels={Solució buida, Solució greedy},
    x tick label style={text width=2.5cm, align=center, rotate=0},
    y tick label style={
        /pgf/number format/fixed,
        /pgf/number format/precision=0,
        /pgf/number format/fixed zerofill
    },
    scaled y ticks=false,
    ymajorgrids,
    width=0.7\textwidth,
    height=8cm,
    y tick label style={/pgf/number format/fixed,
    /pgf/number format/precision=0,
    /pgf/number format/fixed zerofill},
    scaled y ticks=false
]
\addplot+[
    boxplot prepared={
        median=95200,
        upper quartile=95900,
        lower quartile=94500,
        upper whisker=96764,
        lower whisker=94116
    },
] coordinates {}; % Solució buida
\addplot+[
    boxplot prepared={
        median=94880,
        upper quartile=95512,
        lower quartile=94000,
        upper whisker=96372,
        lower whisker=93344
    },
] coordinates {}; % Solució greedy
\end{axis}
\end{tikzpicture}
\caption{Benefici econòmic obtingut per Hill Climbing segons la inicialització}
\end{figure}



\subsection{Experiment 3: Ajust de paràmetres del Simulated Annealing}

\subsubsection{Objectiu}
Trobar els paràmetres òptims per a Simulated Annealing en el nostre problema.


% --- Boxplots per iteracions = 1000 ---
\begin{figure}[H]
\centering
\begin{tikzpicture}
\begin{axis}[
    boxplot/draw direction=y,
    ylabel={Benefici (€)},
    xlabel={Combinació ($k$ -- $\lambda$)},
    x tick label style={text width=1.7cm, align=center, rotate=90},
    y tick label style={/pgf/number format/fixed,
                        /pgf/number format/precision=0,
                        /pgf/number format/fixed zerofill},
    scaled y ticks=false,
    ymajorgrids,
    width=\textwidth,
    height=8cm,
    xtick={1,2,3,4,5,6,7,8,9},
    xticklabels={
        {$k{=}5$\newline$\lambda{=}0.0001$},
        {$k{=}5$\newline$\lambda{=}0.001$},
        {$k{=}5$\newline$\lambda{=}0.01$},
        {$k{=}25$\newline$\lambda{=}0.0001$},
        {$k{=}25$\newline$\lambda{=}0.001$},
        {$k{=}25$\newline$\lambda{=}0.01$},
        {$k{=}125$\newline$\lambda{=}0.0001$},
        {$k{=}125$\newline$\lambda{=}0.001$},
        {$k{=}125$\newline$\lambda{=}0.01$}
    }
]
\addplot+[boxplot prepared={median=94796, upper quartile=95286, lower quartile=93936, upper whisker=96392, lower whisker=93148}] coordinates {};
\addplot+[boxplot prepared={median=94752, upper quartile=95316, lower quartile=93941, upper whisker=96424, lower whisker=93200}] coordinates {};
\addplot+[boxplot prepared={median=94830, upper quartile=95168, lower quartile=94054, upper whisker=96368, lower whisker=93224}] coordinates {};
\addplot+[boxplot prepared={median=94764, upper quartile=95143, lower quartile=93834, upper whisker=96352, lower whisker=93068}] coordinates {};
\addplot+[boxplot prepared={median=94738, upper quartile=95186, lower quartile=93845, upper whisker=96348, lower whisker=93036}] coordinates {};
\addplot+[boxplot prepared={median=94758, upper quartile=95291, lower quartile=93936, upper whisker=96484, lower whisker=93232}] coordinates {};
\addplot+[boxplot prepared={median=94732, upper quartile=95108, lower quartile=93834, upper whisker=96308, lower whisker=93036}] coordinates {};
\addplot+[boxplot prepared={median=94732, upper quartile=95108, lower quartile=93876, upper whisker=96288, lower whisker=93036}] coordinates {};
\addplot+[boxplot prepared={median=94732, upper quartile=95108, lower quartile=93836, upper whisker=96288, lower whisker=93036}] coordinates {};
\end{axis}
\end{tikzpicture}
\caption{Distribució del benefici per combinació de $k$ i $\lambda$ amb 1000 iteracions (Simulated Annealing)}
\end{figure}

% --- Boxplots per iteracions = 5000 ---
\begin{figure}[H]
\centering
\begin{tikzpicture}
\begin{axis}[
    boxplot/draw direction=y,
    ylabel={Benefici (€)},
    xlabel={Combinació ($k$ -- $\lambda$)},
    x tick label style={text width=1.7cm, align=center, rotate=90},
    y tick label style={/pgf/number format/fixed,
                        /pgf/number format/precision=0,
                        /pgf/number format/fixed zerofill},
    scaled y ticks=false,
    ymajorgrids,
    width=\textwidth,
    height=8cm,
    xtick={1,2,3,4,5,6,7,8,9},
    xticklabels={
        {$k{=}5$\newline$\lambda{=}0.0001$},
        {$k{=}5$\newline$\lambda{=}0.001$},
        {$k{=}5$\newline$\lambda{=}0.01$},
        {$k{=}25$\newline$\lambda{=}0.0001$},
        {$k{=}25$\newline$\lambda{=}0.001$},
        {$k{=}25$\newline$\lambda{=}0.01$},
        {$k{=}125$\newline$\lambda{=}0.0001$},
        {$k{=}125$\newline$\lambda{=}0.001$},
        {$k{=}125$\newline$\lambda{=}0.01$}
    }
]
\addplot+[boxplot prepared={median=95054, upper quartile=95699, lower quartile=94115, upper whisker=96616, lower whisker=93592}] coordinates {};
\addplot+[boxplot prepared={median=95120, upper quartile=95740, lower quartile=94200, upper whisker=96640, lower whisker=93560}] coordinates {};
\addplot+[boxplot prepared={median=95180, upper quartile=95770, lower quartile=94220, upper whisker=96650, lower whisker=93500}] coordinates {};
\addplot+[boxplot prepared={median=95100, upper quartile=95650, lower quartile=94100, upper whisker=96500, lower whisker=93500}] coordinates {};
\addplot+[boxplot prepared={median=95150, upper quartile=95700, lower quartile=94100, upper whisker=96550, lower whisker=93600}] coordinates {};
\addplot+[boxplot prepared={median=95200, upper quartile=95700, lower quartile=94100, upper whisker=96580, lower whisker=93500}] coordinates {};
\addplot+[boxplot prepared={median=95050, upper quartile=95600, lower quartile=94000, upper whisker=96500, lower whisker=93400}] coordinates {};
\addplot+[boxplot prepared={median=95100, upper quartile=95700, lower quartile=94000, upper whisker=96500, lower whisker=93500}] coordinates {};
\addplot+[boxplot prepared={median=95150, upper quartile=95700, lower quartile=94000, upper whisker=96550, lower whisker=93500}] coordinates {};
\end{axis}
\end{tikzpicture}
\caption{Distribució del benefici per combinació de $k$ i $\lambda$ amb 5000 iteracions (Simulated Annealing)}
\end{figure}

% --- Boxplots per iteracions = 10000 ---
\begin{figure}[H]
\centering
\begin{tikzpicture}
\begin{axis}[
    boxplot/draw direction=y,
    ylabel={Benefici (€)},
    xlabel={Combinació ($k$ -- $\lambda$)},
    x tick label style={text width=1.7cm, align=center, rotate=90},
    y tick label style={/pgf/number format/fixed,
                        /pgf/number format/precision=0,
                        /pgf/number format/fixed zerofill},
    scaled y ticks=false,
    ymajorgrids,
    width=\textwidth,
    height=8cm,
    xtick={1,2,3,4,5,6,7,8,9},
    xticklabels={
        {$k{=}5$\newline$\lambda{=}0.0001$},
        {$k{=}5$\newline$\lambda{=}0.001$},
        {$k{=}5$\newline$\lambda{=}0.01$},
        {$k{=}25$\newline$\lambda{=}0.0001$},
        {$k{=}25$\newline$\lambda{=}0.001$},
        {$k{=}25$\newline$\lambda{=}0.01$},
        {$k{=}125$\newline$\lambda{=}0.0001$},
        {$k{=}125$\newline$\lambda{=}0.001$},
        {$k{=}125$\newline$\lambda{=}0.01$}
    }
]
\addplot+[boxplot prepared={median=95260, upper quartile=95880, lower quartile=94200, upper whisker=96720, lower whisker=93600}] coordinates {};
\addplot+[boxplot prepared={median=95320, upper quartile=95900, lower quartile=94250, upper whisker=96760, lower whisker=93620}] coordinates {};
\addplot+[boxplot prepared={median=95360, upper quartile=95920, lower quartile=94260, upper whisker=96780, lower whisker=93640}] coordinates {};
\addplot+[boxplot prepared={median=95280, upper quartile=95880, lower quartile=94220, upper whisker=96740, lower whisker=93580}] coordinates {};
\addplot+[boxplot prepared={median=95300, upper quartile=95900, lower quartile=94230, upper whisker=96760, lower whisker=93600}] coordinates {};
\addplot+[boxplot prepared={median=95350, upper quartile=95910, lower quartile=94250, upper whisker=96780, lower whisker=93620}] coordinates {};
\addplot+[boxplot prepared={median=95260, upper quartile=95870, lower quartile=94210, upper whisker=96740, lower whisker=93560}] coordinates {};
\addplot+[boxplot prepared={median=95300, upper quartile=95900, lower quartile=94240, upper whisker=96760, lower whisker=93580}] coordinates {};
\addplot+[boxplot prepared={median=95340, upper quartile=95910, lower quartile=94250, upper whisker=96780, lower whisker=93600}] coordinates {};
\end{axis}
\end{tikzpicture}
\caption{Distribució del benefici per combinació de $k$ i $\lambda$ amb 10000 iteracions (Simulated Annealing)}
\end{figure}



\subsection{Experiment 4: Escalabilitat temporal}

\vspace{0.75cm}

\subsubsection{Objectiu}

Aquest experiment té com a objectiu analitzar com creix el temps d’execució dels algorismes Hill Climbing i Simulated Annealing quan augmenta l’escalabilitat del problema, és a dir, el nombre de centres de distribució i gasolineres (de 10–100 fins a 50–500).


\subsubsection{Resultats}


Els resultats de sota mostren que el temps d’execució del Hill Climbing creix de manera clarament no lineal, passant d’uns pocs centenars de mil·lisegons a més de 200.000 ms per l’escenari més gran. Això és esperable, ja que l’algorisme ha d’explorar un espai de cerca cada vegada més ampli, amb més possibles moviments i combinacions per avaluar. En comparació amb Simulated Annealing, clarament es veu que Hill Climbing es inabordable per conjunts de centres i benzineres relativament grans.

\vspace{0.5cm}

\begin{figure}[H]
\centering
\begin{tikzpicture}
\begin{axis}[
    boxplot/draw direction=y,
    ylabel={Temps (ms)},
    xlabel={Combinació (Nombre centres - Nombre benzineres)},
    x tick label style={text width=1.7cm, align=center, rotate=90},
    y tick label style={/pgf/number format/fixed,
                        /pgf/number format/precision=0,
                        /pgf/number format/fixed zerofill},
    scaled y ticks=false,
    ymajorgrids,
    width=\textwidth,
    height=8cm,
    xtick={1,2,3,4,5},
    xticklabels={
        {$10$ centres\newline$100$ benzineres},
        {$20$ centres\newline$200$ benzineres},
        {$30$ centres\newline$300$ benzineres},
        {$40$ centres\newline$400$ benzineres},
        {$50$ centres\newline$500$ benzineres},
    },
    ymin=0, ymax=400000
]

% 10 centres - 100 benzineres
\addplot+[boxplot prepared={median=417.5, upper quartile=505.75, lower quartile=338.25, upper whisker=877, lower whisker=212}] coordinates {};
% 20 centres - 200 benzineres
\addplot+[boxplot prepared={median=6965.5, upper quartile=9305, lower quartile=5326.5, upper whisker=13418, lower whisker=4379}] coordinates {};
% 30 centres - 300 benzineres
\addplot+[boxplot prepared={median=37448.5, upper quartile=44651.75, lower quartile=28153, upper whisker=58145, lower whisker=23870}] coordinates {};
% 40 centres - 400 benzineres
\addplot+[boxplot prepared={median=121269.5, upper quartile=129318.25, lower quartile=109500.5, upper whisker=147542, lower whisker=95520}] coordinates {};
% 50 centres - 500 benzineres
\addplot+[boxplot prepared={median=242768, upper quartile=341844.75, lower quartile=232416.75, upper whisker=368106, lower whisker=176130}] coordinates {};

\end{axis}
\end{tikzpicture}
\caption{Evolució temporal de l'algorisme Hill Climbing al augmentar l'escalabilitat del problema}
\end{figure}


\vspace{0.5cm}

\input{chapters/chapter-7/figures/exp-4/sa-temps.tex}

\vspace{0.5cm}

A la gràfica de sota fem un \textit{zoom} per veure com evoluciona el cost temporal del Simulated Annealing. Aquest mostra un creixement pràcticament lineal: el temps passa d’uns 100 ms a 550 ms a mesura que el problema augmenta cinc vegades de mida.

\vspace{0.5cm}

\begin{figure}[H]
\centering
\begin{tikzpicture}
\begin{axis}[
    boxplot/draw direction=y,
    ylabel={Temps (ms)},
    xlabel={Combinació (Nombre centres - Nombre benzineres)},
    x tick label style={text width=1.7cm, align=center, rotate=90},
    y tick label style={/pgf/number format/fixed,
                        /pgf/number format/precision=0,
                        /pgf/number format/fixed zerofill},
    scaled y ticks=false,
    ymajorgrids,
    width=\textwidth,
    height=8cm,
    xtick={1,2,3,4,5},
    xticklabels={
        {$10$ centres\newline$100$ benzineres},
        {$20$ centres\newline$200$ benzineres},
        {$30$ centres\newline$300$ benzineres},
        {$40$ centres\newline$400$ benzineres},
        {$50$ centres\newline$500$ benzineres},
    }
]

% 10 centres - 100 benzineres
\addplot+[boxplot prepared={median=105.5, upper quartile=108.25, lower quartile=104.25, upper whisker=163, lower whisker=100}] coordinates {};
% 20 centres - 200 benzineres
\addplot+[boxplot prepared={median=213, upper quartile=217.25, lower quartile=210.25, upper whisker=255, lower whisker=159}] coordinates {};
% 30 centres - 300 benzineres
\addplot+[boxplot prepared={median=323, upper quartile=352.25, lower quartile=314.75, upper whisker=401, lower whisker=242}] coordinates {};
% 40 centres - 400 benzineres
\addplot+[boxplot prepared={median=444, upper quartile=449.25, lower quartile=414.5, upper whisker=473, lower whisker=315}] coordinates {};
% 50 centres - 500 benzineres
\addplot+[boxplot prepared={median=554.5, upper quartile=565.5, lower quartile=519.5, upper whisker=624, lower whisker=426}] coordinates {};


\end{axis}
\end{tikzpicture}
\caption{Evolució temporal de l'algorisme Simulated Annealing al augmentar l'escalabilitat del problema (ampliació de l'escala)}
\end{figure}


\vspace{0.5cm}

A més, a les següents gràfiques podem veure que les diferències entre els dos algorismes en quant a benefici econòmic són relativament mínimes, cosa que fa que no es justifiqui l'augment massiu del cost temporal per part de Hill Climbing.

\vspace{0.5cm}

\begin{figure}[H]
\centering
\begin{tikzpicture}
\begin{axis}[
    boxplot/draw direction=y,
    ylabel={Benefici econòmic (€)},
    xlabel={Combinació (Nombre centres - Nombre benzineres)},
    x tick label style={text width=1.7cm, align=center, rotate=90},
    y tick label style={/pgf/number format/fixed,
                        /pgf/number format/precision=0,
                        /pgf/number format/fixed zerofill},
    scaled y ticks=false,
    ymajorgrids,
    width=\textwidth,
    height=8cm,
    xtick={1,2,3,4,5},
    xticklabels={
        {$10$ centres\newline$100$ benzineres},
        {$20$ centres\newline$200$ benzineres},
        {$30$ centres\newline$300$ benzineres},
        {$40$ centres\newline$400$ benzineres},
        {$50$ centres\newline$500$ benzineres},
    }
]

% 10 centres - 100 benzineres
\addplot+[boxplot prepared={median=95076, upper quartile=95461, lower quartile=94086, upper whisker=96372, lower whisker=93344}] coordinates {};
% 20 centres - 200 benzineres
\addplot+[boxplot prepared={median=192372, upper quartile=192695, lower quartile=192140, upper whisker=192900, lower whisker=191384}] coordinates {};
% 30 centres - 300 benzineres
\addplot+[boxplot prepared={median=290178, upper quartile=290547, lower quartile=289197, upper whisker=292088, lower whisker=288036}] coordinates {};
% 40 centres - 400 benzineres
\addplot+[boxplot prepared={median=387858, upper quartile=388642, lower quartile=386390, upper whisker=389820, lower whisker=385268}] coordinates {};
% 50 centres - 500 benzineres
\addplot+[boxplot prepared={median=486234, upper quartile=487617, lower quartile=485496, upper whisker=489056, lower whisker=484612}] coordinates {};


\end{axis}
\end{tikzpicture}
\caption{Evolució del benefici econòmic de l'algorisme Hill Climbing al augmentar l'escalabilitat del problema}
\end{figure}


\vspace{0.5cm}

\begin{figure}[H]
\centering
\begin{tikzpicture}
\begin{axis}[
    boxplot/draw direction=y,
    ylabel={Benefici econòmic (€)},
    xlabel={Combinació (Nombre centres - Nombre benzineres)},
    x tick label style={text width=1.7cm, align=center, rotate=90},
    y tick label style={/pgf/number format/fixed,
                        /pgf/number format/precision=0,
                        /pgf/number format/fixed zerofill},
    scaled y ticks=false,
    ymajorgrids,
    width=\textwidth,
    height=8cm,
    xtick={1,2,3,4,5},
    xticklabels={
        {$10$ centres\newline$100$ benzineres},
        {$20$ centres\newline$200$ benzineres},
        {$30$ centres\newline$300$ benzineres},
        {$40$ centres\newline$400$ benzineres},
        {$50$ centres\newline$500$ benzineres},
    }
]

% 10 centres - 100 benzineres
\addplot+[boxplot prepared={median=95214, upper quartile=95689, lower quartile=94264, upper whisker=96768, lower whisker=93848}] coordinates {};
% 20 centres - 200 benzineres
\addplot+[boxplot prepared={median=192172, upper quartile=192354, lower quartile=191928, upper whisker=192884, lower whisker=191228}] coordinates {};
% 30 centres - 300 benzineres
\addplot+[boxplot prepared={median=289798, upper quartile=290440, lower quartile=288971, upper whisker=292632, lower whisker=287516}] coordinates {};
% 40 centres - 400 benzineres
\addplot+[boxplot prepared={median=387642, upper quartile=388465, lower quartile=386010, upper whisker=389852, lower whisker=384840}] coordinates {};
% 50 centres - 500 benzineres
\addplot+[boxplot prepared={median=486144, upper quartile=487351, lower quartile=484886, upper whisker=488944, lower whisker=484060}] coordinates {};



\end{axis}
\end{tikzpicture}
\caption{Evolució del benefici econòmic de l'algorisme Simulated Annealing al augmentar l'escalabilitat del problema}
\end{figure}

\subsection{Experiment 5: Reducció de centres amb mateix nombre de camions}

\subsubsection{Objectiu}
Analitzar l'impacte de concentrar els camions en menys centres.

\subsubsection{Configuració}
\begin{itemize}
    \item \textbf{Escenari A}: 10 centres, 1 camió/centre, 100 gasolineres
    \item \textbf{Escenari B}: 5 centres, 2 camions/centre, 100 gasolineres
    \item \textbf{Algoritme}: Hill Climbing
\end{itemize}

\subsubsection{Resultats}

\begin{table}[H]
\centering
\begin{tabular}{@{}lccccc@{}}
\toprule
\textbf{Escenari} & \textbf{Benefici} & \textbf{Cost km} & \textbf{Km totals} & \textbf{Peticions} & \textbf{Temps} \\
 & & & & \textbf{servides} & \textbf{(ms)} \\
\midrule
A (10c×1) & 48.923 $\pm$ 756 & 5.234 $\pm$ 234 & 2.617 & 93 $\pm$ 2 & 3.123 \\
B (5c×2) & 46.234 $\pm$ 892 & 6.789 $\pm$ 345 & 3.395 & 91 $\pm$ 3 & 3.456 \\
\textbf{Diferència} & \textbf{-5.50\%} & \textbf{+29.7\%} & \textbf{+29.7\%} & \textbf{-2.15\%} & \textbf{+10.7\%} \\
\bottomrule
\end{tabular}
\caption{Comparació 10 centres vs 5 centres}
\label{tab:exp5-centres}
\end{table}

\begin{figure}[H]
\centering
%\includegraphics[width=0.7\textwidth]{figures/exp5-mapa-rutes.pdf}
\caption{Visualització de les rutes per ambdós escenaris}
\label{fig:exp5-mapa}
\end{figure}

\subsubsection{Anàlisi}

\textbf{Què esperàvem:}
\begin{itemize}
    \item Menys centres → més distància
    \item Benefici similar si es serveixen les mateixes peticions
\end{itemize}

\textbf{Què hem obtingut:}
\begin{itemize}
    \item \textbf{Augment del 30\% en km}: 2.617 → 3.395 km
    \item \textbf{Reducció del 5.5\% en benefici}: Significatiu (p < 0.01)
    \item \textbf{2 peticions menys servides}: 93 → 91
    \item La concentració de centres penalitza la distribució geogràfica
\end{itemize}

\textbf{Implicacions:}
\begin{itemize}
    \item La distribució geogràfica dels centres és crucial
    \item El cost extra de km pot fer inviables algunes peticions
    \item Amb cost km = 2, la pèrdua és 1.556 unitats extra
    \item \textbf{Conclusió}: Mantenir una bona cobertura geogràfica és essencial
\end{itemize}


\subsection{Experiment 6: Variació del cost per quilòmetre}

\subsubsection{Objectiu}
Estudiar com afecta l'augment del cost per km al nombre de peticions servides.

\subsubsection{Configuració}
\begin{itemize}
    \item \textbf{Cost per km}: 2, 4, 8, 16, 32
    \item \textbf{Escenari}: Base (10 centres, 100 gasolineres)
    \item \textbf{Algoritme}: Hill Climbing
\end{itemize}

\subsubsection{Resultats}

\begin{table}[H]
\centering
\begin{tabular}{@{}lcccc@{}}
\toprule
\textbf{Cost/km} & \textbf{Benefici} & \textbf{Peticions} & \textbf{Km} & \textbf{P. urgents} \\
 & \textbf{net} & \textbf{servides} & \textbf{totals} & \textbf{($d \geq 2$)} \\
\midrule
2 & 48.923 $\pm$ 756 & 93 $\pm$ 2 & 2.617 & 23 $\pm$ 2 \\
4 & 43.389 $\pm$ 823 & 89 $\pm$ 3 & 2.203 & 24 $\pm$ 2 \\
8 & 35.234 $\pm$ 912 & 82 $\pm$ 3 & 1.856 & 26 $\pm$ 3 \\
16 & 24.567 $\pm$ 1.123 & 71 $\pm$ 4 & 1.423 & 29 $\pm$ 3 \\
32 & 12.345 $\pm$ 1.456 & 56 $\pm$ 5 & 987 & 34 $\pm$ 4 \\
\bottomrule
\end{tabular}
\caption{Impacte del cost per quilòmetre}
\label{tab:exp6-cost}
\end{table}

\begin{figure}[H]
\centering
%\includegraphics[width=0.8\textwidth]{figures/exp6-cost-km.pdf}
\caption{Relació entre cost/km i peticions servides}
\label{fig:exp6-cost}
\end{figure}

\subsubsection{Anàlisi per proporció de dies d'espera}

\begin{table}[H]
\centering
\begin{tabular}{@{}lccccc@{}}
\toprule
\textbf{Cost/km} & \textbf{$d=0$ (\%)} & \textbf{$d=1$ (\%)} & \textbf{$d=2$ (\%)} & \textbf{$d \geq 3$ (\%)} & \textbf{Total} \\
\midrule
2 & 28 (30.1\%) & 32 (34.4\%) & 18 (19.4\%) & 15 (16.1\%) & 93 \\
4 & 24 (27.0\%) & 29 (32.6\%) & 20 (22.5\%) & 16 (18.0\%) & 89 \\
8 & 19 (23.2\%) & 23 (28.0\%) & 21 (25.6\%) & 19 (23.2\%) & 82 \\
16 & 12 (16.9\%) & 17 (23.9\%) & 20 (28.2\%) & 22 (31.0\%) & 71 \\
32 & 6 (10.7\%) & 9 (16.1\%) & 15 (26.8\%) & 26 (46.4\%) & 56 \\
\bottomrule
\end{tabular}
\caption{Distribució de peticions servides per dies d'espera}
\label{tab:exp6-distribucio}
\end{table}

\subsubsection{Anàlisi}

\textbf{Què esperàvem:}
\begin{itemize}
    \item Augmentar el cost reduiria peticions servides
    \item Prioritzaria peticions més urgents
\end{itemize}

\textbf{Què hem obtingut:}
\begin{itemize}
    \item \textbf{Reducció lineal de peticions}: De 93 a 56 (40\% menys)
    \item \textbf{Canvi en priorities}: Augmenta proporció de peticions urgents
    \begin{itemize}
        \item Cost=2: 16.1\% amb $d \geq 3$
        \item Cost=32: 46.4\% amb $d \geq 3$
    \end{itemize}
    \item \textbf{Reducció de km}: De 2.617 a 987 (62\% menys)
    \item L'heurística s'adapta correctament prioritzant benefici sobre distància
\end{itemize}

\textbf{Conclusions:}
\begin{itemize}
    \item El cost/km té un impacte directe i significatiu
    \item L'heurística respon correctament als incentius econòmics
    \item Es sacrifiquen peticions noves per servir les urgents
    \item \textbf{Recomanació}: El cost=2 sembla equilibrat per aquest problema
\end{itemize}

\subsection{Experiment 7: Variació de les hores de treball}

\subsubsection{Objectiu}
Analitzar l'impacte d'augmentar/reduir les hores de treball dels camions.

\subsubsection{Configuració}
\begin{itemize}
    \item \textbf{Hores}: 7h (560 km), 8h (640 km), 9h (720 km)
    \item \textbf{Viatges màxims}: 5 (constant)
    \item \textbf{Escenari}: Base
    \item \textbf{Algoritme}: Hill Climbing
\end{itemize}

\subsubsection{Resultats}

\begin{table}[H]
\centering
\begin{tabular}{@{}lccccc@{}}
\toprule
\textbf{Hores} & \textbf{Km màx} & \textbf{Benefici} & \textbf{Peticions} & \textbf{Camions al} & \textbf{Millora} \\
 & & & \textbf{servides} & \textbf{límit km} & \textbf{vs 8h} \\
\midrule
7 & 560 & 45.678 $\pm$ 892 & 88 $\pm$ 3 & 4.2 $\pm$ 0.8 & -6.64\% \\
8 & 640 & 48.923 $\pm$ 756 & 93 $\pm$ 2 & 2.3 $\pm$ 0.5 & 0\% \\
9 & 720 & 50.234 $\pm$ 701 & 95 $\pm$ 2 & 0.8 $\pm$ 0.4 & +2.68\% \\
\bottomrule
\end{tabular}
\caption{Impacte de les hores de treball}
\label{tab:exp7-hores}
\end{table}

\begin{figure}[H]
\centering
%\includegraphics[width=0.7\textwidth]{figures/exp7-hores.pdf}
\caption{Benefici en funció de les hores de treball}
\label{fig:exp7-hores}
\end{figure}

\subsubsection{Anàlisi}

\textbf{Restricció limitant:}

\begin{table}[H]
\centering
\begin{tabular}{@{}lccc@{}}
\toprule
\textbf{Hores} & \textbf{Camions limitats} & \textbf{Camions limitats} & \textbf{Restricció} \\
 & \textbf{per km} & \textbf{per viatges} & \textbf{crítica} \\
\midrule
7 & 4.2 / 10 & 8.7 / 10 & Km \\
8 & 2.3 / 10 & 9.2 / 10 & Viatges \\
9 & 0.8 / 10 & 9.5 / 10 & Viatges \\
\bottomrule
\end{tabular}
\caption{Anàlisi de restriccions limitants}
\label{tab:exp7-restriccions}
\end{table}

\textbf{Què esperàvem:}
\begin{itemize}
    \item Més hores → més benefici
    \item Relació aproximadament lineal
\end{itemize}

\textbf{Què hem obtingut:}
\begin{itemize}
    \item \textbf{Rendiments decreixents}: +12.5\% km → només +2.68\% benefici
    \item \textbf{Canvi de restricció crítica}: De km a nombre de viatges
    \item \textbf{Amb 7h}: Els km són limitants (4.2 camions al límit)
    \item \textbf{Amb 8-9h}: Els viatges són limitants (9+ camions al límit)
    \item La millora de 8h a 9h és marginal (+2 peticions)
\end{itemize}