\subsubsection{Configuració}
\begin{itemize}
    \item \textbf{Centres}: 10
    \item \textbf{Camions per centre}: 1
    \item \textbf{Gasolineres}: 100
    \item \textbf{Semilla}: 1234 (centres i gasolineres)
    \item \textbf{Algoritme}: Simulated Annealing amb paràmetres optimitzats
    \item \textbf{Operadors}: C3 (complet)
    \item \textbf{Solució inicial}: E3 (Greedy)
    \item \textbf{Heurística}: H2
\end{itemize}

\subsubsection{Resultats}

\begin{table}[H]
\centering
\begin{tabular}{@{}lc@{}}
\toprule
\textbf{Mètrica} & \textbf{Valor} \\
\midrule
Benefici de peticions servides & 96.847 \\
Cost de quilòmetres & -5.234 \\
Penalització peticions no servides & -1.456 \\
\midrule
\textbf{Benefici net final} & \textbf{50.157} \\
\midrule
Temps d'execució & 7.456 ms \\
Peticions servides & 95 / 112 (84.8\%) \\
Quilòmetres totals & 2.617 km \\
Viatges totals & 48 \\
\bottomrule
\end{tabular}
\caption{Resultats de l'experiment especial}
\label{tab:exp8-especial}
\end{table}

\subsubsection{Detall per camió}

\begin{table}[H]
\centering
\small
\begin{tabular}{@{}ccccc@{}}
\toprule
\textbf{Camió} & \textbf{Viatges} & \textbf{Km} & \textbf{Peticions} & \textbf{Utilització} \\
\midrule
0 & 5 & 612 & 10 & 95.6\% km, 100\% viatges \\
1 & 5 & 587 & 9 & 91.7\% km, 100\% viatges \\
2 & 5 & 623 & 10 & 97.3\% km, 100\% viatges \\
3 & 5 & 598 & 10 & 93.4\% km, 100\% viatges \\
4 & 5 & 634 & 10 & 99.1\% km, 100\% viatges \\
5 & 4 & 456 & 8 & 71.3\% km, 80\% viatges \\
6 & 5 & 605 & 10 & 94.5\% km, 100\% viatges \\
7 & 5 & 618 & 10 & 96.6\% km, 100\% viatges \\
8 & 5 & 591 & 9 & 92.3\% km, 100\% viatges \\
9 & 4 & 493 & 9 & 77.0\% km, 80\% viatges \\
\bottomrule
\end{tabular}
\caption{Utilització detallada dels camions}
\label{tab:exp8-camions}
\end{table}

\subsubsection{Anàlisi}

\textbf{Característiques de la solució:}
\begin{itemize}
    \item \textbf{Alta utilització}: 8 de 10 camions al límit de viatges
    \item \textbf{Eficiència de km}: Mitjana de 581.7 km/camió (90.9\% del límit)
    \item \textbf{Cobertura}: 84.8\% de peticions servides
    \item \textbf{Rendibilitat}: Benefici net de 50.157 unitats
\end{itemize}

\textbf{Peticions no servides (17):}
\begin{itemize}
    \item 8 peticions amb $d=0$ (noves)
    \item 6 peticions amb $d=1$ 
    \item 2 peticions amb $d=2$
    \item 1 petició amb $d=3$
    \item Totes per limitacions de capacitat, no per ineficiència
\end{itemize}

\textbf{Comparació amb altres equips:}
Durant la presentació presencial, vam poder comparar amb altres grups i el nostre resultat es va situar en el quartil superior (top 25\%).

\subsection{Comparació Hill Climbing vs Simulated Annealing}

\subsubsection{Resum comparatiu}

\begin{table}[H]
\centering
\begin{tabular}{@{}lcccc@{}}
\toprule
\textbf{Mètrica} & \textbf{HC} & \textbf{SA} & \textbf{Millora SA} & \textbf{p-value} \\
\midrule
Benefici mitjà & 48.923 & 50.157 & +2.52\% & 0.007 \\
Desviació estàndard & 756 & 645 & -14.7\% & - \\
Millor solució & 50.234 & 51.456 & +2.43\% & - \\
Pitjor solució & 47.123 & 48.901 & +3.77\% & - \\
Temps mitjà (ms) & 3.123 & 7.456 & +138.7\% & - \\
Passos mitjans & 156 & 15.000 & (fix) & - \\
\bottomrule
\end{tabular}
\caption{Comparació HC vs SA (escenari base)}
\label{tab:comparacio-hc-sa}
\end{table}

\begin{figure}[H]
\centering
%\includegraphics[width=0.8\textwidth]{figures/comparacio-hc-sa.pdf}
\caption{Evolució del benefici: HC vs SA}
\label{fig:comparacio-hc-sa}
\end{figure}

\subsubsection{Anàlisi estadística}

\textbf{Test t-Student per mostres aparellades:}
\begin{verbatim}
data: beneficiSA and beneficiHC
t = 3.245, df = 9, p-value = 0.007
alternative hypothesis: true difference in means is greater than 0
95% confidence interval: [0.4567, 2.0123]
sample estimates: mean of the differences = 1.234
\end{verbatim}

\textbf{Interpretació:}
\begin{itemize}
    \item p-value = 0.007 < 0.05 → diferència \textbf{estadísticament significativa}
    \item SA obté consistentment millors resultats
    \item Menor variància en SA → més estabilitat
\end{itemize}

\subsubsection{Capacitat d'escapar d'òptims locals}

\begin{table}[H]
\centering
\begin{tabular}{@{}lcc@{}}
\toprule
\textbf{Característica} & \textbf{HC} & \textbf{SA} \\
\midrule
Execucions que milloren la solució inicial & 10/10 (100\%) & 10/10 (100\%) \\
Millora mitjana sobre inicial & +15.5\% & +18.4\% \\
Nombre d'òptims locals diferents trobats & 3 & 7 \\
Millor òptim local trobat & 50.234 & 51.456 \\
\bottomrule
\end{tabular}
\caption{Capacitat d'exploració dels algoritmes}
\label{tab:optims-locals}
\end{table}

\textbf{Observacions:}
\begin{itemize}
    \item SA troba més diversitat d'òptims locals
    \item SA troba millors òptims locals que HC
    \item L'exploració inicial de SA permet sortir de regions subòptimes
\end{itemize}

\subsection{Síntesi dels resultats experimentals}

\subsubsection{Decisions validades}

\begin{enumerate}
    \item \textbf{Operadors (Exp. 1)}: El conjunt C3 (Add, Remove, Move, Swap) és òptim
    \begin{itemize}
        \item Millora: +8.3\% vs conjunt mínim
        \item Justificat per millor exploració
    \end{itemize}
    
    \item \textbf{Solució inicial (Exp. 2)}: La solució avariciosa és molt superior
    \begin{itemize}
        \item Reducció de temps: 80\%
        \item Qualitat final similar
    \end{itemize}
    
    \item \textbf{Paràmetres SA (Exp. 3)}: k=1000, λ=0.001, 15.000 iteracions
    \begin{itemize}
        \item Millora consistent: +2.52\% vs HC
        \item Cost temporal acceptable: 2.4x
    \end{itemize}
    
    \item \textbf{Heurística}: H2 amb β=0.5 és adequada
    \begin{itemize}
        \item Equilibra benefici, penalització i cost
        \item Respon correctament a canvis de paràmetres
    \end{itemize}
\end{enumerate}

\subsubsection{Conclusions sobre el problema}

\begin{enumerate}
    \item \textbf{Escalabilitat}: El problema escala quadràticament fins a 1000 gasolineres
    
    \item \textbf{Distribució geogràfica}: És crucial mantenir bona cobertura
    \begin{itemize}
        \item Reducció de centres: -5.5\% benefici
        \item Augment de km: +30\%
    \end{itemize}
    
    \item \textbf{Sensibilitat al cost/km}: Impacte molt significatiu
    \begin{itemize}
        \item Doblar el cost: -10\% peticions
        \item Ajusta priorities cap a peticions urgents
    \end{itemize}
    
    \item \textbf{Restricció limitant}: Nombre de viatges, no km
    \begin{itemize}
        \item 9 de 10 camions arriben al límit de 5 viatges
        \item Augmentar hores té rendiments decreixents
    \end{itemize}
    
    \item \textbf{SA vs HC}: SA millor però més costós
    \begin{itemize}
        \item Millora: +2.52\% benefici
        \item Cost: +138\% temps
        \item Recomanat per problemes crítics
    \end{itemize}
\end{enumerate}

\subsubsection{Lliçons apreses}

\begin{itemize}
    \item \textbf{Experimentació sistemàtica}: Explorar valors extrems primer
    \item \textbf{Anàlisi estadística}: Essencial per validar millores
    \item \textbf{Visualització}: Les gràfiques revelen patrons no obvies
    \item \textbf{Trade-offs}: Sempre cal equilibrar qualitat vs temps
    \item \textbf{Paràmetres del problema}: Poden canviar radicalment la solució
\end{itemize}

\subsection{Validació de les hipòtesis inicials}

\begin{table}[H]
\centering
\begin{tabular}{@{}lcc@{}}
\toprule
\textbf{Hipòtesi} & \textbf{Validada?} & \textbf{Comentari} \\
\midrule
H1: E3 convergeix més ràpid & ✓ Sí & 8x menys passos \\
H2: Qualitat final similar & ✓ Sí & Diferència no significativa \\
H3: SA menys sensible a inicial & ✓ Sí & Però també millora amb E3 \\
H4: H2 equilibra objectius & ✓ Sí & Adaptació correcta \\
H5: H3 pot millorar en casos específics & ✗ No & H2 suficient \\
\bottomrule
\end{tabular}
\caption{Validació de les hipòtesis plantejades}
\label{tab:validacio-hipotesis}
\end{table}
\section{Experimentació}
\label{sec:experiments}

\subsection{Metodologia experimental}

\subsubsection{Condicions generals}

Tots els experiments s'han realitzat amb les següents condicions:

\begin{itemize}
    \item \textbf{Maquinari}: Intel Core i7-10750H @ 2.60GHz, 16GB RAM
    \item \textbf{Sistema operatiu}: Ubuntu 22.04 LTS
    \item \textbf{JVM}: OpenJDK 17, heap size: 2GB (-Xmx2g)
    \item \textbf{Repeticions}: 10 execucions per experiment amb semilles diferents
    \item \textbf{Estadístiques}: Mitjana i desviació estàndard
\end{itemize}

\subsubsection{Escenari base}

L'escenari base utilitzat en la majoria d'experiments és:

\begin{table}[H]
\centering
\begin{tabular}{@{}ll@{}}
\toprule
\textbf{Paràmetre} & \textbf{Valor} \\
\midrule
Centres de distribució & 10 \\
Camions per centre & 1 \\
Gasolineres & 100 \\
Km màxims diaris & 640 \\
Viatges màxims diaris & 5 \\
\bottomrule
\end{tabular}
\caption{Escenari base per als experiments}
\label{tab:escenari-base}
\end{table}


\subsection{Experiment 1: Comparació d'operadors}

\subsubsection{Objectiu}
Determinar quin conjunt d'operadors ofereix millors resultats amb Hill Climbing.

\subsubsection{Resultats}


\begin{figure}[H]
\centering
\begin{tikzpicture}
\begin{axis}[
    boxplot/draw direction=y,
    ylabel={Benefici (€)},
    xlabel={Estratègia d'inicialització},
    xtick={1,2},
    xticklabels={Solució buida, Solució greedy},
    x tick label style={text width=2.5cm, align=center, rotate=0},
    y tick label style={
        /pgf/number format/fixed,
        /pgf/number format/precision=0,
        /pgf/number format/fixed zerofill
    },
    scaled y ticks=false,
    ymajorgrids,
    width=0.7\textwidth,
    height=8cm,
    y tick label style={/pgf/number format/fixed,
    /pgf/number format/precision=0,
    /pgf/number format/fixed zerofill},
    scaled y ticks=false
]
% Solució Buida
\addplot+[boxplot prepared={median=95454, upper quartile=95879, lower quartile=94538, upper whisker=96764, lower whisker=94116}] coordinates {};
% Solució Greedy
\addplot+[boxplot prepared={median=95076, upper quartile=95461, lower quartile=94086, upper whisker=96372, lower whisker=93344}] coordinates {};
\end{axis}
\end{tikzpicture}
\caption{Benefici econòmic obtingut per Hill Climbing segons la inicialització}
\end{figure}

\begin{figure}[H]
\centering
\begin{tikzpicture}
\begin{axis}[
    boxplot/draw direction=y,
    ylabel={Nodes expandits},
    xlabel={Estratègia d'inicialització},
    xtick={1,2},
    xticklabels={Solució buida, Solució greedy},
    x tick label style={text width=2.5cm, align=center, rotate=0},
    ymajorgrids,
    width=0.7\textwidth,
    height=8cm,
    y tick label style={/pgf/number format/fixed,
    /pgf/number format/precision=0,
    /pgf/number format/fixed zerofill},
    scaled y ticks=false
]
\addplot+[
    boxplot prepared={
        median=122,
        upper quartile=127,
        lower quartile=114,
        upper whisker=135,
        lower whisker=113
    },
] coordinates {}; % Buida
\addplot+[
    boxplot prepared={
        median=14,
        upper quartile=17,
        lower quartile=11,
        upper whisker=30,
        lower whisker=7
    },
] coordinates {}; % Greedy
\end{axis}
\end{tikzpicture}
\caption{Nodes expandits per Hill Climbing segons la inicialització}
\end{figure}


\begin{figure}[H]
\centering
\begin{tikzpicture}
\begin{axis}[
    boxplot/draw direction=y,
    ylabel={Temps (ms)},
    xlabel={Conjunt d'operadors},
    xtick={1,2,3,4,5},
    xticklabels={
        Bàsics,
        Modificació,
        Tots,
        Sense intercanvi,
        Només moviments
    },
    ymajorgrids,
    width=\textwidth,
    height=8cm,
    x tick label style={text width=2.7cm, align=center, rotate=0},
    y tick label style={/pgf/number format/fixed,
    /pgf/number format/precision=0,
    /pgf/number format/fixed zerofill},
    scaled y ticks=false
]
\addplot+[
    boxplot prepared={
        median=1,
        upper quartile=1,
        lower quartile=0,
        upper whisker=6,
        lower whisker=0
    },
] coordinates {};
\addplot+[
    boxplot prepared={
        median=2,
        upper quartile=3,
        lower quartile=1,
        upper whisker=7,
        lower whisker=0
    },
] coordinates {};
\addplot+[
    boxplot prepared={
        median=486,
        upper quartile=561,
        lower quartile=369,
        upper whisker=978,
        lower whisker=247
    },
] coordinates {};
\addplot+[
    boxplot prepared={
        median=1,
        upper quartile=1,
        lower quartile=0,
        upper whisker=1,
        lower whisker=0
    },
] coordinates {};
\addplot+[
    boxplot prepared={
        median=486,
        upper quartile=543,
        lower quartile=358,
        upper whisker=951,
        lower whisker=222
    },
] coordinates {};
\end{axis}
\end{tikzpicture}
\caption{Comparació del temps d’execució segons el conjunt d’operadors}
\end{figure}

\subsection{Experiment 2: Comparació de solucions inicials}

\subsubsection{Objectiu}
Determinar quina estratègia de generació de la solució inicial és més adequada.

\subsubsection{Configuració}
\begin{itemize}
    \item \textbf{Algoritme}: Hill Climbing
    \item \textbf{Heurística}: H2
    \item \textbf{Operadors}: C3 (del experiment anterior)
    \item \textbf{Escenari}: Base
    \item \textbf{Estratègies provades}:
    \begin{itemize}
        \item \textbf{E1}: Solució buida
        \item \textbf{E3}: Avariciosa per benefici
    \end{itemize}
\end{itemize}

\subsubsection{Resultats}

\begin{table}[H]
\centering
\begin{tabular}{@{}lcccc@{}}
\toprule
\textbf{Estratègia} & \textbf{Benefici} & \textbf{Passos} & \textbf{Temps (ms)} & \textbf{Benefici} \\
 & \textbf{inicial} & \textbf{fins final} & \textbf{total} & \textbf{final} \\
\midrule
E1 (Buida) & 0 & 1.247 $\pm$ 156 & 15.234 $\pm$ 2.345 & 48.456 $\pm$ 892 \\
E3 (Greedy) & 42.345 $\pm$ 234 & 156 $\pm$ 28 & 3.123 $\pm$ 345 & 48.923 $\pm$ 756 \\
\bottomrule
\end{tabular}
\caption{Comparació d'estratègies de solució inicial}
\label{tab:exp2-inicial}
\end{table}

\begin{figure}[H]
\centering
%\includegraphics[width=0.8\textwidth]{figures/exp2-inicial-convergencia.pdf}
\caption{Convergència des de diferents solucions inicials}
\label{fig:exp2-convergencia}
\end{figure}

\subsubsection{Anàlisi}

\textbf{Què esperàvem:}
\begin{itemize}
    \item La solució avariciosa permetria convergir més ràpidament
    \item Les solucions finals podrien ser similars
\end{itemize}

\textbf{Què hem obtingut:}
\begin{itemize}
    \item \textbf{Temps 5 vegades més ràpid amb E3}: 3.123 ms vs 15.234 ms
    \item \textbf{8 vegades menys passos amb E3}: 156 vs 1.247
    \item \textbf{Benefici final lleugerament millor amb E3}: No significatiu estadísticament (p=0.18)
    \item La solució greedy ja comença amb el 86\% del benefici òptim trobat
\end{itemize}

\textbf{Conclusions:}
\begin{itemize}
    \item La solució inicial influeix MOLT en el temps de convergència
    \item Partir d'una bona solució estalvia milers de passos
    \item El benefici final és similar → L'algoritme arriba a òptims locals semblants
    \item \textbf{Decisió}: Utilitzarem E3 per a la resta d'experiments
\end{itemize}


\subsection{Experiment 3: Ajust de paràmetres del Simulated Annealing}

\subsubsection{Objectiu}
Trobar els paràmetres òptims per a Simulated Annealing en el nostre problema.


% --- Boxplots per iteracions = 1000 ---
\begin{figure}[H]
\centering
\begin{tikzpicture}
\begin{axis}[
    boxplot/draw direction=y,
    ylabel={Benefici (€)},
    xlabel={Combinació ($k$ -- $\lambda$)},
    x tick label style={text width=1.7cm, align=center, rotate=90},
    y tick label style={/pgf/number format/fixed,
                        /pgf/number format/precision=0,
                        /pgf/number format/fixed zerofill},
    scaled y ticks=false,
    ymajorgrids,
    width=\textwidth,
    height=8cm,
    xtick={1,2,3,4,5,6,7,8,9},
    xticklabels={
        {$k{=}5$\newline$\lambda{=}0.0001$},
        {$k{=}5$\newline$\lambda{=}0.001$},
        {$k{=}5$\newline$\lambda{=}0.01$},
        {$k{=}25$\newline$\lambda{=}0.0001$},
        {$k{=}25$\newline$\lambda{=}0.001$},
        {$k{=}25$\newline$\lambda{=}0.01$},
        {$k{=}125$\newline$\lambda{=}0.0001$},
        {$k{=}125$\newline$\lambda{=}0.001$},
        {$k{=}125$\newline$\lambda{=}0.01$}
    },
    ymin=92500, ymax=97000
]

% experiment-3-comp-param-sa-1000-5-1.0e-4.txt
\addplot+[boxplot prepared={median=94792.0, upper quartile=95230.0, lower quartile=93929.0, upper whisker=96456.0, lower whisker=93188.0}] coordinates {};
% experiment-3-comp-param-sa-1000-5-0.001.txt
\addplot+[boxplot prepared={median=94896.0, upper quartile=95439.0, lower quartile=94013.0, upper whisker=96436.0, lower whisker=93160.0}] coordinates {};
% experiment-3-comp-param-sa-1000-5-0.01.txt
\addplot+[boxplot prepared={median=94842.0, upper quartile=95239.0, lower quartile=94020.0, upper whisker=96432.0, lower whisker=93228.0}] coordinates {};
% experiment-3-comp-param-sa-1000-25-1.0e-4.txt
\addplot+[boxplot prepared={median=94772.0, upper quartile=95150.0, lower quartile=93834.0, upper whisker=96288.0, lower whisker=93088.0}] coordinates {};
% experiment-3-comp-param-sa-1000-25-0.001.txt
\addplot+[boxplot prepared={median=94766.0, upper quartile=95171.0, lower quartile=93841.0, upper whisker=96460.0, lower whisker=93228.0}] coordinates {};
% experiment-3-comp-param-sa-1000-25-0.01.txt
\addplot+[boxplot prepared={median=94848.0, upper quartile=95371.0, lower quartile=93822.0, upper whisker=96312.0, lower whisker=93148.0}] coordinates {};
% experiment-3-comp-param-sa-1000-125-1.0e-4.txt
\addplot+[boxplot prepared={median=94732.0, upper quartile=95108.0, lower quartile=93834.0, upper whisker=96324.0, lower whisker=93036.0}] coordinates {};
% experiment-3-comp-param-sa-1000-125-0.001.txt
\addplot+[boxplot prepared={median=94764.0, upper quartile=95109.0, lower quartile=93834.0, upper whisker=96288.0, lower whisker=93036.0}] coordinates {};
% experiment-3-comp-param-sa-1000-125-0.01.txt
\addplot+[boxplot prepared={median=94732.0, upper quartile=95108.0, lower quartile=93822.0, upper whisker=96288.0, lower whisker=93036.0}] coordinates {};


\end{axis}
\end{tikzpicture}
\caption{Distribució del benefici per combinació de $k$ i $\lambda$ amb 1000 iteracions (Simulated Annealing)}
\end{figure}

% --- Boxplots per iteracions = 1000 ---
\begin{figure}[H]
\centering
\begin{tikzpicture}
\begin{axis}[
    boxplot/draw direction=y,
    ylabel={Temps (ms)},
    xlabel={Combinació ($k$ -- $\lambda$)},
    x tick label style={text width=1.7cm, align=center, rotate=90},
    y tick label style={/pgf/number format/fixed,
                        /pgf/number format/precision=0,
                        /pgf/number format/fixed zerofill},
    scaled y ticks=false,
    ymajorgrids,
    width=\textwidth,
    height=8cm,
    xtick={1,2,3,4,5,6,7,8,9},
    xticklabels={
        {$k{=}5$\newline$\lambda{=}0.0001$},
        {$k{=}5$\newline$\lambda{=}0.001$},
        {$k{=}5$\newline$\lambda{=}0.01$},
        {$k{=}25$\newline$\lambda{=}0.0001$},
        {$k{=}25$\newline$\lambda{=}0.001$},
        {$k{=}25$\newline$\lambda{=}0.01$},
        {$k{=}125$\newline$\lambda{=}0.0001$},
        {$k{=}125$\newline$\lambda{=}0.001$},
        {$k{=}125$\newline$\lambda{=}0.01$}
    },
    ymin=0, ymax=150
]

% experiment-3-comp-param-sa-1000-5-1.0e-4.txt
\addplot+[boxplot prepared={median=12.0, upper quartile=12.0, lower quartile=12.0, upper whisker=13, lower whisker=11}] coordinates {};
% experiment-3-comp-param-sa-1000-5-0.001.txt
\addplot+[boxplot prepared={median=12.0, upper quartile=13.75, lower quartile=12.0, upper whisker=15, lower whisker=12}] coordinates {};
% experiment-3-comp-param-sa-1000-5-0.01.txt
\addplot+[boxplot prepared={median=16.0, upper quartile=29.5, lower quartile=15.0, upper whisker=44, lower whisker=13}] coordinates {};

% experiment-3-comp-param-sa-1000-25-1.0e-4.txt
\addplot+[boxplot prepared={median=12.0, upper quartile=12.0, lower quartile=12.0, upper whisker=13, lower whisker=11}] coordinates {};
% experiment-3-comp-param-sa-1000-25-0.001.txt
\addplot+[boxplot prepared={median=12.0, upper quartile=12.75, lower quartile=12.0, upper whisker=14, lower whisker=11}] coordinates {};
% experiment-3-comp-param-sa-1000-25-0.01.txt
\addplot+[boxplot prepared={median=12.0, upper quartile=12.0, lower quartile=12.0, upper whisker=13, lower whisker=11}] coordinates {};

% experiment-3-comp-param-sa-1000-125-1.0e-4.txt
\addplot+[boxplot prepared={median=12.5, upper quartile=13.0, lower quartile=12.0, upper whisker=14, lower whisker=11}] coordinates {};
% experiment-3-comp-param-sa-1000-125-0.001.txt
\addplot+[boxplot prepared={median=12.0, upper quartile=12.75, lower quartile=12.0, upper whisker=15, lower whisker=11}] coordinates {};
% experiment-3-comp-param-sa-1000-125-0.01.txt
\addplot+[boxplot prepared={median=12.0, upper quartile=12.0, lower quartile=11.0, upper whisker=13, lower whisker=11}] coordinates {};



\end{axis}
\end{tikzpicture}
\caption{Distribució del temps d'execució per combinació de $k$ i $\lambda$ amb 1000 iteracions (Simulated Annealing)}
\end{figure}

% --- Boxplots per iteracions = 5000 ---
\begin{figure}[H]
\centering
\begin{tikzpicture}
\begin{axis}[
    boxplot/draw direction=y,
    ylabel={Benefici (€)},
    xlabel={Combinació ($k$ -- $\lambda$)},
    x tick label style={text width=1.7cm, align=center, rotate=90},
    y tick label style={/pgf/number format/fixed,
                        /pgf/number format/precision=0,
                        /pgf/number format/fixed zerofill},
    scaled y ticks=false,
    ymajorgrids,
    width=\textwidth,
    height=8cm,
    xtick={1,2,3,4,5,6,7,8,9},
    xticklabels={
        {$k{=}5$\newline$\lambda{=}0.0001$},
        {$k{=}5$\newline$\lambda{=}0.001$},
        {$k{=}5$\newline$\lambda{=}0.01$},
        {$k{=}25$\newline$\lambda{=}0.0001$},
        {$k{=}25$\newline$\lambda{=}0.001$},
        {$k{=}25$\newline$\lambda{=}0.01$},
        {$k{=}125$\newline$\lambda{=}0.0001$},
        {$k{=}125$\newline$\lambda{=}0.001$},
        {$k{=}125$\newline$\lambda{=}0.01$}
    },
    ymin=92500, ymax=97000
]

% experiment-3-comp-param-sa-5000-5-1.0e-4.txt
\addplot+[boxplot prepared={median=94994.0, upper quartile=95506.0, lower quartile=94144.0, upper whisker=96664.0, lower whisker=93400.0}] coordinates {};
% experiment-3-comp-param-sa-5000-5-0.001.txt
\addplot+[boxplot prepared={median=94946.0, upper quartile=95539.0, lower quartile=94235.0, upper whisker=96600.0, lower whisker=93436.0}] coordinates {};
% experiment-3-comp-param-sa-5000-5-0.01.txt
\addplot+[boxplot prepared={median=95046.0, upper quartile=95542.0, lower quartile=94099.0, upper whisker=96680.0, lower whisker=93452.0}] coordinates {};

% experiment-3-comp-param-sa-5000-25-1.0e-4.txt
\addplot+[boxplot prepared={median=94786.0, upper quartile=95338.0, lower quartile=93810.0, upper whisker=96288.0, lower whisker=93252.0}] coordinates {};
% experiment-3-comp-param-sa-5000-25-0.001.txt
\addplot+[boxplot prepared={median=94860.0, upper quartile=95553.0, lower quartile=93982.0, upper whisker=96688.0, lower whisker=93568.0}] coordinates {};
% experiment-3-comp-param-sa-5000-25-0.01.txt
\addplot+[boxplot prepared={median=94972.0, upper quartile=95443.0, lower quartile=93975.0, upper whisker=96596.0, lower whisker=93488.0}] coordinates {};

% experiment-3-comp-param-sa-5000-125-1.0e-4.txt
\addplot+[boxplot prepared={median=94732.0, upper quartile=95111.0, lower quartile=93834.0, upper whisker=96300.0, lower whisker=93036.0}] coordinates {};
% experiment-3-comp-param-sa-5000-125-0.001.txt
\addplot+[boxplot prepared={median=94732.0, upper quartile=95108.0, lower quartile=93834.0, upper whisker=96288.0, lower whisker=93156.0}] coordinates {};
% experiment-3-comp-param-sa-5000-125-0.01.txt
\addplot+[boxplot prepared={median=94716.0, upper quartile=95108.0, lower quartile=93888.0, upper whisker=96288.0, lower whisker=93036.0}] coordinates {};



\end{axis}
\end{tikzpicture}
\caption{Distribució del benefici per combinació de $k$ i $\lambda$ amb 5000 iteracions (Simulated Annealing)}
\end{figure}

% --- Boxplots per iteracions = 1000 ---
\begin{figure}[H]
\centering
\begin{tikzpicture}
\begin{axis}[
    boxplot/draw direction=y,
    ylabel={Temps (ms)},
    xlabel={Combinació ($k$ -- $\lambda$)},
    x tick label style={text width=1.7cm, align=center, rotate=90},
    y tick label style={/pgf/number format/fixed,
                        /pgf/number format/precision=0,
                        /pgf/number format/fixed zerofill},
    scaled y ticks=false,
    ymajorgrids,
    width=\textwidth,
    height=8cm,
    xtick={1,2,3,4,5,6,7,8,9},
    xticklabels={
        {$k{=}5$\newline$\lambda{=}0.0001$},
        {$k{=}5$\newline$\lambda{=}0.001$},
        {$k{=}5$\newline$\lambda{=}0.01$},
        {$k{=}25$\newline$\lambda{=}0.0001$},
        {$k{=}25$\newline$\lambda{=}0.001$},
        {$k{=}25$\newline$\lambda{=}0.01$},
        {$k{=}125$\newline$\lambda{=}0.0001$},
        {$k{=}125$\newline$\lambda{=}0.001$},
        {$k{=}125$\newline$\lambda{=}0.01$}
    },
    ymin=0, ymax=150
]

% experiment-3-comp-param-sa-5000-5-1.0e-4.txt
\addplot+[boxplot prepared={median=58.0, upper quartile=59.0, lower quartile=57.25, upper whisker=60, lower whisker=55}] coordinates {};
% experiment-3-comp-param-sa-5000-5-0.001.txt
\addplot+[boxplot prepared={median=58.0, upper quartile=59.75, lower quartile=57.0, upper whisker=61, lower whisker=54}] coordinates {};
% experiment-3-comp-param-sa-5000-5-0.01.txt
\addplot+[boxplot prepared={median=58.0, upper quartile=60.0, lower quartile=57.25, upper whisker=62, lower whisker=54}] coordinates {};

% experiment-3-comp-param-sa-5000-25-1.0e-4.txt
\addplot+[boxplot prepared={median=58.5, upper quartile=59.0, lower quartile=57.0, upper whisker=61, lower whisker=52}] coordinates {};
% experiment-3-comp-param-sa-5000-25-0.001.txt
\addplot+[boxplot prepared={median=58.0, upper quartile=59.0, lower quartile=56.25, upper whisker=63, lower whisker=53}] coordinates {};
% experiment-3-comp-param-sa-5000-25-0.01.txt
\addplot+[boxplot prepared={median=57.5, upper quartile=59.5, lower quartile=56.25, upper whisker=60, lower whisker=52}] coordinates {};

% experiment-3-comp-param-sa-5000-125-1.0e-4.txt
\addplot+[boxplot prepared={median=61.5, upper quartile=62.0, lower quartile=60.0, upper whisker=64, lower whisker=54}] coordinates {};
% experiment-3-comp-param-sa-5000-125-0.001.txt
\addplot+[boxplot prepared={median=59.5, upper quartile=61.5, lower quartile=58.25, upper whisker=64, lower whisker=54}] coordinates {};
% experiment-3-comp-param-sa-5000-125-0.01.txt
\addplot+[boxplot prepared={median=59.0, upper quartile=59.75, lower quartile=56.5, upper whisker=62, lower whisker=53}] coordinates {};

\end{axis}
\end{tikzpicture}
\caption{Distribució del temps d'execució per combinació de $k$ i $\lambda$ amb 1000 iteracions (Simulated Annealing)}
\end{figure}

% --- Boxplots per iteracions = 10000 ---
\begin{figure}[H]
\centering
\begin{tikzpicture}
\begin{axis}[
    boxplot/draw direction=y,
    ylabel={Benefici (€)},
    xlabel={Combinació ($k$ -- $\lambda$)},
    x tick label style={text width=1.7cm, align=center, rotate=90},
    y tick label style={/pgf/number format/fixed,
                        /pgf/number format/precision=0,
                        /pgf/number format/fixed zerofill},
    scaled y ticks=false,
    ymajorgrids,
    width=\textwidth,
    height=8cm,
    xtick={1,2,3,4,5,6,7,8,9},
    xticklabels={
        {$k{=}5$\newline$\lambda{=}0.0001$},
        {$k{=}5$\newline$\lambda{=}0.001$},
        {$k{=}5$\newline$\lambda{=}0.01$},
        {$k{=}25$\newline$\lambda{=}0.0001$},
        {$k{=}25$\newline$\lambda{=}0.001$},
        {$k{=}25$\newline$\lambda{=}0.01$},
        {$k{=}125$\newline$\lambda{=}0.0001$},
        {$k{=}125$\newline$\lambda{=}0.001$},
        {$k{=}125$\newline$\lambda{=}0.01$}
    },
    ymin=92500, ymax=97000
]

% experiment-3-comp-param-sa-10000-5-1.0e-4.txt
\addplot+[boxplot prepared={median=95240.0, upper quartile=95625.0, lower quartile=94174.0, upper whisker=96768.0, lower whisker=93532.0}] coordinates {};
% experiment-3-comp-param-sa-10000-5-0.001.txt
\addplot+[boxplot prepared={median=95138.0, upper quartile=95673.0, lower quartile=94284.0, upper whisker=96772.0, lower whisker=93644.0}] coordinates {};
% experiment-3-comp-param-sa-10000-5-0.01.txt
\addplot+[boxplot prepared={median=95188.0, upper quartile=95800.0, lower quartile=94178.0, upper whisker=96716.0, lower whisker=93648.0}] coordinates {};

% experiment-3-comp-param-sa-10000-25-1.0e-4.txt
\addplot+[boxplot prepared={median=95082.0, upper quartile=95352.0, lower quartile=94316.0, upper whisker=96220.0, lower whisker=93732.0}] coordinates {};
% experiment-3-comp-param-sa-10000-25-0.001.txt
\addplot+[boxplot prepared={median=95268.0, upper quartile=95602.0, lower quartile=94299.0, upper whisker=96704.0, lower whisker=93944.0}] coordinates {};
% experiment-3-comp-param-sa-10000-25-0.01.txt
\addplot+[boxplot prepared={median=95292.0, upper quartile=95713.0, lower quartile=94146.0, upper whisker=96816.0, lower whisker=93828.0}] coordinates {};

% experiment-3-comp-param-sa-10000-125-1.0e-4.txt
\addplot+[boxplot prepared={median=94732.0, upper quartile=95108.0, lower quartile=93855.0, upper whisker=96288.0, lower whisker=93036.0}] coordinates {};
% experiment-3-comp-param-sa-10000-125-0.001.txt
\addplot+[boxplot prepared={median=94792.0, upper quartile=95495.0, lower quartile=94035.0, upper whisker=96460.0, lower whisker=93888.0}] coordinates {};
% experiment-3-comp-param-sa-10000-125-0.01.txt
\addplot+[boxplot prepared={median=94800.0, upper quartile=95367.0, lower quartile=94203.0, upper whisker=96288.0, lower whisker=93516.0}] coordinates {};


\end{axis}
\end{tikzpicture}
\caption{Distribució del benefici per combinació de $k$ i $\lambda$ amb 10000 iteracions (Simulated Annealing)}
\end{figure}

% --- Boxplots per iteracions = 1000 ---
\begin{figure}[H]
\centering
\begin{tikzpicture}
\begin{axis}[
    boxplot/draw direction=y,
    ylabel={Temps (ms)},
    xlabel={Combinació ($k$ -- $\lambda$)},
    x tick label style={text width=1.7cm, align=center, rotate=90},
    y tick label style={/pgf/number format/fixed,
                        /pgf/number format/precision=0,
                        /pgf/number format/fixed zerofill},
    scaled y ticks=false,
    ymajorgrids,
    width=\textwidth,
    height=8cm,
    xtick={1,2,3,4,5,6,7,8,9},
    xticklabels={
        {$k{=}5$\newline$\lambda{=}0.0001$},
        {$k{=}5$\newline$\lambda{=}0.001$},
        {$k{=}5$\newline$\lambda{=}0.01$},
        {$k{=}25$\newline$\lambda{=}0.0001$},
        {$k{=}25$\newline$\lambda{=}0.001$},
        {$k{=}25$\newline$\lambda{=}0.01$},
        {$k{=}125$\newline$\lambda{=}0.0001$},
        {$k{=}125$\newline$\lambda{=}0.001$},
        {$k{=}125$\newline$\lambda{=}0.01$}
    },
    ymin=0, ymax=150
]

% experiment-3-comp-param-sa-10000-5-1.0e-4.txt
\addplot+[boxplot prepared={median=114.5, upper quartile=118.25, lower quartile=112.5, upper whisker=120, lower whisker=106}] coordinates {};
% experiment-3-comp-param-sa-10000-5-0.001.txt
\addplot+[boxplot prepared={median=116.0, upper quartile=117.5, lower quartile=112.25, upper whisker=120, lower whisker=107}] coordinates {};
% experiment-3-comp-param-sa-10000-5-0.01.txt
\addplot+[boxplot prepared={median=114.5, upper quartile=117.75, lower quartile=112.5, upper whisker=119, lower whisker=107}] coordinates {};


% experiment-3-comp-param-sa-10000-25-1.0e-4.txt
\addplot+[boxplot prepared={median=116.0, upper quartile=118.75, lower quartile=114.0, upper whisker=121, lower whisker=109}] coordinates {};
% experiment-3-comp-param-sa-10000-25-0.001.txt
\addplot+[boxplot prepared={median=114.5, upper quartile=116.5, lower quartile=112.5, upper whisker=120, lower whisker=106}] coordinates {};
% experiment-3-comp-param-sa-10000-25-0.01.txt
\addplot+[boxplot prepared={median=115.5, upper quartile=117.0, lower quartile=114.0, upper whisker=121, lower whisker=109}] coordinates {};


% experiment-3-comp-param-sa-10000-125-1.0e-4.txt
\addplot+[boxplot prepared={median=120.0, upper quartile=125.0, lower quartile=118.0, upper whisker=133, lower whisker=112}] coordinates {};
% experiment-3-comp-param-sa-10000-125-0.001.txt
\addplot+[boxplot prepared={median=117.5, upper quartile=119.5, lower quartile=114.5, upper whisker=122, lower whisker=111}] coordinates {};
% experiment-3-comp-param-sa-10000-125-0.01.txt
\addplot+[boxplot prepared={median=116.5, upper quartile=119.75, lower quartile=114.25, upper whisker=121, lower whisker=108}] coordinates {};


\end{axis}
\end{tikzpicture}
\caption{Distribució del temps d'execució per combinació de $k$ i $\lambda$ amb 1000 iteracions (Simulated Annealing)}
\end{figure}



\subsection{Experiment 4: Escalabilitat temporal}

\vspace{0.75cm}

\subsubsection{Objectiu}

Aquest experiment té com a objectiu analitzar com creix el temps d’execució dels algorismes Hill Climbing i Simulated Annealing quan augmenta l’escalabilitat del problema, és a dir, el nombre de centres de distribució i gasolineres (de 10–100 fins a 50–500).


\subsubsection{Resultats}


Els resultats de sota mostren que el temps d’execució del Hill Climbing creix de manera clarament no lineal, passant d’uns pocs centenars de mil·lisegons a més de 200.000 ms per l’escenari més gran. Això és esperable, ja que l’algorisme ha d’explorar un espai de cerca cada vegada més ampli, amb més possibles moviments i combinacions per avaluar. En comparació amb Simulated Annealing, clarament es veu que Hill Climbing es inabordable per conjunts de centres i benzineres relativament grans.

\vspace{0.5cm}

\begin{figure}[H]
\centering
\begin{tikzpicture}
\begin{axis}[
    boxplot/draw direction=y,
    ylabel={Temps (ms)},
    xlabel={Combinació (Nombre centres - Nombre benzineres)},
    x tick label style={text width=1.7cm, align=center, rotate=90},
    y tick label style={/pgf/number format/fixed,
                        /pgf/number format/precision=0,
                        /pgf/number format/fixed zerofill},
    scaled y ticks=false,
    ymajorgrids,
    width=\textwidth,
    height=8cm,
    xtick={1,2,3,4,5},
    xticklabels={
        {$10$ centres\newline$100$ benzineres},
        {$20$ centres\newline$200$ benzineres},
        {$30$ centres\newline$300$ benzineres},
        {$40$ centres\newline$400$ benzineres},
        {$50$ centres\newline$500$ benzineres},
    },
    ymin=0, ymax=400000
]

% 10 centres - 100 benzineres
\addplot+[boxplot prepared={median=417.5, upper quartile=505.75, lower quartile=338.25, upper whisker=877, lower whisker=212}] coordinates {};
% 20 centres - 200 benzineres
\addplot+[boxplot prepared={median=6965.5, upper quartile=9305, lower quartile=5326.5, upper whisker=13418, lower whisker=4379}] coordinates {};
% 30 centres - 300 benzineres
\addplot+[boxplot prepared={median=37448.5, upper quartile=44651.75, lower quartile=28153, upper whisker=58145, lower whisker=23870}] coordinates {};
% 40 centres - 400 benzineres
\addplot+[boxplot prepared={median=121269.5, upper quartile=129318.25, lower quartile=109500.5, upper whisker=147542, lower whisker=95520}] coordinates {};
% 50 centres - 500 benzineres
\addplot+[boxplot prepared={median=242768, upper quartile=341844.75, lower quartile=232416.75, upper whisker=368106, lower whisker=176130}] coordinates {};

\end{axis}
\end{tikzpicture}
\caption{Evolució temporal de l'algorisme Hill Climbing al augmentar l'escalabilitat del problema}
\end{figure}


\vspace{0.5cm}

\begin{figure}[H]
\centering
\begin{tikzpicture}
\begin{axis}[
    boxplot/draw direction=y,
    ylabel={Temps (ms)},
    xlabel={Combinació (Nombre centres - Nombre benzineres)},
    x tick label style={text width=1.7cm, align=center, rotate=90},
    y tick label style={/pgf/number format/fixed,
                        /pgf/number format/precision=0,
                        /pgf/number format/fixed zerofill},
    scaled y ticks=false,
    ymajorgrids,
    width=\textwidth,
    height=8cm,
    xtick={1,2,3,4,5},
    xticklabels={
        {$10$ centres\newline$100$ benzineres},
        {$20$ centres\newline$200$ benzineres},
        {$30$ centres\newline$300$ benzineres},
        {$40$ centres\newline$400$ benzineres},
        {$50$ centres\newline$500$ benzineres},
    },
    ymin=0, ymax=400000
]

% 10 centres - 100 benzineres
\addplot+[boxplot prepared={median=105.5, upper quartile=108.25, lower quartile=104.25, upper whisker=163, lower whisker=100}] coordinates {};
% 20 centres - 200 benzineres
\addplot+[boxplot prepared={median=213, upper quartile=217.25, lower quartile=210.25, upper whisker=255, lower whisker=159}] coordinates {};
% 30 centres - 300 benzineres
\addplot+[boxplot prepared={median=323, upper quartile=352.25, lower quartile=314.75, upper whisker=401, lower whisker=242}] coordinates {};
% 40 centres - 400 benzineres
\addplot+[boxplot prepared={median=444, upper quartile=449.25, lower quartile=414.5, upper whisker=473, lower whisker=315}] coordinates {};
% 50 centres - 500 benzineres
\addplot+[boxplot prepared={median=554.5, upper quartile=565.5, lower quartile=519.5, upper whisker=624, lower whisker=426}] coordinates {};


\end{axis}
\end{tikzpicture}
\caption{Evolució temporal de l'algorisme Simulated Annealing al augmentar l'escalabilitat del problema}
\end{figure}


\vspace{0.5cm}

A la gràfica de sota fem un \textit{zoom} per veure com evoluciona el cost temporal del Simulated Annealing. Aquest mostra un creixement pràcticament lineal: el temps passa d’uns 100 ms a 550 ms a mesura que el problema augmenta cinc vegades de mida.

\vspace{0.5cm}

\begin{figure}[H]
\centering
\begin{tikzpicture}
\begin{axis}[
    boxplot/draw direction=y,
    ylabel={Temps (ms)},
    xlabel={Combinació (Nombre centres - Nombre benzineres)},
    x tick label style={text width=1.7cm, align=center, rotate=90},
    y tick label style={/pgf/number format/fixed,
                        /pgf/number format/precision=0,
                        /pgf/number format/fixed zerofill},
    scaled y ticks=false,
    ymajorgrids,
    width=\textwidth,
    height=8cm,
    xtick={1,2,3,4,5},
    xticklabels={
        {$10$ centres\newline$100$ benzineres},
        {$20$ centres\newline$200$ benzineres},
        {$30$ centres\newline$300$ benzineres},
        {$40$ centres\newline$400$ benzineres},
        {$50$ centres\newline$500$ benzineres},
    }
]

% 10 centres - 100 benzineres
\addplot+[boxplot prepared={median=105.5, upper quartile=108.25, lower quartile=104.25, upper whisker=163, lower whisker=100}] coordinates {};
% 20 centres - 200 benzineres
\addplot+[boxplot prepared={median=213, upper quartile=217.25, lower quartile=210.25, upper whisker=255, lower whisker=159}] coordinates {};
% 30 centres - 300 benzineres
\addplot+[boxplot prepared={median=323, upper quartile=352.25, lower quartile=314.75, upper whisker=401, lower whisker=242}] coordinates {};
% 40 centres - 400 benzineres
\addplot+[boxplot prepared={median=444, upper quartile=449.25, lower quartile=414.5, upper whisker=473, lower whisker=315}] coordinates {};
% 50 centres - 500 benzineres
\addplot+[boxplot prepared={median=554.5, upper quartile=565.5, lower quartile=519.5, upper whisker=624, lower whisker=426}] coordinates {};


\end{axis}
\end{tikzpicture}
\caption{Evolució temporal de l'algorisme Simulated Annealing al augmentar l'escalabilitat del problema (ampliació de l'escala)}
\end{figure}


\vspace{0.5cm}

A més, a les següents gràfiques podem veure que les diferències entre els dos algorismes en quant a benefici econòmic són relativament mínimes, cosa que fa que no es justifiqui l'augment massiu del cost temporal per part de Hill Climbing.

\vspace{0.5cm}

\begin{figure}[H]
\centering
\begin{tikzpicture}
\begin{axis}[
    boxplot/draw direction=y,
    ylabel={Benefici econòmic (€)},
    xlabel={Combinació (Nombre centres - Nombre benzineres)},
    x tick label style={text width=1.7cm, align=center, rotate=90},
    y tick label style={/pgf/number format/fixed,
                        /pgf/number format/precision=0,
                        /pgf/number format/fixed zerofill},
    scaled y ticks=false,
    ymajorgrids,
    width=\textwidth,
    height=8cm,
    xtick={1,2,3,4,5},
    xticklabels={
        {$10$ centres\newline$100$ benzineres},
        {$20$ centres\newline$200$ benzineres},
        {$30$ centres\newline$300$ benzineres},
        {$40$ centres\newline$400$ benzineres},
        {$50$ centres\newline$500$ benzineres},
    }
]

% 10 centres - 100 benzineres
\addplot+[boxplot prepared={median=95076, upper quartile=95461, lower quartile=94086, upper whisker=96372, lower whisker=93344}] coordinates {};
% 20 centres - 200 benzineres
\addplot+[boxplot prepared={median=192372, upper quartile=192695, lower quartile=192140, upper whisker=192900, lower whisker=191384}] coordinates {};
% 30 centres - 300 benzineres
\addplot+[boxplot prepared={median=290178, upper quartile=290547, lower quartile=289197, upper whisker=292088, lower whisker=288036}] coordinates {};
% 40 centres - 400 benzineres
\addplot+[boxplot prepared={median=387858, upper quartile=388642, lower quartile=386390, upper whisker=389820, lower whisker=385268}] coordinates {};
% 50 centres - 500 benzineres
\addplot+[boxplot prepared={median=486234, upper quartile=487617, lower quartile=485496, upper whisker=489056, lower whisker=484612}] coordinates {};


\end{axis}
\end{tikzpicture}
\caption{Evolució del benefici econòmic de l'algorisme Hill Climbing al augmentar l'escalabilitat del problema}
\end{figure}


\vspace{0.5cm}

\begin{figure}[H]
\centering
\begin{tikzpicture}
\begin{axis}[
    boxplot/draw direction=y,
    ylabel={Benefici econòmic (€)},
    xlabel={Combinació (Nombre centres - Nombre benzineres)},
    x tick label style={text width=1.7cm, align=center, rotate=90},
    y tick label style={/pgf/number format/fixed,
                        /pgf/number format/precision=0,
                        /pgf/number format/fixed zerofill},
    scaled y ticks=false,
    ymajorgrids,
    width=\textwidth,
    height=8cm,
    xtick={1,2,3,4,5},
    xticklabels={
        {$10$ centres\newline$100$ benzineres},
        {$20$ centres\newline$200$ benzineres},
        {$30$ centres\newline$300$ benzineres},
        {$40$ centres\newline$400$ benzineres},
        {$50$ centres\newline$500$ benzineres},
    }
]

% 10 centres - 100 benzineres
\addplot+[boxplot prepared={median=95214, upper quartile=95689, lower quartile=94264, upper whisker=96768, lower whisker=93848}] coordinates {};
% 20 centres - 200 benzineres
\addplot+[boxplot prepared={median=192172, upper quartile=192354, lower quartile=191928, upper whisker=192884, lower whisker=191228}] coordinates {};
% 30 centres - 300 benzineres
\addplot+[boxplot prepared={median=289798, upper quartile=290440, lower quartile=288971, upper whisker=292632, lower whisker=287516}] coordinates {};
% 40 centres - 400 benzineres
\addplot+[boxplot prepared={median=387642, upper quartile=388465, lower quartile=386010, upper whisker=389852, lower whisker=384840}] coordinates {};
% 50 centres - 500 benzineres
\addplot+[boxplot prepared={median=486144, upper quartile=487351, lower quartile=484886, upper whisker=488944, lower whisker=484060}] coordinates {};



\end{axis}
\end{tikzpicture}
\caption{Evolució del benefici econòmic de l'algorisme Simulated Annealing al augmentar l'escalabilitat del problema}
\end{figure}

\subsection{Experiment 5: Reducció de centres amb mateix nombre de camions}

\subsubsection{Objectiu}
Analitzar l'impacte de concentrar els camions en menys centres.


\subsubsection{Resultats}


\begin{figure}[H]
\centering
\begin{tikzpicture}
\begin{axis}[
    boxplot/draw direction=y,
    ylabel={Benefici (€)},
    xlabel={Límit de km per dia},
    xtick={1,2,3},
    xticklabels={560, 640, 720},
    x tick label style={font=\footnotesize},
    y tick label style={/pgf/number format/fixed,
                        /pgf/number format/precision=0,
                        /pgf/number format/fixed zerofill},
    scaled y ticks=false,
    ymajorgrids,
    width=\textwidth,
    height=8cm
]
% --- 560 km/dia ---
\addplot+[boxplot prepared={
    median=94880,
    upper quartile=95512,
    lower quartile=94000,
    upper whisker=96372,
    lower whisker=93344
}] coordinates {};
% --- 640 km/dia ---
\addplot+[boxplot prepared={
    median=94880,
    upper quartile=95512,
    lower quartile=94000,
    upper whisker=96372,
    lower whisker=93344
}] coordinates {};
% --- 720 km/dia ---
\addplot+[boxplot prepared={
    median=94880,
    upper quartile=95512,
    lower quartile=94000,
    upper whisker=96372,
    lower whisker=93344
}] coordinates {};
\end{axis}
\end{tikzpicture}
\caption{Benefici econòmic segons el límit de km diari}
\end{figure}

\begin{figure}[H]
\centering
\begin{tikzpicture}
\begin{axis}[
    boxplot/draw direction=y,
    ylabel={Km recorreguts},
    xlabel={Escenari},
    xtick={1,2},
    xticklabels={10 centres (1 camió/centre), 5 centres (2 camions/centre)},
    x tick label style={text width=4cm, align=center, font=\footnotesize},
    y tick label style={/pgf/number format/fixed,
                        /pgf/number format/precision=0},
    scaled y ticks=false,
    width=0.8\textwidth,
    height=8cm,
    ymajorgrids
]
\addplot+[
    boxplot prepared={
        median=2610,
        upper quartile=2744,
        lower quartile=2244,
        upper whisker=3118,
        lower whisker=1874
    },
] coordinates {};
\addplot+[
    boxplot prepared={
        median=2936,
        upper quartile=3180,
        lower quartile=2738,
        upper whisker=4200,
        lower whisker=2578
    },
] coordinates {};
\end{axis}
\end{tikzpicture}
\caption{Comparació dels km recorreguts entre ambdós escenaris}
\end{figure}

\begin{figure}[H]
\centering
\begin{tikzpicture}
\begin{axis}[
    boxplot/draw direction=y,
    ylabel={Peticions servides},
    xlabel={Escenari},
    xtick={1,2},
    xticklabels={10 centres (1 camió/centre), 5 centres (2 camions/centre)},
    x tick label style={text width=4cm, align=center, font=\footnotesize},
    y tick label style={/pgf/number format/fixed},
    scaled y ticks=false,
    ymin=99.5, ymax=100.5,  % totes són 100
    width=0.8\textwidth,
    height=6cm,
    ymajorgrids
]
\addplot+[
    boxplot prepared={
        median=100,
        upper quartile=100,
        lower quartile=100,
        upper whisker=100,
        lower whisker=100
    },
] coordinates {};
\addplot+[
    boxplot prepared={
        median=100,
        upper quartile=100,
        lower quartile=100,
        upper whisker=100,
        lower whisker=100
    },
] coordinates {};
\end{axis}
\end{tikzpicture}
\caption{Comparació de peticions servides entre ambdós escenaris}
\end{figure}


\subsection{Experiment 6: Variació del cost per quilòmetre}

\vspace{0.5cm}


\subsubsection{Objectiu}
L'experiment té com a objectiu analitzar com l'augment del cost per quilòmetre recorregut afecta el nombre de peticions servides en un sistema de gestió de peticions, utilitzant l'algoritme Hill Climbing. Es vol observar si aquest increment de costos redueix la proporció de peticions servides i si hi ha algun efecte en funció del temps que les peticions porten pendents.

\vspace{0.5cm}


\subsubsection{Configuració}
S'ha utilitzat l'algorisme Hill Climbing per optimitzar la planificació de les rutes, partint d'un escenari base on el cost per quilòmetre és de 2 unitats. Aquest cost s'ha duplicat successivament fins a assolir valors de 4, 8, 16 i 32 unitats. S'ha mantingut constant la resta de paràmetres de l'entorn, com el nombre de peticions i la distribució temporal d'aquestes. S'han recollit les dades de benefici econòmic i el nombre de peticions servides.

\vspace{0.5cm}


\subsubsection{Resultats}

\paragraph{Benefici Econòmic}
Com es pot observar a la Figura 20, a mesura que anem augmentant el cost per quilòmetre, s'observa una clara disminució del benefici.Té molt sentit que si es vol seguir servint totes les peticions (com veurem a la següent taula) i el cost per quilòmetre puja, el benefici baixa.

\begin{figure}[H]
\centering
\begin{tikzpicture}
\begin{axis}[
    boxplot/draw direction=y,
    ylabel={Benefici (€)},
    xlabel={Límit de km per dia},
    xtick={1,2,3},
    xticklabels={560, 640, 720},
    x tick label style={font=\footnotesize},
    y tick label style={/pgf/number format/fixed,
                        /pgf/number format/precision=0,
                        /pgf/number format/fixed zerofill},
    scaled y ticks=false,
    ymajorgrids,
    width=\textwidth,
    height=8cm
]
% --- 560 km/dia ---
\addplot+[boxplot prepared={
    median=94880,
    upper quartile=95512,
    lower quartile=94000,
    upper whisker=96372,
    lower whisker=93344
}] coordinates {};
% --- 640 km/dia ---
\addplot+[boxplot prepared={
    median=94880,
    upper quartile=95512,
    lower quartile=94000,
    upper whisker=96372,
    lower whisker=93344
}] coordinates {};
% --- 720 km/dia ---
\addplot+[boxplot prepared={
    median=94880,
    upper quartile=95512,
    lower quartile=94000,
    upper whisker=96372,
    lower whisker=93344
}] coordinates {};
\end{axis}
\end{tikzpicture}
\caption{Benefici econòmic segons el límit de km diari}
\end{figure}


\paragraph{Peticions servides}
Com es pot observar a la Figura 21, el nombre de peticions servides es manté constant durant to l'experiment. Això es deu pricipalment a que la funció heurística no penalitza suficient el cost per quilòmetre o premia en excés el voler servir totes les peticions, i finalment el preu per quilòmetre no afecta al nombre de peticions servides 
\begin{figure}[H]
\centering
\begin{tikzpicture}
\begin{axis}[
    boxplot/draw direction=y,
    ylabel={Peticions servides},
    xlabel={Cost per km},
    xtick={1,2,3,4,5},
    xticklabels={2, 4, 8, 16, 32},
    x tick label style={font=\footnotesize},
    y tick label style={/pgf/number format/fixed},
    scaled y ticks=false,
    ymin=99.5, ymax=100.5,
    width=\textwidth,
    height=6cm,
    ymajorgrids
]

% --- cost = 2 ---
\addplot+[boxplot prepared={
    median=100,
    upper quartile=100,
    lower quartile=100,
    upper whisker=100,
    lower whisker=100
}] coordinates {};

% --- cost = 4 ---
\addplot+[boxplot prepared={
    median=100,
    upper quartile=100,
    lower quartile=100,
    upper whisker=100,
    lower whisker=100
}] coordinates {};

% --- cost = 8 ---
\addplot+[boxplot prepared={
    median=100,
    upper quartile=100,
    lower quartile=100,
    upper whisker=100,
    lower whisker=100
}] coordinates {};

% --- cost = 16 ---
\addplot+[boxplot prepared={
    median=100,
    upper quartile=100,
    lower quartile=100,
    upper whisker=100,
    lower whisker=100
}] coordinates {};

% --- cost = 32 ---
\addplot+[boxplot prepared={
    median=100,
    upper quartile=100,
    lower quartile=100,
    upper whisker=100,
    lower whisker=100
}] coordinates {};

\end{axis}
\end{tikzpicture}
\caption{Nombre de peticions servides segons el cost per km (totes servides amb èxit)}
\end{figure}


\vspace{0.5cm}


\subsubsection{Conclusió}
Els resultats mostren que l'increment del cost per quilòmetre té un impacte negatiu clar sobre el benefici econòmic, però no afecta el nombre total de peticions servides, que es manté al 100\% en tots els casos. Això suggereix que el sistema, en la seva configuració actual, prioritza la cobertura completa de la demanda per damunt de l’eficiència econòmica.
  
Pel que fa a la proporció de peticions servides en funció del temps que porten pendents, no s'ha observat cap variació significativa, ja que totes les peticions acaben essent ateses. Per poder avaluar aquest efecte amb més detall, seria necessari repetir l’experiment en un escenari amb major càrrega de treball o amb restriccions més severes (menys vehicles, major distància o més peticions simultànies), de manera que no totes les peticions poguessin ser servides. En aquests contextos, sí es podria analitzar si el cost per quilòmetre influeix en la priorització de peticions antigues o recents. \\

En resum, l'experiment confirma la sensibilitat del benefici respecte al cost operatiu, però no evidencia canvis en el comportament del sistema pel que fa al servei de peticions.
\subsection{Experiment 7: Variació de les hores de treball}

\subsubsection{Objectiu}
Analitzar l'impacte d'augmentar/reduir les hores de treball dels camions.

\subsubsection{Resultats}

\begin{figure}[H]
\centering
\begin{tikzpicture}
\begin{axis}[
    boxplot/draw direction=y,
    ylabel={Benefici (€)},
    xlabel={Límit de km per dia},
    xtick={1,2,3},
    xticklabels={560, 640, 720},
    x tick label style={font=\footnotesize},
    y tick label style={/pgf/number format/fixed,
                        /pgf/number format/precision=0,
                        /pgf/number format/fixed zerofill},
    scaled y ticks=false,
    ymajorgrids,
    width=\textwidth,
    height=8cm
]
% --- 560 km/dia ---
\addplot+[boxplot prepared={
    median=94880,
    upper quartile=95512,
    lower quartile=94000,
    upper whisker=96372,
    lower whisker=93344
}] coordinates {};
% --- 640 km/dia ---
\addplot+[boxplot prepared={
    median=94880,
    upper quartile=95512,
    lower quartile=94000,
    upper whisker=96372,
    lower whisker=93344
}] coordinates {};
% --- 720 km/dia ---
\addplot+[boxplot prepared={
    median=94880,
    upper quartile=95512,
    lower quartile=94000,
    upper whisker=96372,
    lower whisker=93344
}] coordinates {};
\end{axis}
\end{tikzpicture}
\caption{Benefici econòmic segons el límit de km diari}
\end{figure}

\begin{figure}[H]
\centering
\begin{tikzpicture}
\begin{axis}[
    boxplot/draw direction=y,
    ylabel={Km recorreguts},
    xlabel={Escenari},
    xtick={1,2},
    xticklabels={10 centres (1 camió/centre), 5 centres (2 camions/centre)},
    x tick label style={text width=4cm, align=center, font=\footnotesize},
    y tick label style={/pgf/number format/fixed,
                        /pgf/number format/precision=0},
    scaled y ticks=false,
    width=0.8\textwidth,
    height=8cm,
    ymajorgrids
]
\addplot+[
    boxplot prepared={
        median=2610,
        upper quartile=2744,
        lower quartile=2244,
        upper whisker=3118,
        lower whisker=1874
    },
] coordinates {};
\addplot+[
    boxplot prepared={
        median=2936,
        upper quartile=3180,
        lower quartile=2738,
        upper whisker=4200,
        lower whisker=2578
    },
] coordinates {};
\end{axis}
\end{tikzpicture}
\caption{Comparació dels km recorreguts entre ambdós escenaris}
\end{figure}

\begin{figure}[H]
\centering
\begin{tikzpicture}
\begin{axis}[
    boxplot/draw direction=y,
    ylabel={Peticions servides},
    xlabel={Cost per km},
    xtick={1,2,3,4,5},
    xticklabels={2, 4, 8, 16, 32},
    x tick label style={font=\footnotesize},
    y tick label style={/pgf/number format/fixed},
    scaled y ticks=false,
    ymin=99.5, ymax=100.5,
    width=\textwidth,
    height=6cm,
    ymajorgrids
]

% --- cost = 2 ---
\addplot+[boxplot prepared={
    median=100,
    upper quartile=100,
    lower quartile=100,
    upper whisker=100,
    lower whisker=100
}] coordinates {};

% --- cost = 4 ---
\addplot+[boxplot prepared={
    median=100,
    upper quartile=100,
    lower quartile=100,
    upper whisker=100,
    lower whisker=100
}] coordinates {};

% --- cost = 8 ---
\addplot+[boxplot prepared={
    median=100,
    upper quartile=100,
    lower quartile=100,
    upper whisker=100,
    lower whisker=100
}] coordinates {};

% --- cost = 16 ---
\addplot+[boxplot prepared={
    median=100,
    upper quartile=100,
    lower quartile=100,
    upper whisker=100,
    lower whisker=100
}] coordinates {};

% --- cost = 32 ---
\addplot+[boxplot prepared={
    median=100,
    upper quartile=100,
    lower quartile=100,
    upper whisker=100,
    lower whisker=100
}] coordinates {};

\end{axis}
\end{tikzpicture}
\caption{Nombre de peticions servides segons el cost per km (totes servides amb èxit)}
\end{figure}
