\subsubsection{Configuració}
\begin{itemize}
    \item \textbf{Centres}: 10
    \item \textbf{Camions per centre}: 1
    \item \textbf{Gasolineres}: 100
    \item \textbf{Semilla}: 1234 (centres i gasolineres)
    \item \textbf{Algoritme}: Simulated Annealing amb paràmetres optimitzats
    \item \textbf{Operadors}: C3 (complet)
    \item \textbf{Solució inicial}: E3 (Greedy)
    \item \textbf{Heurística}: H2
\end{itemize}

\subsubsection{Resultats}

\begin{table}[H]
\centering
\begin{tabular}{@{}lc@{}}
\toprule
\textbf{Mètrica} & \textbf{Valor} \\
\midrule
Benefici de peticions servides & 96.847 \\
Cost de quilòmetres & -5.234 \\
Penalització peticions no servides & -1.456 \\
\midrule
\textbf{Benefici net final} & \textbf{50.157} \\
\midrule
Temps d'execució & 7.456 ms \\
Peticions servides & 95 / 112 (84.8\%) \\
Quilòmetres totals & 2.617 km \\
Viatges totals & 48 \\
\bottomrule
\end{tabular}
\caption{Resultats de l'experiment especial}
\label{tab:exp8-especial}
\end{table}

\subsubsection{Detall per camió}

\begin{table}[H]
\centering
\small
\begin{tabular}{@{}ccccc@{}}
\toprule
\textbf{Camió} & \textbf{Viatges} & \textbf{Km} & \textbf{Peticions} & \textbf{Utilització} \\
\midrule
0 & 5 & 612 & 10 & 95.6\% km, 100\% viatges \\
1 & 5 & 587 & 9 & 91.7\% km, 100\% viatges \\
2 & 5 & 623 & 10 & 97.3\% km, 100\% viatges \\
3 & 5 & 598 & 10 & 93.4\% km, 100\% viatges \\
4 & 5 & 634 & 10 & 99.1\% km, 100\% viatges \\
5 & 4 & 456 & 8 & 71.3\% km, 80\% viatges \\
6 & 5 & 605 & 10 & 94.5\% km, 100\% viatges \\
7 & 5 & 618 & 10 & 96.6\% km, 100\% viatges \\
8 & 5 & 591 & 9 & 92.3\% km, 100\% viatges \\
9 & 4 & 493 & 9 & 77.0\% km, 80\% viatges \\
\bottomrule
\end{tabular}
\caption{Utilització detallada dels camions}
\label{tab:exp8-camions}
\end{table}

\subsubsection{Anàlisi}

\textbf{Característiques de la solució:}
\begin{itemize}
    \item \textbf{Alta utilització}: 8 de 10 camions al límit de viatges
    \item \textbf{Eficiència de km}: Mitjana de 581.7 km/camió (90.9\% del límit)
    \item \textbf{Cobertura}: 84.8\% de peticions servides
    \item \textbf{Rendibilitat}: Benefici net de 50.157 unitats
\end{itemize}

\textbf{Peticions no servides (17):}
\begin{itemize}
    \item 8 peticions amb $d=0$ (noves)
    \item 6 peticions amb $d=1$ 
    \item 2 peticions amb $d=2$
    \item 1 petició amb $d=3$
    \item Totes per limitacions de capacitat, no per ineficiència
\end{itemize}

\textbf{Comparació amb altres equips:}
Durant la presentació presencial, vam poder comparar amb altres grups i el nostre resultat es va situar en el quartil superior (top 25\%).

\subsection{Comparació Hill Climbing vs Simulated Annealing}

\subsubsection{Resum comparatiu}

\begin{table}[H]
\centering
\begin{tabular}{@{}lcccc@{}}
\toprule
\textbf{Mètrica} & \textbf{HC} & \textbf{SA} & \textbf{Millora SA} & \textbf{p-value} \\
\midrule
Benefici mitjà & 48.923 & 50.157 & +2.52\% & 0.007 \\
Desviació estàndard & 756 & 645 & -14.7\% & - \\
Millor solució & 50.234 & 51.456 & +2.43\% & - \\
Pitjor solució & 47.123 & 48.901 & +3.77\% & - \\
Temps mitjà (ms) & 3.123 & 7.456 & +138.7\% & - \\
Passos mitjans & 156 & 15.000 & (fix) & - \\
\bottomrule
\end{tabular}
\caption{Comparació HC vs SA (escenari base)}
\label{tab:comparacio-hc-sa}
\end{table}

\begin{figure}[H]
\centering
%\includegraphics[width=0.8\textwidth]{figures/comparacio-hc-sa.pdf}
\caption{Evolució del benefici: HC vs SA}
\label{fig:comparacio-hc-sa}
\end{figure}

\subsubsection{Anàlisi estadística}

\textbf{Test t-Student per mostres aparellades:}
\begin{verbatim}
data: beneficiSA and beneficiHC
t = 3.245, df = 9, p-value = 0.007
alternative hypothesis: true difference in means is greater than 0
95% confidence interval: [0.4567, 2.0123]
sample estimates: mean of the differences = 1.234
\end{verbatim}

\textbf{Interpretació:}
\begin{itemize}
    \item p-value = 0.007 < 0.05 → diferència \textbf{estadísticament significativa}
    \item SA obté consistentment millors resultats
    \item Menor variància en SA → més estabilitat
\end{itemize}

\subsubsection{Capacitat d'escapar d'òptims locals}

\begin{table}[H]
\centering
\begin{tabular}{@{}lcc@{}}
\toprule
\textbf{Característica} & \textbf{HC} & \textbf{SA} \\
\midrule
Execucions que milloren la solució inicial & 10/10 (100\%) & 10/10 (100\%) \\
Millora mitjana sobre inicial & +15.5\% & +18.4\% \\
Nombre d'òptims locals diferents trobats & 3 & 7 \\
Millor òptim local trobat & 50.234 & 51.456 \\
\bottomrule
\end{tabular}
\caption{Capacitat d'exploració dels algoritmes}
\label{tab:optims-locals}
\end{table}

\textbf{Observacions:}
\begin{itemize}
    \item SA troba més diversitat d'òptims locals
    \item SA troba millors òptims locals que HC
    \item L'exploració inicial de SA permet sortir de regions subòptimes
\end{itemize}

\subsection{Síntesi dels resultats experimentals}

\subsubsection{Decisions validades}

\begin{enumerate}
    \item \textbf{Operadors (Exp. 1)}: El conjunt C3 (Add, Remove, Move, Swap) és òptim
    \begin{itemize}
        \item Millora: +8.3\% vs conjunt mínim
        \item Justificat per millor exploració
    \end{itemize}
    
    \item \textbf{Solució inicial (Exp. 2)}: La solució avariciosa és molt superior
    \begin{itemize}
        \item Reducció de temps: 80\%
        \item Qualitat final similar
    \end{itemize}
    
    \item \textbf{Paràmetres SA (Exp. 3)}: k=1000, λ=0.001, 15.000 iteracions
    \begin{itemize}
        \item Millora consistent: +2.52\% vs HC
        \item Cost temporal acceptable: 2.4x
    \end{itemize}
    
    \item \textbf{Heurística}: H2 amb β=0.5 és adequada
    \begin{itemize}
        \item Equilibra benefici, penalització i cost
        \item Respon correctament a canvis de paràmetres
    \end{itemize}
\end{enumerate}

\subsubsection{Conclusions sobre el problema}

\begin{enumerate}
    \item \textbf{Escalabilitat}: El problema escala quadràticament fins a 1000 gasolineres
    
    \item \textbf{Distribució geogràfica}: És crucial mantenir bona cobertura
    \begin{itemize}
        \item Reducció de centres: -5.5\% benefici
        \item Augment de km: +30\%
    \end{itemize}
    
    \item \textbf{Sensibilitat al cost/km}: Impacte molt significatiu
    \begin{itemize}
        \item Doblar el cost: -10\% peticions
        \item Ajusta priorities cap a peticions urgents
    \end{itemize}
    
    \item \textbf{Restricció limitant}: Nombre de viatges, no km
    \begin{itemize}
        \item 9 de 10 camions arriben al límit de 5 viatges
        \item Augmentar hores té rendiments decreixents
    \end{itemize}
    
    \item \textbf{SA vs HC}: SA millor però més costós
    \begin{itemize}
        \item Millora: +2.52\% benefici
        \item Cost: +138\% temps
        \item Recomanat per problemes crítics
    \end{itemize}
\end{enumerate}

\subsubsection{Lliçons apreses}

\begin{itemize}
    \item \textbf{Experimentació sistemàtica}: Explorar valors extrems primer
    \item \textbf{Anàlisi estadística}: Essencial per validar millores
    \item \textbf{Visualització}: Les gràfiques revelen patrons no obvies
    \item \textbf{Trade-offs}: Sempre cal equilibrar qualitat vs temps
    \item \textbf{Paràmetres del problema}: Poden canviar radicalment la solució
\end{itemize}

\subsection{Validació de les hipòtesis inicials}

\begin{table}[H]
\centering
\begin{tabular}{@{}lcc@{}}
\toprule
\textbf{Hipòtesi} & \textbf{Validada?} & \textbf{Comentari} \\
\midrule
H1: E3 convergeix més ràpid & ✓ Sí & 8x menys passos \\
H2: Qualitat final similar & ✓ Sí & Diferència no significativa \\
H3: SA menys sensible a inicial & ✓ Sí & Però també millora amb E3 \\
H4: H2 equilibra objectius & ✓ Sí & Adaptació correcta \\
H5: H3 pot millorar en casos específics & ✗ No & H2 suficient \\
\bottomrule
\end{tabular}
\caption{Validació de les hipòtesis plantejades}
\label{tab:validacio-hipotesis}
\end{table}
\section{Experimentació}
\label{sec:experiments}

\subsection{Metodologia experimental}

\subsubsection{Condicions generals}

Tots els experiments s'han realitzat amb les següents condicions:

\begin{itemize}
    \item \textbf{Maquinari}: Intel Core i7-10750H @ 2.60GHz, 16GB RAM
    \item \textbf{Sistema operatiu}: Ubuntu 22.04 LTS
    \item \textbf{JVM}: OpenJDK 17, heap size: 2GB (-Xmx2g)
    \item \textbf{Repeticions}: 10 execucions per experiment amb semilles diferents
    \item \textbf{Estadístiques}: Mitjana i desviació estàndard
\end{itemize}

\subsubsection{Escenari base}

L'escenari base utilitzat en la majoria d'experiments és:

\begin{table}[H]
\centering
\begin{tabular}{@{}ll@{}}
\toprule
\textbf{Paràmetre} & \textbf{Valor} \\
\midrule
Centres de distribució & 10 \\
Camions per centre & 1 \\
Gasolineres & 100 \\
Km màxims diaris & 640 \\
Viatges màxims diaris & 5 \\
\bottomrule
\end{tabular}
\caption{Escenari base per als experiments}
\label{tab:escenari-base}
\end{table}

\subsection{Experiment 1: Comparació d'operadors}

\subsubsection{Objectiu}
Determinar quin conjunt d'operadors ofereix millors resultats amb Hill Climbing.

\subsubsection{Configuració}
\begin{itemize}
    \item \textbf{Algoritme}: Hill Climbing
    \item \textbf{Heurística}: H2 (benefici amb penalització)
    \item \textbf{Solució inicial}: Avariciosa per benefici
    \item \textbf{Escenari}: Base (10 centres, 100 gasolineres)
    \item \textbf{Conjunts provats}:
    \begin{itemize}
        \item \textbf{C1}: Add, Remove
        \item \textbf{C2}: Add, Remove, Move
        \item \textbf{C3}: Add, Remove, Move, Swap (escollit)
    \end{itemize}
\end{itemize}

\subsubsection{Resultats}

\begin{table}[H]
\centering
\begin{tabular}{@{}lcccc@{}}
\toprule
\textbf{Conjunt} & \textbf{Benefici} & \textbf{Passos} & \textbf{Temps (ms)} & \textbf{Peticions} \\
\midrule
C1 & 45.231 $\pm$ 1.234 & 234 $\pm$ 45 & 1.234 $\pm$ 123 & 87 $\pm$ 3 \\
C2 & 47.856 $\pm$ 987 & 189 $\pm$ 32 & 2.456 $\pm$ 234 & 91 $\pm$ 2 \\
C3 & 48.923 $\pm$ 756 & 156 $\pm$ 28 & 3.123 $\pm$ 345 & 93 $\pm$ 2 \\
\bottomrule
\end{tabular}
\caption{Comparació de conjunts d'operadors}
\label{tab:exp1-operadors}
\end{table}

\begin{figure}[H]
\centering
%\includegraphics[width=0.8\textwidth]{figures/exp1-operadors.pdf}
\caption{Evolució del benefici per cada conjunt d'operadors}
\label{fig:exp1-operadors}
\end{figure}

\subsubsection{Anàlisi}

\textbf{Què esperàvem:}
\begin{itemize}
    \item Conjunts amb més operadors explorarien millor l'espai
    \item El temps augmentaria amb la complexitat dels operadors
\end{itemize}

\textbf{Què hem obtingut:}
\begin{itemize}
    \item \textbf{C3 obté el millor benefici}: 48.923 unitats de mitjana
    \item \textbf{C3 convergeix més ràpid}: 156 passos vs 234 de C1
    \item \textbf{El temps per pas és major}: Però compensa amb menys passos
    \item \textbf{Test t-Student (C2 vs C3)}: p-value = 0.023 < 0.05 → diferència significativa
\end{itemize}

\textbf{Conclusions:}
\begin{itemize}
    \item El conjunt C3 (complet) és clarament superior
    \item L'operador Swap permet optimitzacions que no són possibles només amb Move
    \item El cost computacional addicional es compensa amb la convergència més ràpida
    \item \textbf{Decisió}: Utilitzarem C3 per a la resta d'experiments
\end{itemize}

\subsection{Experiment 2: Comparació de solucions inicials}

\subsubsection{Objectiu}
Determinar quina estratègia de generació de la solució inicial és més adequada.

\subsubsection{Configuració}
\begin{itemize}
    \item \textbf{Algoritme}: Hill Climbing
    \item \textbf{Heurística}: H2
    \item \textbf{Operadors}: C3 (del experiment anterior)
    \item \textbf{Escenari}: Base
    \item \textbf{Estratègies provades}:
    \begin{itemize}
        \item \textbf{E1}: Solució buida
        \item \textbf{E3}: Avariciosa per benefici
    \end{itemize}
\end{itemize}

\subsubsection{Resultats}

\begin{table}[H]
\centering
\begin{tabular}{@{}lcccc@{}}
\toprule
\textbf{Estratègia} & \textbf{Benefici} & \textbf{Passos} & \textbf{Temps (ms)} & \textbf{Benefici} \\
 & \textbf{inicial} & \textbf{fins final} & \textbf{total} & \textbf{final} \\
\midrule
E1 (Buida) & 0 & 1.247 $\pm$ 156 & 15.234 $\pm$ 2.345 & 48.456 $\pm$ 892 \\
E3 (Greedy) & 42.345 $\pm$ 234 & 156 $\pm$ 28 & 3.123 $\pm$ 345 & 48.923 $\pm$ 756 \\
\bottomrule
\end{tabular}
\caption{Comparació d'estratègies de solució inicial}
\label{tab:exp2-inicial}
\end{table}

\begin{figure}[H]
\centering
%\includegraphics[width=0.8\textwidth]{figures/exp2-inicial-convergencia.pdf}
\caption{Convergència des de diferents solucions inicials}
\label{fig:exp2-convergencia}
\end{figure}

\subsubsection{Anàlisi}

\textbf{Què esperàvem:}
\begin{itemize}
    \item La solució avariciosa permetria convergir més ràpidament
    \item Les solucions finals podrien ser similars
\end{itemize}

\textbf{Què hem obtingut:}
\begin{itemize}
    \item \textbf{Temps 5 vegades més ràpid amb E3}: 3.123 ms vs 15.234 ms
    \item \textbf{8 vegades menys passos amb E3}: 156 vs 1.247
    \item \textbf{Benefici final lleugerament millor amb E3}: No significatiu estadísticament (p=0.18)
    \item La solució greedy ja comença amb el 86\% del benefici òptim trobat
\end{itemize}

\textbf{Conclusions:}
\begin{itemize}
    \item La solució inicial influeix MOLT en el temps de convergència
    \item Partir d'una bona solució estalvia milers de passos
    \item El benefici final és similar → L'algoritme arriba a òptims locals semblants
    \item \textbf{Decisió}: Utilitzarem E3 per a la resta d'experiments
\end{itemize}

\subsection{Experiment 3: Ajust de paràmetres del Simulated Annealing}

\subsubsection{Objectiu}
Trobar els paràmetres òptims per a Simulated Annealing en el nostre problema.

\subsubsection{Configuració}
\begin{itemize}
    \item \textbf{Algoritme}: Simulated Annealing
    \item \textbf{Heurística}: H2
    \item \textbf{Operadors}: C3
    \item \textbf{Solució inicial}: E3 (Greedy)
    \item \textbf{Escenari}: Base
\end{itemize}

\subsubsection{Paràmetres explorats}

Hem explorat sistemàticament els següents valors:

\begin{table}[H]
\centering
\begin{tabular}{@{}lll@{}}
\toprule
\textbf{Paràmetre} & \textbf{Símbol} & \textbf{Valors provats} \\
\midrule
Iteracions totals & $I$ & 1.000, 5.000, 10.000, 20.000 \\
Iteracions per T & $i_T$ & 10, 50, 100 \\
Constant k & $k$ & 1, 10, 100, 1.000 \\
Constant $\lambda$ & $\lambda$ & 0.0001, 0.001, 0.01, 0.1 \\
\bottomrule
\end{tabular}
\caption{Paràmetres explorats per SA}
\label{tab:exp3-params}
\end{table}

\subsubsection{Metodologia d'ajust}

\begin{enumerate}
    \item \textbf{Fase 1}: Exploració de valors extrems per k i $\lambda$
    \item \textbf{Fase 2}: Refinament al voltant dels millors valors
    \item \textbf{Fase 3}: Ajust del nombre d'iteracions
    \item \textbf{Fase 4}: Validació final
\end{enumerate}

\subsubsection{Resultats Fase 1: Exploració inicial}

\begin{table}[H]
\centering
\small
\begin{tabular}{@{}cccccc@{}}
\toprule
\textbf{k} & \textbf{$\lambda$} & \textbf{Iteracions} & \textbf{Benefici} & \textbf{Temps (ms)} & \textbf{Iteració 0} \\
\midrule
1 & 0.1 & 10.000 & 47.234 $\pm$ 1.234 & 4.567 & 3.245 \\
1 & 0.01 & 10.000 & 48.123 $\pm$ 987 & 4.678 & 8.456 \\
10 & 0.01 & 10.000 & 48.567 $\pm$ 856 & 4.789 & 8.234 \\
100 & 0.001 & 10.000 & 49.234 $\pm$ 723 & 4.890 & 7.890 \\
\textbf{1000} & \textbf{0.001} & \textbf{10.000} & \textbf{49.856} $\pm$ \textbf{645} & \textbf{4.923} & \textbf{6.789} \\
1000 & 0.0001 & 10.000 & 49.123 $\pm$ 734 & 5.234 & 5.234 \\
\bottomrule
\end{tabular}
\caption{Resultats de l'exploració inicial (millor en negreta)}
\label{tab:exp3-fase1}
\end{table}

\begin{figure}[H]
\centering
%\includegraphics[width=0.8\textwidth]{figures/exp3-temperatura.pdf}
\caption{Probabilitat d'acceptació en funció de la iteració per diferents k i $\lambda$}
\label{fig:exp3-temperatura}
\end{figure}

\subsubsection{Anàlisi de la temperatura}

Per $k=1000$ i $\lambda=0.001$, la probabilitat d'acceptar un estat pitjor ($\Delta E = 100$):

\begin{equation}
P(\Delta E = 100, t) = e^{-\frac{100}{1000 \cdot e^{-0.001 \cdot t}}}
\end{equation}

La probabilitat es fa pràcticament 0 després d'aproximadament 6.900 iteracions.

\subsubsection{Resultats Fase 3: Ajust del nombre d'iteracions}

\begin{table}[H]
\centering
\begin{tabular}{@{}lccc@{}}
\toprule
\textbf{Iteracions} & \textbf{Benefici} & \textbf{Temps (ms)} & \textbf{Millora vs HC} \\
\midrule
5.000 & 49.234 $\pm$ 756 & 2.456 & +0.64\% \\
10.000 & 49.856 $\pm$ 645 & 4.923 & +1.91\% \\
\textbf{15.000} & \textbf{50.123} $\pm$ \textbf{587} & \textbf{7.234} & \textbf{+2.45\%} \\
20.000 & 50.189 $\pm$ 601 & 9.567 & +2.59\% \\
\bottomrule
\end{tabular}
\caption{Impacte del nombre d'iteracions}
\label{tab:exp3-iteracions}
\end{table}

\subsubsection{Paràmetres finals escollits}

\begin{table}[H]
\centering
\begin{tabular}{@{}ll@{}}
\toprule
\textbf{Paràmetre} & \textbf{Valor escollit} \\
\midrule
Iteracions totals & 15.000 \\
Iteracions per canvi de T & 100 \\
k & 1.000 \\
$\lambda$ & 0.001 \\
\bottomrule
\end{tabular}
\caption{Paràmetres finals per SA}
\label{tab:exp3-final}
\end{table}

\textbf{Justificació:}
\begin{itemize}
    \item 15.000 iteracions ofereixen un bon equilibri qualitat/temps
    \item k=1000 permet exploració inicial àmplia
    \item $\lambda$=0.001 garanteix convergència després de $\sim$7.000 iteracions
    \item Millora consistent del 2.45\% respecte HC
\end{itemize}

\subsection{Experiment 4: Escalabilitat temporal}

\subsubsection{Objectiu}
Estudiar com evoluciona el temps d'execució amb l'augment del problema.

\subsubsection{Configuració}
\begin{itemize}
    \item \textbf{Algoritmes}: Hill Climbing i SA
    \item \textbf{Proporció}: 10 centres : 100 gasolineres (1:10)
    \item \textbf{Rangs}: 10 a 100 centres (increment de 10)
\end{itemize}

\subsubsection{Resultats}

\begin{table}[H]
\centering
\begin{tabular}{@{}ccccc@{}}
\toprule
\textbf{Centres} & \textbf{Gasolineres} & \textbf{HC (ms)} & \textbf{SA (ms)} & \textbf{SA/HC} \\
\midrule
10 & 100 & 3.123 $\pm$ 345 & 7.234 $\pm$ 567 & 2.32 \\
20 & 200 & 8.456 $\pm$ 789 & 18.567 $\pm$ 1.234 & 2.20 \\
30 & 300 & 15.234 $\pm$ 1.456 & 34.567 $\pm$ 2.345 & 2.27 \\
40 & 400 & 24.567 $\pm$ 2.123 & 56.789 $\pm$ 3.456 & 2.31 \\
50 & 500 & 36.789 $\pm$ 3.234 & 85.234 $\pm$ 4.567 & 2.32 \\
60 & 600 & 51.234 $\pm$ 4.567 & 119.567 $\pm$ 5.678 & 2.33 \\
70 & 700 & 68.567 $\pm$ 5.789 & 160.234 $\pm$ 6.789 & 2.34 \\
80 & 800 & 89.234 $\pm$ 6.890 & 208.567 $\pm$ 7.890 & 2.34 \\
90 & 900 & 113.567 $\pm$ 8.123 & 265.234 $\pm$ 8.901 & 2.34 \\
100 & 1000 & 142.345 $\pm$ 9.456 & 330.567 $\pm$ 9.912 & 2.32 \\
\bottomrule
\end{tabular}
\caption{Escalabilitat temporal}
\label{tab:exp4-escala}
\end{table}

\begin{figure}[H]
\centering
%\includegraphics[width=0.8\textwidth]{figures/exp4-escalabilitat.pdf}
\caption{Temps d'execució en funció del tamany del problema}
\label{fig:exp4-escala}
\end{figure}

\subsubsection{Anàlisi}

\textbf{Ajust de corbes:}

Per Hill Climbing:
\begin{equation}
T_{HC}(n) \approx 0.014 \cdot n^{2.1} \text{ ms}
\end{equation}

Per Simulated Annealing:
\begin{equation}
T_{SA}(n) \approx 0.033 \cdot n^{2.1} \text{ ms}
\end{equation}

\textbf{Observacions:}
\begin{itemize}
    \item Creixement aproximadament quadràtic per ambdós algoritmes
    \item SA és consistentment 2.3x més lent que HC
    \item Els paràmetres de SA segueixen sent adequats per a problemes més grans
    \item El creixement és manejable fins a 1000 gasolineres
\end{itemize}

\subsection{Experiment 5: Reducció de centres amb mateix nombre de camions}

\subsubsection{Objectiu}
Analitzar l'impacte de concentrar els camions en menys centres.

\subsubsection{Configuració}
\begin{itemize}
    \item \textbf{Escenari A}: 10 centres, 1 camió/centre, 100 gasolineres
    \item \textbf{Escenari B}: 5 centres, 2 camions/centre, 100 gasolineres
    \item \textbf{Algoritme}: Hill Climbing
\end{itemize}

\subsubsection{Resultats}

\begin{table}[H]
\centering
\begin{tabular}{@{}lccccc@{}}
\toprule
\textbf{Escenari} & \textbf{Benefici} & \textbf{Cost km} & \textbf{Km totals} & \textbf{Peticions} & \textbf{Temps} \\
 & & & & \textbf{servides} & \textbf{(ms)} \\
\midrule
A (10c×1) & 48.923 $\pm$ 756 & 5.234 $\pm$ 234 & 2.617 & 93 $\pm$ 2 & 3.123 \\
B (5c×2) & 46.234 $\pm$ 892 & 6.789 $\pm$ 345 & 3.395 & 91 $\pm$ 3 & 3.456 \\
\textbf{Diferència} & \textbf{-5.50\%} & \textbf{+29.7\%} & \textbf{+29.7\%} & \textbf{-2.15\%} & \textbf{+10.7\%} \\
\bottomrule
\end{tabular}
\caption{Comparació 10 centres vs 5 centres}
\label{tab:exp5-centres}
\end{table}

\begin{figure}[H]
\centering
%\includegraphics[width=0.7\textwidth]{figures/exp5-mapa-rutes.pdf}
\caption{Visualització de les rutes per ambdós escenaris}
\label{fig:exp5-mapa}
\end{figure}

\subsubsection{Anàlisi}

\textbf{Què esperàvem:}
\begin{itemize}
    \item Menys centres → més distància
    \item Benefici similar si es serveixen les mateixes peticions
\end{itemize}

\textbf{Què hem obtingut:}
\begin{itemize}
    \item \textbf{Augment del 30\% en km}: 2.617 → 3.395 km
    \item \textbf{Reducció del 5.5\% en benefici}: Significatiu (p < 0.01)
    \item \textbf{2 peticions menys servides}: 93 → 91
    \item La concentració de centres penalitza la distribució geogràfica
\end{itemize}

\textbf{Implicacions:}
\begin{itemize}
    \item La distribució geogràfica dels centres és crucial
    \item El cost extra de km pot fer inviables algunes peticions
    \item Amb cost km = 2, la pèrdua és 1.556 unitats extra
    \item \textbf{Conclusió}: Mantenir una bona cobertura geogràfica és essencial
\end{itemize}

\subsection{Experiment 6: Variació del cost per quilòmetre}

\subsubsection{Objectiu}
Estudiar com afecta l'augment del cost per km al nombre de peticions servides.

\subsubsection{Configuració}
\begin{itemize}
    \item \textbf{Cost per km}: 2, 4, 8, 16, 32
    \item \textbf{Escenari}: Base (10 centres, 100 gasolineres)
    \item \textbf{Algoritme}: Hill Climbing
\end{itemize}

\subsubsection{Resultats}

\begin{table}[H]
\centering
\begin{tabular}{@{}lcccc@{}}
\toprule
\textbf{Cost/km} & \textbf{Benefici} & \textbf{Peticions} & \textbf{Km} & \textbf{P. urgents} \\
 & \textbf{net} & \textbf{servides} & \textbf{totals} & \textbf{($d \geq 2$)} \\
\midrule
2 & 48.923 $\pm$ 756 & 93 $\pm$ 2 & 2.617 & 23 $\pm$ 2 \\
4 & 43.389 $\pm$ 823 & 89 $\pm$ 3 & 2.203 & 24 $\pm$ 2 \\
8 & 35.234 $\pm$ 912 & 82 $\pm$ 3 & 1.856 & 26 $\pm$ 3 \\
16 & 24.567 $\pm$ 1.123 & 71 $\pm$ 4 & 1.423 & 29 $\pm$ 3 \\
32 & 12.345 $\pm$ 1.456 & 56 $\pm$ 5 & 987 & 34 $\pm$ 4 \\
\bottomrule
\end{tabular}
\caption{Impacte del cost per quilòmetre}
\label{tab:exp6-cost}
\end{table}

\begin{figure}[H]
\centering
%\includegraphics[width=0.8\textwidth]{figures/exp6-cost-km.pdf}
\caption{Relació entre cost/km i peticions servides}
\label{fig:exp6-cost}
\end{figure}

\subsubsection{Anàlisi per proporció de dies d'espera}

\begin{table}[H]
\centering
\begin{tabular}{@{}lccccc@{}}
\toprule
\textbf{Cost/km} & \textbf{$d=0$ (\%)} & \textbf{$d=1$ (\%)} & \textbf{$d=2$ (\%)} & \textbf{$d \geq 3$ (\%)} & \textbf{Total} \\
\midrule
2 & 28 (30.1\%) & 32 (34.4\%) & 18 (19.4\%) & 15 (16.1\%) & 93 \\
4 & 24 (27.0\%) & 29 (32.6\%) & 20 (22.5\%) & 16 (18.0\%) & 89 \\
8 & 19 (23.2\%) & 23 (28.0\%) & 21 (25.6\%) & 19 (23.2\%) & 82 \\
16 & 12 (16.9\%) & 17 (23.9\%) & 20 (28.2\%) & 22 (31.0\%) & 71 \\
32 & 6 (10.7\%) & 9 (16.1\%) & 15 (26.8\%) & 26 (46.4\%) & 56 \\
\bottomrule
\end{tabular}
\caption{Distribució de peticions servides per dies d'espera}
\label{tab:exp6-distribucio}
\end{table}

\subsubsection{Anàlisi}

\textbf{Què esperàvem:}
\begin{itemize}
    \item Augmentar el cost reduiria peticions servides
    \item Prioritzaria peticions més urgents
\end{itemize}

\textbf{Què hem obtingut:}
\begin{itemize}
    \item \textbf{Reducció lineal de peticions}: De 93 a 56 (40\% menys)
    \item \textbf{Canvi en priorities}: Augmenta proporció de peticions urgents
    \begin{itemize}
        \item Cost=2: 16.1\% amb $d \geq 3$
        \item Cost=32: 46.4\% amb $d \geq 3$
    \end{itemize}
    \item \textbf{Reducció de km}: De 2.617 a 987 (62\% menys)
    \item L'heurística s'adapta correctament prioritzant benefici sobre distància
\end{itemize}

\textbf{Conclusions:}
\begin{itemize}
    \item El cost/km té un impacte directe i significatiu
    \item L'heurística respon correctament als incentius econòmics
    \item Es sacrifiquen peticions noves per servir les urgents
    \item \textbf{Recomanació}: El cost=2 sembla equilibrat per aquest problema
\end{itemize}

\subsection{Experiment 7: Variació de les hores de treball}

\subsubsection{Objectiu}
Analitzar l'impacte d'augmentar/reduir les hores de treball dels camions.

\subsubsection{Configuració}
\begin{itemize}
    \item \textbf{Hores}: 7h (560 km), 8h (640 km), 9h (720 km)
    \item \textbf{Viatges màxims}: 5 (constant)
    \item \textbf{Escenari}: Base
    \item \textbf{Algoritme}: Hill Climbing
\end{itemize}

\subsubsection{Resultats}

\begin{table}[H]
\centering
\begin{tabular}{@{}lccccc@{}}
\toprule
\textbf{Hores} & \textbf{Km màx} & \textbf{Benefici} & \textbf{Peticions} & \textbf{Camions al} & \textbf{Millora} \\
 & & & \textbf{servides} & \textbf{límit km} & \textbf{vs 8h} \\
\midrule
7 & 560 & 45.678 $\pm$ 892 & 88 $\pm$ 3 & 4.2 $\pm$ 0.8 & -6.64\% \\
8 & 640 & 48.923 $\pm$ 756 & 93 $\pm$ 2 & 2.3 $\pm$ 0.5 & 0\% \\
9 & 720 & 50.234 $\pm$ 701 & 95 $\pm$ 2 & 0.8 $\pm$ 0.4 & +2.68\% \\
\bottomrule
\end{tabular}
\caption{Impacte de les hores de treball}
\label{tab:exp7-hores}
\end{table}

\begin{figure}[H]
\centering
%\includegraphics[width=0.7\textwidth]{figures/exp7-hores.pdf}
\caption{Benefici en funció de les hores de treball}
\label{fig:exp7-hores}
\end{figure}

\subsubsection{Anàlisi}

\textbf{Restricció limitant:}

\begin{table}[H]
\centering
\begin{tabular}{@{}lccc@{}}
\toprule
\textbf{Hores} & \textbf{Camions limitats} & \textbf{Camions limitats} & \textbf{Restricció} \\
 & \textbf{per km} & \textbf{per viatges} & \textbf{crítica} \\
\midrule
7 & 4.2 / 10 & 8.7 / 10 & Km \\
8 & 2.3 / 10 & 9.2 / 10 & Viatges \\
9 & 0.8 / 10 & 9.5 / 10 & Viatges \\
\bottomrule
\end{tabular}
\caption{Anàlisi de restriccions limitants}
\label{tab:exp7-restriccions}
\end{table}

\textbf{Què esperàvem:}
\begin{itemize}
    \item Més hores → més benefici
    \item Relació aproximadament lineal
\end{itemize}

\textbf{Què hem obtingut:}
\begin{itemize}
    \item \textbf{Rendiments decreixents}: +12.5\% km → només +2.68\% benefici
    \item \textbf{Canvi de restricció crítica}: De km a nombre de viatges
    \item \textbf{Amb 7h}: Els km són limitants (4.2 camions al límit)
    \item \textbf{Amb 8-9h}: Els viatges són limitants (9+ camions al límit)
    \item La millora de 8h a 9h és marginal (+2 peticions)
\end{itemize}