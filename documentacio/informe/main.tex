\documentclass[12pt]{article}
\usepackage[a4paper, left = 1in, right = 1in, top = 1in, bottom = 1in]{geometry}
\usepackage[utf8]{inputenc}
\usepackage[catalan]{babel}
\usepackage[T1]{fontenc}
\usepackage{geometry}
\usepackage{graphicx}
\usepackage{amsmath}
\usepackage{amssymb}
\usepackage{hyperref}
\usepackage{fancyhdr}
\usepackage{listings}
\usepackage{xcolor}
\usepackage{caption}
\usepackage{subcaption}
\usepackage{booktabs}
\usepackage{algorithm}
\usepackage{algpseudocode}
\usepackage{float}
\usepackage{emptypage}

% Dibuix i gràfics
\usepackage{tikz}          % Gràfics vectorials en general
\usepackage{pgfplots}      % Gràfics d'eixos, boxplots, scatter, etc.

% Configuració recomanada per a pgfplots
\pgfplotsset{compat=1.18}  % Usa la versió més recent disponible
\usepgfplotslibrary{statistics}  % Necessari per als boxplots
\usepgfplotslibrary{colorbrewer} % (opcional) paletes de colors millors
\usepgfplotslibrary{fillbetween} % (opcional) per zones ombrejades

% Altres paquets útils per al teu informe
\usepackage{float}         % Permet usar [H] per fixar figures
\usepackage{caption}       % Control de llegendes
\usepackage{subcaption}    % Figures múltiples en una sola (si ho necessites)
\usepackage{xcolor}        % Colors millorats


% Colors i estil general dels gràfics
\pgfplotsset{
    every axis/.append style={
        tick style={semithick},
        axis line style={semithick},
        label style={font=\small},
        tick label style={font=\small},
    },
    boxplot/every box/.style={
        draw=black,
        fill=white,
        line width=0.7pt,
        solid
    },
    boxplot/every whisker/.style={
        solid,               % línies contínues
        semithick,
        draw=black
    },
    boxplot/every median/.style={
        very thick,
        black,               % color de la mediana
        solid                % evita patrons discontinus
    },
    boxplot/every outlier/.style={
        mark=o,
        mark size=1pt,
        draw=black
    }
}




% Configuració de la geometria
\geometry{
    left=2.5cm,
    right=2.5cm,
    top=3cm,
    bottom=3cm
}

% Configuració de capçaleres i peus
\pagestyle{fancy}
\fancyhf{}
\fancyhead[L]{\leftmark}
\fancyhead[R]{\thepage}
\renewcommand{\headrulewidth}{0.4pt}
\setlength{\headheight}{15.5pt}

% Configuració de listings per codi Java
\lstdefinestyle{javastyle}{
    language=Java,
    basicstyle=\ttfamily\small,
    keywordstyle=\color{blue}\bfseries,
    commentstyle=\color{green!60!black}\itshape,
    stringstyle=\color{red},
    numbers=left,
    numberstyle=\tiny\color{gray},
    stepnumber=1,
    numbersep=8pt,
    backgroundcolor=\color{white},
    showspaces=false,
    showstringspaces=false,
    showtabs=false,
    frame=single,
    tabsize=2,
    captionpos=b,
    breaklines=true,
    breakatwhitespace=false,
    escapeinside={(*@}{@*)},
    xleftmargin=2em,
    framexleftmargin=1.5em
}
\lstset{style=javastyle}

% Configuració de hyperref
\hypersetup{
    colorlinks=true,
    linkcolor=black,
    filecolor=magenta,
    urlcolor=cyan,
    citecolor=blue,
    pdftitle={Pràctica de Búsqueda Local},
    pdfauthor={Nom dels autors}
}

% Informació del document
\title{
    \includegraphics[width=0.3\textwidth]{figures/upc-logo.png}\\[1cm]
    {\LARGE Pràctica de Cerca Local}\\[0.5cm]
    {\Large Planificació de Rutes de Proveïment de Benzineres}\\[0.5cm]

}

\author{
\large
\textsc{Grau IA -- Q1 Curs 2025-2026}\\[2mm] 
\normalsize Departament de Ciències de la Computació \\
\normalsize Universitat Politècnica de Catalunya \\[1cm]
%
\begin{tabular}{cc}
{\textsc{Anel Ademovic Suljic}} & {\textsc{Aleix Pitarch}} \\
\normalsize \texttt{anel.suljic@estudiantat.upc.edu} & \normalsize \texttt{aleix.pitarch@estudiantat.upc.edu} \\[4mm]
\multicolumn{2}{c}{\textsc{Joan Solina}} \\
\multicolumn{2}{c}{\normalsize \texttt{joan.solina@estudiantat.upc.edu}}\\
\end{tabular}
}

\date{26 d'octubre de 2025}

\pagestyle{plain}
\setlength{\parindent}{0pt}

\begin{document}

\maketitle
\newpage
\tableofcontents


% Capítols principals
\newpage
\section{Introducció}
\label{sec:introduction}

\subsection{Context del problema }
Aquesta pràctica se centra en la resolució d'un problema de planificació de rutes de proveïment per a una companyia de distribució de combustible mitjançant algorismes de cerca local. La pràctica implica el disseny i la implementació dels elements necessaris per a algorismes Hill Climbing i Simulated Annealing així com la realització d’experiments per avaluar el seu comportament en diferents escenaris. A més, es durà a terme una anàlisi comparativa dels resultats obtinguts amb ambdós algorismes, amb l’objectiu d’extreure’n conclusions rellevants.

\subsection{Objectius}
Els objectius principals d'aquesta pràctica són:

\begin{itemize}
    \item Modelar el problema com un problema de búsqueda local
    \item Implementar una representació eficient de l'espai d'estats
    \item Dissenyar operadors adequats per explorar l'espai de solucions
    \item Definir funcions heurístiques que optimitzin els criteris del problema
    \item Experimentar amb els algoritmes Hill Climbing i Simulated Annealing
    \item Analitzar i comparar els resultats obtinguts
\end{itemize}

\newpage

\section{Definició del problema}
\label{sec:problem}

\vspace{0.5cm}

\subsection{Introducció al problema}

El problema que ens ocupa neix d'una necessitat real en el sector de la distribució de combustibles. Imaginem una companyia que opera amb diversos centres de distribució des d'on surten camions cisterna per proveir a les gasolineres. Cada dia, la companyia rep peticions de proveïment de diferents gasolineres i ha de decidir com organitzar els seus recursos per atendre-les de la manera més eficient possible.

Aquest problema presenta diversos factors que cal optimitzar simultàniament. Per una banda, la companyia vol maximitzar els seus ingressos servint el màxim nombre de peticions possibles, especialment aquelles que porten més temps pendents. Per altra banda, cal minimitzar els costos operatius, principalment els derivats dels quilòmetres recorreguts pels camions. Tot això s'ha de fer respectant les restriccions sobre la capacitat dels camions i les hores de treball dels conductors.

\vspace{0.5cm}

\subsection{Elements del problema}

Per comprendre completament el problema, cal primer identificar tots els elements que hi intervenen i com es relacionen entre ells. Aquestes relacions determinaran posteriorment com modelarem l'espai de cerca.

\vspace{0.5cm}

\subsubsection{Centres de distribució i camions cisterna}

La companyia disposa de diversos centres de distribució els quals estan situats en unes coordenades específiques dins d'una quadrícula de $100 \times 100$ km$^2$. Cada centre de distribució té assignat un camió cisterna el qual la seva capacitat és exactament el doble de la capacitat d'un dipòsit estàndard de gasolinera. Això significa que en cada viatge, un camió pot omplir com a màxim dos dipòsits, ja sigui de la mateixa gasolinera o de gasolineres diferents.

Cada dia, un camió pot recórrer un màxim de 640 quilòmetres, que corresponen a 8 hores de treball a una velocitat constant de 80 km/h. A més, cada camió pot fer un màxim de 5 viatges diaris, entenent per viatge el recorregut des del centre de distribució fins a les gasolineres assignades i tornada al centre.

\vspace{0.5cm}

\subsubsection{Gasolineres i peticions d'abastiment}

 Cada gasolinera disposa de diversos dipòsits per emmagatzemar combustible, i el seu mode d'operació és seqüencial: utilitzen un dipòsit fins que s'acaba, i aleshores passen al següent. Aquesta simplificació ens permet tractar cada dipòsit com una entitat independent.

Quan un dipòsit d'una gasolinera s'acaba, la gasolinera genera una petició de proveïment a la companyia de distribució. Aquestes peticions van acumulant-se dia rere dia fins que són ateses. És important destacar que una mateixa gasolinera pot tenir múltiples peticions pendents simultàniament si se li han acabat diversos dipòsits, fins a un màxim de 3.

Cada petició porta associat un comptador que indica quants dies porta pendent. Aquest comptador és crucial perquè determina el preu que la companyia cobra per atendre la petició. Una petició nova (0 dies d'espera) es cobra al 102\% del preu base, incentivant així un servei ràpid. A mesura que passen els dies sense ser atesa, el preu va disminuint exponencialment segons la fórmula que descriurem més endavant.

\vspace{0.5cm}

\subsubsection{Estructura d'un viatge}

Un viatge és la unitat bàsica d'operació en aquest problema. Cada viatge segueix sempre la mateixa estructura: el camió surt del centre de distribució amb el dipòsit ple, visita una o dues gasolineres on descarrega combustible, i torna al centre. Aquest cicle es repeteix tantes vegades com sigui necessari dins dels límits permesos.

La flexibilitat en el nombre de gasolineres per viatge (una o dues) dóna lloc a diferents estratègies. Un camió podria optar per fer viatges curts visitant una sola gasolinera propera per minimitzar distàncies, o podria intentar maximitzar l'ús de la seva capacitat visitant dues gasolineres en cada viatge. Aquesta decisió depèn de múltiples factors: la ubicació de les gasolineres, les peticions pendents, i els quilòmetres disponibles.

\vspace{0.5cm}

\subsection{Model de càlcul de distàncies}

Per calcular la distància entre qualsevol parell de punts (ja siguin centres de distribució o gasolineres), utilitzem la distància de Manhattan:

\begin{equation}
d(A, B) = |A_x - B_x| + |A_y - B_y|
\label{eq:distance}
\end{equation}

on $(A_x, A_y)$ i $(B_x, B_y)$ són les coordenades dels punts A i B respectivament, i $|\cdot|$ denota el valor absolut.

Per calcular la distància total d'un viatge que visita dues gasolineres $G_1$ i $G_2$ partint d'un centre de distribució $D$, calculem:

\begin{equation}
d_{\text{viatge}} = d(D, G_1) + d(G_1, G_2) + d(G_2, D)
\label{eq:distance-viatge}
\end{equation}

Si el viatge només visita una gasolinera, la distància es simplifica a:
\begin{equation}
d_{\text{viatge}} = 2 \times d(D, G_1)
\label{eq:distance-viatge-simple}
\end{equation}

\vspace{0.5cm}

\subsection{Justificació com a problema de cerca local}

És important entendre per què aquest problema és adequat per ser resolt amb tècniques de búsqueda local i no amb altres aproximacions.

En primer lloc, l'espai de solucions és exponencialment gran. Amb $n$ camions, $p$ peticions, i la possibilitat de fer fins a 5 viatges per camió, el nombre de possibles assignacions és de l'ordre de $(n \times 5)^p$. Per l'escenari base amb 10 camions i aproximadament 100 peticions, això suposa més de $10^{100}$ possibles solucions, fent inviable una exploració exhaustiva.

En segon lloc, no tenim un objectiu clarament definit com "arribar a un estat específic", sinó que volem optimitzar una funció de qualitat. Això és característic dels problemes d'optimització i els fa especialment adequats per búsqueda local.

En tercer lloc, existeix una noció natural de "veïnatge" entre solucions. Petits canvis en l'assignació de peticions a camions generen solucions similars amb valors de benefici també similars, cosa que permet una exploració gradual de l'espai.

Finalment, el problema exhibeix una estructura de "paisatge" amb múltiples òptims locals però on moure's en la direcció correcta generalment millora la solució. Aquesta propietat fa que algoritmes com Hill Climbing siguin efectius, tot i que poden quedar atrapats en òptims locals, motivant l'ús d'algoritmes més sofisticats com Simulated Annealing.

Totes aquestes característiques fan que el nostre problema sigui un candidat ideal per aplicar les tècniques de búsqueda local que hem estudiat a classe, i ens permeten explorar experimentalment les seves fortaleses i limitacions.
\newpage
\section{Representació de l'estat}
\label{sec:state}

\vspace{0.5cm}

\subsection{La importància de la representació}

La representació de l'estat és, sens dubte, una de les decisions de disseny més crítiques en qualsevol problema de cerca. Una bona representació no només facilita la implementació dels operadors i la funció heurística, sinó que també determina l'eficiència computacional de tota la solució. 

Quan vam començar a pensar en com representar un estat del problema, vam considerar diverses alternatives. Podríem haver representat una solució com una matriu d'assignacions, com una llista de tuples (petició, camió, viatge), o com una estructura més complexa que inclogués informació redundant per accelerar consultes. Després d'analitzar els pros i contres de cada opció, vam optar per una representació basada en llistes de viatges per camió, que expliquem en detall a continuació.

\vspace{0.5cm}

\subsection{Anàlisi previ de l'espai de cerca}

Abans de decidir com representar un estat, és fonamental comprendre la magnitud de l'espai amb el qual treballem. Aquesta anàlisi ens ajudarà a justificar les decisions que prendrem.

\vspace{0.5cm}

\subsubsection{Càlcul teòric de la mida de l'espai}

Considerem un escenari amb $n$ camions, $p$ peticions a servir, i la restricció que cada camió pot fer màxim $v$ viatges. Per cada petició, hem de decidir:

\begin{enumerate}
    \item Si la servim o no
    \item En cas afirmatiu, quin camió l'atendrà
    \item A quin viatge d'aquest camió s'assignarà
    \item En quin ordre dins del viatge
\end{enumerate}

Només considerant les assignacions de peticions a camions (ignorant l'ordre i l'agrupació en viatges), tenim $(n+1)^p$ possibilitats: cada petició pot ser assignada a qualsevol dels $n$ camions o quedar sense servir. Per al nostre escenari base amb 10 camions i aproximadament 100 peticions, això ja suposa $11^{100} \approx 10^{104}$ possibilitats.

\vspace{0.5cm}

\subsection{Estructura de dades escollida}

Després d'analitzar diverses alternatives, hem optat per una representació basada en llistes de viatges organitzades per camió. Aquesta decisió es fonamenta en diversos raonaments que exposem a continuació.

\vspace{0.5cm}

\subsubsection{Components de l'estat}

L'estat del nostre problema es compon de diverses parts que hem de distingir clarament entre dades estàtiques (compartides per tots els estats) i dades dinàmiques (específiques de cada estat).

\paragraph{Dades estàtiques:}

Les dades estàtiques són aquelles que no varien durant la cerca. Aquestes es declaren com a variables de classe (static en Java) i es comparteixen entre totes les instàncies d'estats:

\begin{lstlisting}[caption={Dades estàtiques de l'estat}, label={lst:static-data}]
public class PracticaBoard {
    // Informacio del problema (compartida per tots els estats)
    private static Gasolineras gasolineres;
    private static CentrosDistribucion centres;
    private static Map<Integer, Peticio> peticions;
    
    // Constants del problema
    private static final int KM_MAX = 640;
    private static final int VIATGES_MAX = 5;
    private static final double COST_KM = 2.0;
    private static final double PREU_BASE = 1000.0;
    
    // ...
}
\end{lstlisting}

Aquesta separació és crucial per l'eficiència espacial. Si tinguéssim 10.000 estats en memòria simultàniament (cosa que pot passar durant la cerca), duplicar aquesta informació en cada estat seria extremadament ineficient. Amb l'aproximació estàtica, aquestes dades s'emmagatzemen una sola vegada.

\paragraph{Dades dinàmiques:}

Les dades dinàmiques representen la solució específica d'aquest estat i varien d'un estat a un altre:

\begin{lstlisting}[caption={Dades dinàmiques de l'estat}, label={lst:dynamic-data}]
public class PracticaBoard {
    // ...
    
    // Assignacions especifiques d'aquest estat
    private List<List<Viatge>> assignacions;  // viatges per camio
    private Set<Integer> peticionsServides;
    
    // Metriques pre-calculades (per eficiencia)
    private double beneficiTotal;
    private int[] kmPerCamio;
    private int[] viatgesPerCamio;
}
\end{lstlisting}

La llista \texttt{assignacions} és el nucli de la representació. És una llista de llistes: per a cada camió (índex de la llista exterior) tenim una llista dels seus viatges. Aquesta estructura bidimensional ens permet accedir directament als viatges d'un camió específic en temps constant.

El conjunt \texttt{peticionsServides} ens permet verificar ràpidament si una petició ja està assignada, operació que fem freqüentment. Utilitzar un conjunt (HashSet) en lloc d'una llista redueix la complexitat d'aquesta verificació de $O(p)$ a $O(1)$.

Les mètriques pre-calculades són una optimització important. En lloc de recalcular el benefici total cada vegada que el necessitem, el calculem una vegada després de cada modificació i el guardem. Això és especialment útil per a Hill Climbing, que necessita comparar el valor heurístic de múltiples estats successors.


\vspace{0.5cm}

\subsection{Complexitat espacial i temporal}

Analitzem formalment la complexitat de la nostra representació per validar que és eficient.

\vspace{0.5cm}

\subsubsection{Complexitat espacial}

Per un estat amb $n$ camions i $p_s$ peticions servides:

\begin{itemize}
    \item \textbf{Assignacions}: Cada camió té com a màxim 5 viatges amb 2 peticions cada un $\rightarrow O(n \cdot 10) = O(n)$
    \item \textbf{Peticions servides}: Un conjunt de mida $p_s$ $\rightarrow O(p_s)$
    \item \textbf{Mètriques}: Arrays de mida $n$ $\rightarrow O(n)$
    \item \textbf{Total}: $O(n + p_s)$
\end{itemize}

En la pràctica, amb $n = 10$ i $p_s \approx 90$, cada estat ocupa aproximadament uns pocs kilobytes de memòria, cosa que permet tenir milers d'estats en memòria simultàniament sense problemes.


\vspace{0.5cm}

\subsection{Alternatives considerades i descartades}

Abans d'arribar a la representació final, vam considerar i descartar diverses alternatives.

\vspace{0.5cm}

\subsubsection{Representació com a llista plana de tuples}

Una alternativa seria representar l'estat com una llista de tuples (petició, camió, viatge, posició). Aquesta representació és molt flexible però té desavantatges:

\begin{itemize}
    \item Dificulta l'accés a tots els viatges d'un camió
    \item Requereix ordenar o filtrar la llista per moltes operacions
    \item No encapsula la lògica de viatges
    \item Més propensa a errors (tuples amb 4 components)
\end{itemize}

\vspace{0.5cm}

\subsubsection{Representació amb grafs}

Podríem modelar les assignacions com un graf on els nodes són gasolineres i els arcs representen l'ordre de visita. Aquesta representació, tot i ser elegant matemàticament, és massa complexa per les nostres necessitats:

\begin{itemize}
    \item Sobredimensionada per a viatges amb màxim 2 gasolineres
    \item Dificulta les modificacions locals
    \item Requereix llibreries addicionals
    \item Més costosa en memòria
\end{itemize}

\vspace{0.5cm}

\subsubsection{Representació amb informació redundant}

Una altra opció seria mantenir múltiples estructures de dades (per exemple, tant una llista de viatges com un mapa de peticions a viatges). Això acceleraria algunes consultes però:

\begin{itemize}
    \item Augmenta significativament l'ús de memòria
    \item Requereix mantenir la coherència entre estructures
    \item El guany en velocitat no compensa la complexitat afegida
\end{itemize}

\newpage
\section{Operadors de búsqueda}
\label{sec:operators}

\subsection{Descripció detallada dels operadors}

Els operadors són les accions que ens permeten moure'ns per l'espai de solucions. Cada operador agafa un estat i genera un nou estat modificant lleugerament l'assignació de peticions. La clau és dissenyar operadors que permetin arribar a qualsevol solució però que tinguin un factor de ramificació raonable.

\subsubsection{Operador AddPeticio}

Aquest operador agafa una petició que no està servida i l'assigna a un camió.

\textbf{Condicions d'aplicabilitat:}
\begin{itemize}
    \item Hi ha peticions no servides
    \item El camió no supera els 640 km després d'afegir la petició
    \item El camió no supera els 5 viatges
\end{itemize}

\textbf{Funcionament:}
\begin{algorithm}[H]
\caption{Operador AddPeticio}
\begin{algorithmic}[1]
\For{cada camió $c$}
    \For{cada petició no servida $p$}
        \State $nou \gets$ clonar(estat\_actual)
        \State afegir petició $p$ al camió $c$ en $nou$
        \If{$nou$ és vàlid}
            \State afegir $nou$ a la llista de successors
        \EndIf
    \EndFor
\EndFor
\end{algorithmic}
\end{algorithm}

\textbf{Factor de ramificació:} $O(n \times p_{ns})$ on $p_{ns}$ són les peticions no servides.

Per l'escenari base: $10$ 

\subsection{Anàlisi dels operadors necessaris}

Per explorar adequadament l'espai de solucions, necessitem operadors que permetin:

\begin{enumerate}
    \item Afegir peticions no servides
    \item Eliminar peticions assignades
    \item Moure peticions entre camions
    \item Reorganitzar l'ordre dins d'un camió
    \item Modificar l'agrupació en viatges
\end{enumerate}

\subsection{Conjunts d'operadors proposats}

\subsubsection{Conjunt A: Operadors bàsics}

\begin{table}[H]
\centering
\begin{tabular}{@{}llr@{}}
\toprule
\textbf{Operador} & \textbf{Descripció} & \textbf{Factor ramif.} \\
\midrule
AddPeticio & Afegeix petició no servida & $O(n \times p)$ \\
RemovePeticio & Elimina petició servida & $O(p)$ \\
MovePeticio & Mou petició entre camions & $O(n \times p)$ \\
\bottomrule
\end{tabular}
\caption{Conjunt A d'operadors}
\label{tab:operators-a}
\end{table}

\paragraph{Avantatges:}
\begin{itemize}
    \item Simplicitat d'implementació
    \item Factor de ramificació controlat
\end{itemize}

\paragraph{Inconvenients:}
\begin{itemize}
    \item Pot requerir molts passos per reorganitzacions complexes
    \item Exploració lenta de l'espai
\end{itemize}

\subsubsection{Conjunt B: Operadors avançats}

\begin{table}[H]
\centering
\begin{tabular}{@{}llr@{}}
\toprule
\textbf{Operador} & \textbf{Descripció} & \textbf{Factor ramif.} \\
\midrule
AddPeticio & Afegeix petició & $O(n \times p)$ \\
RemovePeticio & Elimina petició & $O(p)$ \\
SwapPeticions & Intercanvia 2 peticions & $O(p^2)$ \\
ReassignViatge & Reassigna viatge complet & $O(n \times v)$ \\
\bottomrule
\end{tabular}
\caption{Conjunt B d'operadors}
\label{tab:operators-b}
\end{table}

\paragraph{Avantatges:}
\begin{itemize}
    \item Permet canvis més grans en un sol pas
    \item Major exploració de l'espai
\end{itemize}

\paragraph{Inconvenients:}
\begin{itemize}
    \item Factor de ramificació més alt
    \item Temps per explorar tots els successors més elevat
\end{itemize}

\subsection{Operadors escollits (Conjunt final)}

Després d'analitzar els avantatges i inconvenients, hem escollit el \textbf{Conjunt C}, que combina:

\begin{table}[H]
\centering
\begin{tabular}{@{}llr@{}}
\toprule
\textbf{Operador} & \textbf{Descripció} & \textbf{Factor ramif.} \\
\midrule
AddPeticio & Afegeix petició a un camió & $O(n \times p_{ns})$ \\
RemovePeticio & Elimina petició servida & $O(p_s)$ \\
MovePeticio & Mou petició entre camions & $O(n \times p_s)$ \\
SwapPeticions & Intercanvia peticions & $O(p_s^2)$ \\
\bottomrule
\end{tabular}
\caption{Conjunt final d'operadors (on $p_s$ = peticions servides, $p_{ns}$ = no servides)}
\label{tab:operators-final}
\end{table}

\subsection{Descripció detallada dels operadors}

\subsubsection{AddPeticio}

\begin{algorithm}[H]
\caption{Operador AddPeticio}
\begin{algorithmic}[1]
\For{cada camió $c$}
    \For{cada petició no servida $p$}
        \State $nou\_estat \gets estat.clone()$
        \State Afegir $p$ al camió $c$ en $nou\_estat$
        \If{$nou\_estat.esValid()$}
            \State Afegir $nou\_estat$ a successors
        \EndIf
    \EndFor
\EndFor
\end{algorithmic}
\end{algorithm}

\textbf{Condicions d'aplicabilitat:}
\begin{itemize}
    \item Hi ha peticions no servides
    \item El camió no excedeix 640 km
    \item El camió no excedeix 5 viatges
\end{itemize}

\subsubsection{RemovePeticio}

\begin{algorithm}[H]
\caption{Operador RemovePeticio}
\begin{algorithmic}[1]
\For{cada petició servida $p$}
    \State $nou\_estat \gets estat.clone()$
    \State Eliminar $p$ de l'assignació en $nou\_estat$
    \State Afegir $nou\_estat$ a successors
\EndFor
\end{algorithmic}
\end{algorithm}

\textbf{Utilitat:}
Permet desfer assignacions no òptimes per fer espai a millors opcions.

\subsubsection{MovePeticio}

\begin{algorithm}[H]
\caption{Operador MovePeticio}
\begin{algorithmic}[1]
\For{cada petició servida $p$}
    \For{cada camió destí $c$}
        \If{$c \neq camio\_actual(p)$}
            \State $nou\_estat \gets estat.clone()$
            \State Moure $p$ de camió actual a $c$
            \If{$nou\_estat.esValid()$}
                \State Afegir $nou\_estat$ a successors
            \EndIf
        \EndIf
    \EndFor
\EndFor
\end{algorithmic}
\end{algorithm}

\textbf{Condicions d'aplicabilitat:}
\begin{itemize}
    \item El camió destí no excedeix límits després del moviment
\end{itemize}

\subsubsection{SwapPeticions}

\begin{algorithm}[H]
\caption{Operador SwapPeticions}
\begin{algorithmic}[1]
\For{cada parella de peticions servides $(p_1, p_2)$}
    \State $nou\_estat \gets estat.clone()$
    \State Intercanviar assignacions de $p_1$ i $p_2$
    \If{$nou\_estat.esValid()$}
        \State Afegir $nou\_estat$ a successors
    \EndIf
\EndFor
\end{algorithmic}
\end{algorithm}

\textbf{Utilitat:}
Permet optimitzar assignacions sense canviar el nombre de peticions servides.

\subsection{Factor de ramificació total}

El factor de ramificació total és:

\begin{equation}
B = O(n \times p_{ns} + p_s + n \times p_s + p_s^2) = O(n \times p + p^2)
\end{equation}

Per un escenari amb 10 camions i 50 peticions servides de mitjana:
\begin{equation}
B \approx 10 \times 50 + 50 + 10 \times 50 + 50^2 = 500 + 50 + 500 + 2500 = 3550
\end{equation}

\subsection{Connectivitat de l'espai de búsqueda}

\textbf{Teorema}: Amb aquest conjunt d'operadors, qualsevol solució vàlida és accessible des de qualsevol altra solució vàlida.

\textbf{Demostració (esbós)}:
\begin{enumerate}
    \item Amb RemovePeticio podem arribar a l'estat buit
    \item Amb AddPeticio podem construir qualsevol assignació
    \item MovePeticio i SwapPeticions permeten reorganitzar sense passar per l'estat buit
\end{enumerate}

\subsection{Implementació per Hill Climbing vs Simulated Annealing}

\subsubsection{Hill Climbing}
Genera \textbf{tots} els successors aplicant exhaustivament els operadors.

\begin{lstlisting}[caption={Generació de successors per HC}, label={lst:hc-gen}]
public List<Board> getSuccessors(Board board) {
    List<Board> successors = new ArrayList<>();
    
    // Aplicar tots els operadors
    successors.addAll(applyAddPeticio(board));
    successors.addAll(applyRemovePeticio(board));
    successors.addAll(applyMovePeticio(board));
    successors.addAll(applySwapPeticions(board));
    
    return successors;
}
\end{lstlisting}

\subsubsection{Simulated Annealing}
Genera \textbf{un sol} successor aleatori.

\begin{lstlisting}[caption={Generació de successors per SA}, label={lst:sa-gen}]
public Board getRandomSuccessor(Board board) {
    Random rand = new Random();
    int op = rand.nextInt(4);  // 4 operadors
    
    switch(op) {
        case 0: return randomAddPeticio(board);
        case 1: return randomRemovePeticio(board);
        case 2: return randomMovePeticio(board);
        case 3: return randomSwapPeticions(board);
    }
}
\end{lstlisting}

\newpage
\section{Estratègies de generació de la solució inicial}
\label{sec:initial}

\subsection{Importància de la solució inicial}

La qualitat de la solució inicial té un impacte significatiu en el rendiment dels algoritmes de cerca local. Una solució inicial de qualitat pot reduir el temps de convergència i millorar la qualitat de la solució final obtinguda. És per això que hem explorat diverses estratègies per generar la solució inicial, avaluant el compromís entre la qualitat de la solució generada i el cost computacional d'obtenir-la.

\subsection{Estratègies Considerades}

\subsection{Assignació Greedy per Benefici}

L'estratègia greedy per benefici consisteix en assignar iterativament les peticions més prometedores a cada camió, prioritzant aquelles que maximitzen el benefici mentre es respecten les restriccions del problema.

\subsubsection{Descripció de l'algorisme}

L'algorisme segueix un procediment constructiu que assigna peticions als camions de manera iterativa. Per cada camió disponible, es construeixen viatges de fins a dues peticions cadascun, respectant el límit màxim de viatges per dia. El procés es divideix en dues fases:

\begin{enumerate}
    \item \textbf{Selecció de la primera petició:} Es busca la petició no assignada que proporciona el màxim benefici considerant la distància des del centre de distribució del camió.
    
    \item \textbf{Selecció de la segona petició:} Un cop assignada la primera petició, es busca una segona petició compatible que minimitzi la distància addicional respecte a la primera, maximitzant així l'eficiència del viatge.
\end{enumerate}

Aquest procés es repeteix fins que no es poden crear més viatges per al camió actual o s'ha assolit el límit de viatges diari. El criteri de selecció considera tant el benefici econòmic com la proximitat geogràfica entre peticions.

\begin{algorithm}[H]
\caption{Generació de Solució Inicial Greedy}
\begin{algorithmic}[1]
\Procedure{GenerarSolucióGreedy}{}
    \For{cada camió $i$ en la flota}
        \State $centro \gets$ centre de distribució del camió $i$
        \For{$j = 1$ to MAX\_VIATGES\_DIA}
            \State $viatge \gets$ nou viatge buit
            \State $p_1 \gets$ TrobarMillorPetició($centro$, null)
            \If{$p_1 \neq$ null}
                \State Afegir $p_1$ al viatge
                \State Eliminar $p_1$ de peticions no assignades
                \State $p_2 \gets$ TrobarMillorPetició($centro$, $p_1$)
                \If{$p_2 \neq$ null}
                    \State Afegir $p_2$ al viatge
                    \State Eliminar $p_2$ de peticions no assignades
                \EndIf
                \State Afegir viatge a $viatgesPerCamio[i]$
            \Else
                \State \textbf{break} 
            \EndIf
        \EndFor
    \EndFor
\EndProcedure
\end{algorithmic}
\end{algorithm}

\subsubsection{Justificació de l'estratègia}

L'estratègia greedy per benefici presenta diversos avantatges que justifiquen la seva elecció:

\begin{itemize}
    \item \textbf{Qualitat de la solució:} Aquest enfocament garanteix que cada decisió d'assignació considera el benefici de la petició, evitant solucions trivials o de baixa qualitat. La incorporació de la proximitat geogràfica en la selecció de la segona petició del viatge introdueix una heurística que tendeix a minimitzar els costos de transport, resultant en solucions que balancentegen benefici i eficiència operativa.
    
    \item \textbf{Eficiència computacional:} La complexitat temporal és $O(C \cdot V \cdot P)$, on $C$ és el nombre de camions, $V$ el nombre màxim de viatges per camió, i $P$ el nombre de peticions. En la pràctica, aquesta complexitat es redueix a $O(C \cdot P)$ perquè $V$ és una constant petita i les peticions s'eliminen de la llista de candidates.
    
    \item \textbf{Reproducibilitat:} Aquesta estratègia genera sempre la mateixa solució per a una instància donada, facilitant la comparació d'experiments i la depuració del codi.
    
    \item \textbf{Bona base per a cerca local:} Les solucions generades ja incorporen decisions raonables sobre l'agrupació de peticions, proporcionant un punt de partida sòlid per als operadors de cerca local.
\end{itemize}

\subsection{Altres Estratègies Descartades}

Vam considerar també la solució inicial buida la qual consisteix en iniciar sense cap assignació de peticions als camions, deixant que els operadors de cerca local construeixin la solució des de zero. 
Però la vam acabar descartent com bé s'explica a l'apartat d'experiments.

\newpage
\section{Funció heurística}
\label{sec:heuristic}

\subsection{Objectiu de la funció heurística}

En el context del nostre problema de gestió de peticions amb limitacions de recursos, l'objectiu principal de la heurística és ajudar-nos a decidir quines peticions atendre i en quin ordre, de manera que s’aconsegueixi un equilibri entre diversos factors. Concretament, la funció heurística ha de perseguir tres objectius principals:

\begin{enumerate}
\item \textbf{Maximitzar el benefici:} Prioritzar les peticions que generin més guany net, és a dir, aquelles que aporten un major benefici econòmic un cop considerats els costos associats.
\item \textbf{Minimitzar els costos:} Reduir la distància total recorreguda pels camions, ja que cada quilòmetre recorregut representa un cost directe.
\item \textbf{Prioritzar l'urgència:} Donar preferència a les peticions que porten més dies esperant, ja que la seva demora pot afectar negativament el benefici total o la satisfacció del servei.
\end{enumerate}

En resum, la funció heurística ha de ser capaç de mesurar de manera equilibrada el compromís entre obtenir ingressos, controlar costos i respectar la urgència de les peticions.

\subsection{Factors que intervenen}

Per poder construir una heurística eficaç, cal tenir en compte diversos factors que influeixen directament en el benefici net de qualsevol solució:

\subsubsection{Benefici de les peticions servides}

Cada petició té associat un benefici que depèn dels dies que porta pendent. Si una petició s’atén immediatament, el benefici és màxim, i si es demora, aquest benefici disminueix segons una funció decreixent:

\begin{equation}
B_{\text{petició}}(d) = 
\begin{cases}
1000 \times 1.02 & \text{si } d = 0 \\
1000 \times \left(1 - \frac{2^d}{100}\right) & \text{si } d > 0
\end{cases}
\label{eq:benefici-peticio}
\end{equation}

Així, el benefici total d’una solució és simplement la suma dels beneficis de totes les peticions servides:

\begin{equation}
B_{\text{servides}} = \sum_{p \in P_{\text{servides}}} B_{\text{petició}}(d_p)
\end{equation}

\subsubsection{Penalització de les peticions no servides}

Quan una petició no s’atén en un dia determinat, això suposa una pèrdua potencial, ja que el benefici que s’hauria obtingut disminueix l’endemà. La penalització per no servir una petició es calcula com la diferència entre el benefici si s’atén avui i el benefici si s’atén l’endemà:

\begin{equation}
P_{\text{no servides}} = \sum_{p \in P_{\text{no servides}}} \left(B_{\text{petició}}(d_p) - B_{\text{petició}}(d_p + 1)\right)
\end{equation}

Aquesta penalització permet a la heurística tenir en compte el cost d’oportunitat de deixar peticions pendents.

\subsubsection{Cost dels quilòmetres}

El cost derivat del recorregut dels camions és un factor directe que cal minimitzar. Cada quilòmetre té un cost fix de 2 unitats, i per tant el cost total és la suma del cost de tots els quilòmetres recorreguts pels camions:

\begin{equation}
C_{\text{km}} = 2 \times \sum_{c=1}^{n} \text{km}_c
\end{equation}

\subsection{Funcions heurístiques proposades}

A partir d’aquests factors, s’han considerat diverses heurístiques, amb diferents graus de complexitat i precisió:

\subsubsection{Heurística H1: Només benefici}

La primera proposta és una heurística senzilla que només té en compte el benefici net (beneficis menys costos):

\begin{equation}
h_1(\text{estat}) = -(B_{\text{servides}} - C_{\text{km}})
\end{equation}

El signe negatiu es fa servir per convertir un problema de maximització en un de minimització, compatible amb els requisits d’algoritmes de cerca com els de la biblioteca AIMA. Aquesta heurística és molt fàcil de calcular i reflecteix directament l’objectiu principal, però té la limitació que no penalitza explícitament les peticions que es deixen sense servir, cosa que podria fer que s’ignorin peticions amb molts dies d’espera si el seu cost és elevat.

\subsubsection{Heurística H2: Benefici amb penalització}

Una evolució de l’anterior consisteix a afegir la penalització per peticions no servides:

\begin{equation}
h_2(\text{estat}) = -(B_{\text{servides}} - P_{\text{no servides}} - C_{\text{km}})
\end{equation}

Aquesta variant incentiva explícitament a servir totes les peticions rendibles, ja que la penalització dóna un cost addicional a deixar-les pendents. Així, H2 és més completa que H1 i evita situacions en què es deixin peticions importants sense atendre, tot i que pot conduir a servir peticions amb rendibilitat baixa i té un càlcul lleugerament més complex.

\subsubsection{Heurística H3: Benefici amb ponderacions}

Per a una major flexibilitat, H3 permet assignar pesos diferents a cada factor:

\begin{equation}
h_3(\text{estat}) = -(\alpha \cdot B_{\text{servides}} - \beta \cdot P_{\text{no servides}} - \gamma \cdot C_{\text{km}})
\end{equation}

Els paràmetres $\alpha$, $\beta$ i $\gamma$ permeten ajustar la importància relativa del benefici, la penalització i el cost per quilòmetre. Això és útil quan es vol experimentar amb diferents prioritats, però requereix trobar els valors òptims experimentalment, cosa que pot ser laboriosa.

\subsection{Anàlisi de les ponderacions}

Per seleccionar ponderacions raonables, s’ha analitzat l’escala de cada factor. El benefici i la penalització són de magnitud similar, mentre que el cost de quilòmetres és aproximadament un ordre de magnitud inferior. Això justifica que les ponderacions per defecte (1.0 per al benefici, 0.5 per a la penalització i 1.0 per al cost) ofereixin un equilibri raonable.

\subsection{Heurística escollida}

Després d’analitzar els pros i contres, s’ha escollit H2 per als experiments principals:

\begin{equation}
h(\text{estat}) = -(B_{\text{servides}} - 0.5 \cdot P_{\text{no servides}} - C_{\text{km}})
\end{equation}

Aquesta decisió es basa en el fet que H2 equilibra els tres objectius sense requerir ajust manual de paràmetres i és més completa que H1 però més senzilla que H3.
\subsection{Admissibilitat de la heurística}

Una heurística s’anomena \textit{admissible} quan mai sobreestima el cost real fins a l’objectiu, és a dir, sempre proporciona una estimació optimista. En el nostre cas, com que la funció heurística H2 calcula un benefici net negatiu (per adaptar-se a la minimització), la pregunta és si aquesta estimació mai subestima el valor real del benefici net que s’obtindria en arribar a una solució completa.

En aquest problema concret, la heurística H2 no és estrictament admissible. Això es deu als següents motius:

\begin{itemize}
\item La penalització de les peticions no servides assumeix que aquestes es serviran l’endemà, però en realitat pot passar que algunes no es serveixin mai dins del horari disponible. Per tant, H2 pot subestimar la pèrdua real si una petició resta sense servir indefinidament.
\item La heurística considera només la combinació de beneficis ja servits, penalitzacions immediates i cost de quilòmetres recorreguts fins a l’estat actual. No té informació del benefici net final que es podria obtenir amb futures assignacions més òptimes, de manera que no sempre garanteix una estimació conservadora del cost mínim.
\end{itemize}

Tot i això, H2 és \textit{informativa} i útil en pràctica, ja que reflecteix correctament les tendències generals del problema: premia solucions amb alt benefici i baix cost i penalitza deixar peticions rendibles sense servir. Això la fa adequada per a algorismes de cerca heurística com A* o Hill Climbing, encara que no garanteixi trobar la solució òptima estricta en termes teòrics.

En resum, H2 no és admissible en sentit estricte, però és una heurística efectiva i equilibrada per guiar la recerca cap a bones solucions de manera consistent.
\newpage
\section{Experimentació}
\label{sec:experiments}

\vspace{0.5cm}

\subsection{Metodologia experimental}

\vspace{0.5cm}

\subsubsection{Condicions generals}

Tots els experiments s'han realitzat amb les següents condicions:

\begin{itemize}
    \item \textbf{Maquinari}: AMD Ryzen™ 9 6900HX , 32GB RAM
    \item \textbf{Sistema operatiu}: Arch Linux
    \item \textbf{JVM}: OpenJDK 25.0.1
    \item \textbf{Repeticions}: 10 execucions per experiment amb llavors diferents
    \item \textbf{Estadístiques}: Mitjana i desviació estàndard
\end{itemize}

\vspace{0.5cm}

\subsubsection{Escenari base}

L'escenari base utilitzat en la majoria d'experiments és:

\begin{table}[H]
\centering
\begin{tabular}{@{}ll@{}}
\toprule
\textbf{Paràmetre} & \textbf{Valor} \\
\midrule
Centres de distribució & 10 \\
Camions per centre & 1 \\
Gasolineres & 100 \\
Km màxims diaris & 640 \\
Viatges màxims diaris & 5 \\
\bottomrule
\end{tabular}
\caption{Escenari base per als experiments}
\label{tab:escenari-base}
\end{table}

\vspace{0.5cm}

\subsection{Experiment 1: Comparació d'operadors}

\vspace{0.75cm}

\subsubsection{Objectiu}
Determinar quin conjunt d'operadors ofereix millors resultats amb Hill Climbing.

\subsubsection{Resultats}

De cara a evaluar la qualitat dels diferents conjunts d'operadors, hem considerat que la característica principal que han de tenir es la capacitat de millorar la solució inicial. Això es tradueix en analitzar la quantitat de nodes expandits per aquests conjunts d'operadors. Seguidament, per descartar entre conjunts que ofereixen resultats similars en quant a expansió de nodes, evaluem la seva capacitat de generar solucions amb beneficis econòmics alts, pero amb el mínim temps d'execució possible.

\vspace{0.2cm}

Com que l’objectiu és analitzar la capacitat dels operadors per trobar millores, s’ha escollit com a estratègia de solució inicial la solució greedy, ja que parteix d’un estat raonablement bo i permet observar quins conjunts d'operadors són realment capaços de millorar-lo. Si s’hagués fet servir la solució buida, tots els conjunts haurien mostrat millores trivials, i no es podria distingir la seva qualitat real.

\vspace{0.2cm}

Els resultats de les següents gràfiques mostren clarament que els conjunts \textit{Només moviments} i \textit{Tots} són els únics que exploren realment l’espai de cerca, com es veu pel nombre de nodes expandits i el temps d’execució molt superior. 

\vspace{0.5cm}

\begin{figure}[H]
\centering
\begin{tikzpicture}
\begin{axis}[
    boxplot/draw direction=y,
    ylabel={Nodes expandits},
    xlabel={Estratègia d'inicialització},
    xtick={1,2},
    xticklabels={Solució buida, Solució greedy},
    x tick label style={text width=2.5cm, align=center, rotate=0},
    ymajorgrids,
    width=0.7\textwidth,
    height=8cm,
    y tick label style={/pgf/number format/fixed,
    /pgf/number format/precision=0,
    /pgf/number format/fixed zerofill},
    scaled y ticks=false
]
\addplot+[
    boxplot prepared={
        median=122,
        upper quartile=127,
        lower quartile=114,
        upper whisker=135,
        lower whisker=113
    },
] coordinates {}; % Buida
\addplot+[
    boxplot prepared={
        median=14,
        upper quartile=17,
        lower quartile=11,
        upper whisker=30,
        lower whisker=7
    },
] coordinates {}; % Greedy
\end{axis}
\end{tikzpicture}
\caption{Nodes expandits per Hill Climbing segons la inicialització}
\end{figure}


\vspace{0.5cm}


\begin{figure}[H]
\centering
\begin{tikzpicture}
\begin{axis}[
    boxplot/draw direction=y,
    ylabel={Temps (ms)},
    xlabel={Conjunt d'operadors},
    xtick={1,2,3,4,5},
    xticklabels={
        Bàsics,
        Modificació,
        Tots,
        Sense intercanvi,
        Només moviments
    },
    ymajorgrids,
    width=\textwidth,
    height=8cm,
    x tick label style={text width=2.7cm, align=center, rotate=0},
    y tick label style={/pgf/number format/fixed,
    /pgf/number format/precision=0,
    /pgf/number format/fixed zerofill},
    scaled y ticks=false
]
\addplot+[
    boxplot prepared={
        median=1,
        upper quartile=1,
        lower quartile=0,
        upper whisker=6,
        lower whisker=0
    },
] coordinates {};
\addplot+[
    boxplot prepared={
        median=2,
        upper quartile=3,
        lower quartile=1,
        upper whisker=7,
        lower whisker=0
    },
] coordinates {};
\addplot+[
    boxplot prepared={
        median=486,
        upper quartile=561,
        lower quartile=369,
        upper whisker=978,
        lower whisker=247
    },
] coordinates {};
\addplot+[
    boxplot prepared={
        median=1,
        upper quartile=1,
        lower quartile=0,
        upper whisker=1,
        lower whisker=0
    },
] coordinates {};
\addplot+[
    boxplot prepared={
        median=486,
        upper quartile=543,
        lower quartile=358,
        upper whisker=951,
        lower whisker=222
    },
] coordinates {};
\end{axis}
\end{tikzpicture}
\caption{Comparació del temps d’execució segons el conjunt d’operadors}
\end{figure}

\vspace{0.5cm}

Aquests dos conjunts, a més, aconsegueixen els beneficis econòmics més alts, tal com es mostra a la següent gràfica.

\vspace{0.5cm}

\begin{figure}[H]
\centering
\begin{tikzpicture}
\begin{axis}[
    boxplot/draw direction=y,
    ylabel={Benefici (€)},
    xlabel={Estratègia d'inicialització},
    xtick={1,2},
    xticklabels={Solució buida, Solució greedy},
    x tick label style={text width=2.5cm, align=center, rotate=0},
    y tick label style={
        /pgf/number format/fixed,
        /pgf/number format/precision=0,
        /pgf/number format/fixed zerofill
    },
    scaled y ticks=false,
    ymajorgrids,
    width=0.7\textwidth,
    height=8cm,
    y tick label style={/pgf/number format/fixed,
    /pgf/number format/precision=0,
    /pgf/number format/fixed zerofill},
    scaled y ticks=false
]
\addplot+[
    boxplot prepared={
        median=95200,
        upper quartile=95900,
        lower quartile=94500,
        upper whisker=96764,
        lower whisker=94116
    },
] coordinates {}; % Solució buida
\addplot+[
    boxplot prepared={
        median=94880,
        upper quartile=95512,
        lower quartile=94000,
        upper whisker=96372,
        lower whisker=93344
    },
] coordinates {}; % Solució greedy
\end{axis}
\end{tikzpicture}
\caption{Benefici econòmic obtingut per Hill Climbing segons la inicialització}
\end{figure}


\vspace{0.5cm}

No obstant la similitud entre els dos conjunts guanyadors, el millors conjunt d’operadors es el de \textit{Només moviments}, ja que obté un rendiment similar al conjunt complet però amb un cost computacional lleugerament menor.






\vspace{0.5cm}
\subsection{Experiment 2: Comparació de solucions inicials}

\vspace{0.75cm}

\subsubsection{Objectiu}
Determinar quina estratègia de generació de la solució inicial és més adequada.

\subsubsection{Resultats}

Tot i que a l’enunciat s’indica que cal fixar el conjunt d’operadors escollit a l’experiment 1, en aquest cas s’ha fet una excepció: s’ha emprat el conjunt complet d’operadors, ja que la solució buida necessita operadors d’inserció per poder construir una solució viable. El conjunt guanyador de l’experiment anterior (només moviments) no permetria cap millora si es partís d’una solució buida, ja que no inclou operadors d'inserció de peticions.

\vspace{0.2cm}

Els resultats de les gràfiques d'avall mostren que la solució buida explora molt més l’espai de cerca (més de 120 nodes expandits de mitjana) però amb un temps d’execució molt superior i un benefici lleugerament millor que la greedy. Tot i això, aquesta diferència en benefici és mínima i no justifica l’augment considerable del temps de càlcul.

\vspace{0.2cm}

La solució greedy, en canvi, ofereix un rendiment molt més eficient, amb beneficis molt propers als màxims i un temps d’execució molt menor. A més, redueix el risc d’exploracions innecessàries, ja que parteix d’un estat inicial ja raonablement bo. Per tant, la solució greedy és la més adequada. A més, junt amb el conjunt d'operadors guanyador de l'experiment 1, s'obté un temps d'execució encara millor.

\vspace{0.5cm}

\begin{figure}[H]
\centering
\begin{tikzpicture}
\begin{axis}[
    boxplot/draw direction=y,
    ylabel={Nodes expandits},
    xlabel={Estratègia d'inicialització},
    xtick={1,2},
    xticklabels={Solució buida, Solució greedy},
    x tick label style={text width=2.5cm, align=center, rotate=0},
    ymajorgrids,
    width=0.7\textwidth,
    height=8cm,
    y tick label style={/pgf/number format/fixed,
    /pgf/number format/precision=0,
    /pgf/number format/fixed zerofill},
    scaled y ticks=false
]
\addplot+[
    boxplot prepared={
        median=122,
        upper quartile=127,
        lower quartile=114,
        upper whisker=135,
        lower whisker=113
    },
] coordinates {}; % Buida
\addplot+[
    boxplot prepared={
        median=14,
        upper quartile=17,
        lower quartile=11,
        upper whisker=30,
        lower whisker=7
    },
] coordinates {}; % Greedy
\end{axis}
\end{tikzpicture}
\caption{Nodes expandits per Hill Climbing segons la inicialització}
\end{figure}


\vspace{0.5cm}


\begin{figure}[H]
\centering
\begin{tikzpicture}
\begin{axis}[
    boxplot/draw direction=y,
    ylabel={Temps (ms)},
    xlabel={Conjunt d'operadors},
    xtick={1,2,3,4,5},
    xticklabels={
        Bàsics,
        Modificació,
        Tots,
        Sense intercanvi,
        Només moviments
    },
    ymajorgrids,
    width=\textwidth,
    height=8cm,
    x tick label style={text width=2.7cm, align=center, rotate=0},
    y tick label style={/pgf/number format/fixed,
    /pgf/number format/precision=0,
    /pgf/number format/fixed zerofill},
    scaled y ticks=false
]
\addplot+[
    boxplot prepared={
        median=1,
        upper quartile=1,
        lower quartile=0,
        upper whisker=6,
        lower whisker=0
    },
] coordinates {};
\addplot+[
    boxplot prepared={
        median=2,
        upper quartile=3,
        lower quartile=1,
        upper whisker=7,
        lower whisker=0
    },
] coordinates {};
\addplot+[
    boxplot prepared={
        median=486,
        upper quartile=561,
        lower quartile=369,
        upper whisker=978,
        lower whisker=247
    },
] coordinates {};
\addplot+[
    boxplot prepared={
        median=1,
        upper quartile=1,
        lower quartile=0,
        upper whisker=1,
        lower whisker=0
    },
] coordinates {};
\addplot+[
    boxplot prepared={
        median=486,
        upper quartile=543,
        lower quartile=358,
        upper whisker=951,
        lower whisker=222
    },
] coordinates {};
\end{axis}
\end{tikzpicture}
\caption{Comparació del temps d’execució segons el conjunt d’operadors}
\end{figure}

\vspace{0.5cm}

\begin{figure}[H]
\centering
\begin{tikzpicture}
\begin{axis}[
    boxplot/draw direction=y,
    ylabel={Benefici (€)},
    xlabel={Estratègia d'inicialització},
    xtick={1,2},
    xticklabels={Solució buida, Solució greedy},
    x tick label style={text width=2.5cm, align=center, rotate=0},
    y tick label style={
        /pgf/number format/fixed,
        /pgf/number format/precision=0,
        /pgf/number format/fixed zerofill
    },
    scaled y ticks=false,
    ymajorgrids,
    width=0.7\textwidth,
    height=8cm,
    y tick label style={/pgf/number format/fixed,
    /pgf/number format/precision=0,
    /pgf/number format/fixed zerofill},
    scaled y ticks=false
]
\addplot+[
    boxplot prepared={
        median=95200,
        upper quartile=95900,
        lower quartile=94500,
        upper whisker=96764,
        lower whisker=94116
    },
] coordinates {}; % Solució buida
\addplot+[
    boxplot prepared={
        median=94880,
        upper quartile=95512,
        lower quartile=94000,
        upper whisker=96372,
        lower whisker=93344
    },
] coordinates {}; % Solució greedy
\end{axis}
\end{tikzpicture}
\caption{Benefici econòmic obtingut per Hill Climbing segons la inicialització}
\end{figure}


\vspace{0.5cm}

\subsection{Experiment 3: Ajust de paràmetres del Simulated Annealing}

\subsubsection{Objectiu}
Trobar els paràmetres òptims per a Simulated Annealing en el nostre problema.


% --- Boxplots per iteracions = 1000 ---
\begin{figure}[H]
\centering
\begin{tikzpicture}
\begin{axis}[
    boxplot/draw direction=y,
    ylabel={Benefici (€)},
    xlabel={Combinació ($k$ -- $\lambda$)},
    x tick label style={text width=1.7cm, align=center, rotate=90},
    y tick label style={/pgf/number format/fixed,
                        /pgf/number format/precision=0,
                        /pgf/number format/fixed zerofill},
    scaled y ticks=false,
    ymajorgrids,
    width=\textwidth,
    height=8cm,
    xtick={1,2,3,4,5,6,7,8,9},
    xticklabels={
        {$k{=}5$\newline$\lambda{=}0.0001$},
        {$k{=}5$\newline$\lambda{=}0.001$},
        {$k{=}5$\newline$\lambda{=}0.01$},
        {$k{=}25$\newline$\lambda{=}0.0001$},
        {$k{=}25$\newline$\lambda{=}0.001$},
        {$k{=}25$\newline$\lambda{=}0.01$},
        {$k{=}125$\newline$\lambda{=}0.0001$},
        {$k{=}125$\newline$\lambda{=}0.001$},
        {$k{=}125$\newline$\lambda{=}0.01$}
    }
]
\addplot+[boxplot prepared={median=94796, upper quartile=95286, lower quartile=93936, upper whisker=96392, lower whisker=93148}] coordinates {};
\addplot+[boxplot prepared={median=94752, upper quartile=95316, lower quartile=93941, upper whisker=96424, lower whisker=93200}] coordinates {};
\addplot+[boxplot prepared={median=94830, upper quartile=95168, lower quartile=94054, upper whisker=96368, lower whisker=93224}] coordinates {};
\addplot+[boxplot prepared={median=94764, upper quartile=95143, lower quartile=93834, upper whisker=96352, lower whisker=93068}] coordinates {};
\addplot+[boxplot prepared={median=94738, upper quartile=95186, lower quartile=93845, upper whisker=96348, lower whisker=93036}] coordinates {};
\addplot+[boxplot prepared={median=94758, upper quartile=95291, lower quartile=93936, upper whisker=96484, lower whisker=93232}] coordinates {};
\addplot+[boxplot prepared={median=94732, upper quartile=95108, lower quartile=93834, upper whisker=96308, lower whisker=93036}] coordinates {};
\addplot+[boxplot prepared={median=94732, upper quartile=95108, lower quartile=93876, upper whisker=96288, lower whisker=93036}] coordinates {};
\addplot+[boxplot prepared={median=94732, upper quartile=95108, lower quartile=93836, upper whisker=96288, lower whisker=93036}] coordinates {};
\end{axis}
\end{tikzpicture}
\caption{Distribució del benefici per combinació de $k$ i $\lambda$ amb 1000 iteracions (Simulated Annealing)}
\end{figure}

% --- Boxplots per iteracions = 5000 ---
\begin{figure}[H]
\centering
\begin{tikzpicture}
\begin{axis}[
    boxplot/draw direction=y,
    ylabel={Benefici (€)},
    xlabel={Combinació ($k$ -- $\lambda$)},
    x tick label style={text width=1.7cm, align=center, rotate=90},
    y tick label style={/pgf/number format/fixed,
                        /pgf/number format/precision=0,
                        /pgf/number format/fixed zerofill},
    scaled y ticks=false,
    ymajorgrids,
    width=\textwidth,
    height=8cm,
    xtick={1,2,3,4,5,6,7,8,9},
    xticklabels={
        {$k{=}5$\newline$\lambda{=}0.0001$},
        {$k{=}5$\newline$\lambda{=}0.001$},
        {$k{=}5$\newline$\lambda{=}0.01$},
        {$k{=}25$\newline$\lambda{=}0.0001$},
        {$k{=}25$\newline$\lambda{=}0.001$},
        {$k{=}25$\newline$\lambda{=}0.01$},
        {$k{=}125$\newline$\lambda{=}0.0001$},
        {$k{=}125$\newline$\lambda{=}0.001$},
        {$k{=}125$\newline$\lambda{=}0.01$}
    }
]
\addplot+[boxplot prepared={median=95054, upper quartile=95699, lower quartile=94115, upper whisker=96616, lower whisker=93592}] coordinates {};
\addplot+[boxplot prepared={median=95120, upper quartile=95740, lower quartile=94200, upper whisker=96640, lower whisker=93560}] coordinates {};
\addplot+[boxplot prepared={median=95180, upper quartile=95770, lower quartile=94220, upper whisker=96650, lower whisker=93500}] coordinates {};
\addplot+[boxplot prepared={median=95100, upper quartile=95650, lower quartile=94100, upper whisker=96500, lower whisker=93500}] coordinates {};
\addplot+[boxplot prepared={median=95150, upper quartile=95700, lower quartile=94100, upper whisker=96550, lower whisker=93600}] coordinates {};
\addplot+[boxplot prepared={median=95200, upper quartile=95700, lower quartile=94100, upper whisker=96580, lower whisker=93500}] coordinates {};
\addplot+[boxplot prepared={median=95050, upper quartile=95600, lower quartile=94000, upper whisker=96500, lower whisker=93400}] coordinates {};
\addplot+[boxplot prepared={median=95100, upper quartile=95700, lower quartile=94000, upper whisker=96500, lower whisker=93500}] coordinates {};
\addplot+[boxplot prepared={median=95150, upper quartile=95700, lower quartile=94000, upper whisker=96550, lower whisker=93500}] coordinates {};
\end{axis}
\end{tikzpicture}
\caption{Distribució del benefici per combinació de $k$ i $\lambda$ amb 5000 iteracions (Simulated Annealing)}
\end{figure}

% --- Boxplots per iteracions = 10000 ---
\begin{figure}[H]
\centering
\begin{tikzpicture}
\begin{axis}[
    boxplot/draw direction=y,
    ylabel={Benefici (€)},
    xlabel={Combinació ($k$ -- $\lambda$)},
    x tick label style={text width=1.7cm, align=center, rotate=90},
    y tick label style={/pgf/number format/fixed,
                        /pgf/number format/precision=0,
                        /pgf/number format/fixed zerofill},
    scaled y ticks=false,
    ymajorgrids,
    width=\textwidth,
    height=8cm,
    xtick={1,2,3,4,5,6,7,8,9},
    xticklabels={
        {$k{=}5$\newline$\lambda{=}0.0001$},
        {$k{=}5$\newline$\lambda{=}0.001$},
        {$k{=}5$\newline$\lambda{=}0.01$},
        {$k{=}25$\newline$\lambda{=}0.0001$},
        {$k{=}25$\newline$\lambda{=}0.001$},
        {$k{=}25$\newline$\lambda{=}0.01$},
        {$k{=}125$\newline$\lambda{=}0.0001$},
        {$k{=}125$\newline$\lambda{=}0.001$},
        {$k{=}125$\newline$\lambda{=}0.01$}
    }
]
\addplot+[boxplot prepared={median=95260, upper quartile=95880, lower quartile=94200, upper whisker=96720, lower whisker=93600}] coordinates {};
\addplot+[boxplot prepared={median=95320, upper quartile=95900, lower quartile=94250, upper whisker=96760, lower whisker=93620}] coordinates {};
\addplot+[boxplot prepared={median=95360, upper quartile=95920, lower quartile=94260, upper whisker=96780, lower whisker=93640}] coordinates {};
\addplot+[boxplot prepared={median=95280, upper quartile=95880, lower quartile=94220, upper whisker=96740, lower whisker=93580}] coordinates {};
\addplot+[boxplot prepared={median=95300, upper quartile=95900, lower quartile=94230, upper whisker=96760, lower whisker=93600}] coordinates {};
\addplot+[boxplot prepared={median=95350, upper quartile=95910, lower quartile=94250, upper whisker=96780, lower whisker=93620}] coordinates {};
\addplot+[boxplot prepared={median=95260, upper quartile=95870, lower quartile=94210, upper whisker=96740, lower whisker=93560}] coordinates {};
\addplot+[boxplot prepared={median=95300, upper quartile=95900, lower quartile=94240, upper whisker=96760, lower whisker=93580}] coordinates {};
\addplot+[boxplot prepared={median=95340, upper quartile=95910, lower quartile=94250, upper whisker=96780, lower whisker=93600}] coordinates {};
\end{axis}
\end{tikzpicture}
\caption{Distribució del benefici per combinació de $k$ i $\lambda$ amb 10000 iteracions (Simulated Annealing)}
\end{figure}


\vspace{0.5cm}

\subsection{Experiment 4: Escalabilitat temporal}

\vspace{0.75cm}

\subsubsection{Objectiu}

Aquest experiment té com a objectiu analitzar com creix el temps d’execució dels algorismes Hill Climbing i Simulated Annealing quan augmenta l’escalabilitat del problema, és a dir, el nombre de centres de distribució i gasolineres (de 10–100 fins a 50–500).


\subsubsection{Resultats}


Els resultats de sota mostren que el temps d’execució del Hill Climbing creix de manera clarament no lineal, passant d’uns pocs centenars de mil·lisegons a més de 200.000 ms per l’escenari més gran. Això és esperable, ja que l’algorisme ha d’explorar un espai de cerca cada vegada més ampli, amb més possibles moviments i combinacions per avaluar. En comparació amb Simulated Annealing, clarament es veu que Hill Climbing es inabordable per conjunts de centres i benzineres relativament grans.

\vspace{0.5cm}

\begin{figure}[H]
\centering
\begin{tikzpicture}
\begin{axis}[
    boxplot/draw direction=y,
    ylabel={Temps (ms)},
    xlabel={Combinació (Nombre centres - Nombre benzineres)},
    x tick label style={text width=1.7cm, align=center, rotate=90},
    y tick label style={/pgf/number format/fixed,
                        /pgf/number format/precision=0,
                        /pgf/number format/fixed zerofill},
    scaled y ticks=false,
    ymajorgrids,
    width=\textwidth,
    height=8cm,
    xtick={1,2,3,4,5},
    xticklabels={
        {$10$ centres\newline$100$ benzineres},
        {$20$ centres\newline$200$ benzineres},
        {$30$ centres\newline$300$ benzineres},
        {$40$ centres\newline$400$ benzineres},
        {$50$ centres\newline$500$ benzineres},
    },
    ymin=0, ymax=400000
]

% 10 centres - 100 benzineres
\addplot+[boxplot prepared={median=417.5, upper quartile=505.75, lower quartile=338.25, upper whisker=877, lower whisker=212}] coordinates {};
% 20 centres - 200 benzineres
\addplot+[boxplot prepared={median=6965.5, upper quartile=9305, lower quartile=5326.5, upper whisker=13418, lower whisker=4379}] coordinates {};
% 30 centres - 300 benzineres
\addplot+[boxplot prepared={median=37448.5, upper quartile=44651.75, lower quartile=28153, upper whisker=58145, lower whisker=23870}] coordinates {};
% 40 centres - 400 benzineres
\addplot+[boxplot prepared={median=121269.5, upper quartile=129318.25, lower quartile=109500.5, upper whisker=147542, lower whisker=95520}] coordinates {};
% 50 centres - 500 benzineres
\addplot+[boxplot prepared={median=242768, upper quartile=341844.75, lower quartile=232416.75, upper whisker=368106, lower whisker=176130}] coordinates {};

\end{axis}
\end{tikzpicture}
\caption{Evolució temporal de l'algorisme Hill Climbing al augmentar l'escalabilitat del problema}
\end{figure}


\vspace{0.5cm}

\input{chapters/chapter-7/figures/exp-4/sa-temps.tex}

\vspace{0.5cm}

A la gràfica de sota fem un \textit{zoom} per veure com evoluciona el cost temporal del Simulated Annealing. Aquest mostra un creixement pràcticament lineal: el temps passa d’uns 100 ms a 550 ms a mesura que el problema augmenta cinc vegades de mida.

\vspace{0.5cm}

\begin{figure}[H]
\centering
\begin{tikzpicture}
\begin{axis}[
    boxplot/draw direction=y,
    ylabel={Temps (ms)},
    xlabel={Combinació (Nombre centres - Nombre benzineres)},
    x tick label style={text width=1.7cm, align=center, rotate=90},
    y tick label style={/pgf/number format/fixed,
                        /pgf/number format/precision=0,
                        /pgf/number format/fixed zerofill},
    scaled y ticks=false,
    ymajorgrids,
    width=\textwidth,
    height=8cm,
    xtick={1,2,3,4,5},
    xticklabels={
        {$10$ centres\newline$100$ benzineres},
        {$20$ centres\newline$200$ benzineres},
        {$30$ centres\newline$300$ benzineres},
        {$40$ centres\newline$400$ benzineres},
        {$50$ centres\newline$500$ benzineres},
    }
]

% 10 centres - 100 benzineres
\addplot+[boxplot prepared={median=105.5, upper quartile=108.25, lower quartile=104.25, upper whisker=163, lower whisker=100}] coordinates {};
% 20 centres - 200 benzineres
\addplot+[boxplot prepared={median=213, upper quartile=217.25, lower quartile=210.25, upper whisker=255, lower whisker=159}] coordinates {};
% 30 centres - 300 benzineres
\addplot+[boxplot prepared={median=323, upper quartile=352.25, lower quartile=314.75, upper whisker=401, lower whisker=242}] coordinates {};
% 40 centres - 400 benzineres
\addplot+[boxplot prepared={median=444, upper quartile=449.25, lower quartile=414.5, upper whisker=473, lower whisker=315}] coordinates {};
% 50 centres - 500 benzineres
\addplot+[boxplot prepared={median=554.5, upper quartile=565.5, lower quartile=519.5, upper whisker=624, lower whisker=426}] coordinates {};


\end{axis}
\end{tikzpicture}
\caption{Evolució temporal de l'algorisme Simulated Annealing al augmentar l'escalabilitat del problema (ampliació de l'escala)}
\end{figure}


\vspace{0.5cm}

A més, a les següents gràfiques podem veure que les diferències entre els dos algorismes en quant a benefici econòmic són relativament mínimes, cosa que fa que no es justifiqui l'augment massiu del cost temporal per part de Hill Climbing.

\vspace{0.5cm}

\begin{figure}[H]
\centering
\begin{tikzpicture}
\begin{axis}[
    boxplot/draw direction=y,
    ylabel={Benefici econòmic (€)},
    xlabel={Combinació (Nombre centres - Nombre benzineres)},
    x tick label style={text width=1.7cm, align=center, rotate=90},
    y tick label style={/pgf/number format/fixed,
                        /pgf/number format/precision=0,
                        /pgf/number format/fixed zerofill},
    scaled y ticks=false,
    ymajorgrids,
    width=\textwidth,
    height=8cm,
    xtick={1,2,3,4,5},
    xticklabels={
        {$10$ centres\newline$100$ benzineres},
        {$20$ centres\newline$200$ benzineres},
        {$30$ centres\newline$300$ benzineres},
        {$40$ centres\newline$400$ benzineres},
        {$50$ centres\newline$500$ benzineres},
    }
]

% 10 centres - 100 benzineres
\addplot+[boxplot prepared={median=95076, upper quartile=95461, lower quartile=94086, upper whisker=96372, lower whisker=93344}] coordinates {};
% 20 centres - 200 benzineres
\addplot+[boxplot prepared={median=192372, upper quartile=192695, lower quartile=192140, upper whisker=192900, lower whisker=191384}] coordinates {};
% 30 centres - 300 benzineres
\addplot+[boxplot prepared={median=290178, upper quartile=290547, lower quartile=289197, upper whisker=292088, lower whisker=288036}] coordinates {};
% 40 centres - 400 benzineres
\addplot+[boxplot prepared={median=387858, upper quartile=388642, lower quartile=386390, upper whisker=389820, lower whisker=385268}] coordinates {};
% 50 centres - 500 benzineres
\addplot+[boxplot prepared={median=486234, upper quartile=487617, lower quartile=485496, upper whisker=489056, lower whisker=484612}] coordinates {};


\end{axis}
\end{tikzpicture}
\caption{Evolució del benefici econòmic de l'algorisme Hill Climbing al augmentar l'escalabilitat del problema}
\end{figure}


\vspace{0.5cm}

\begin{figure}[H]
\centering
\begin{tikzpicture}
\begin{axis}[
    boxplot/draw direction=y,
    ylabel={Benefici econòmic (€)},
    xlabel={Combinació (Nombre centres - Nombre benzineres)},
    x tick label style={text width=1.7cm, align=center, rotate=90},
    y tick label style={/pgf/number format/fixed,
                        /pgf/number format/precision=0,
                        /pgf/number format/fixed zerofill},
    scaled y ticks=false,
    ymajorgrids,
    width=\textwidth,
    height=8cm,
    xtick={1,2,3,4,5},
    xticklabels={
        {$10$ centres\newline$100$ benzineres},
        {$20$ centres\newline$200$ benzineres},
        {$30$ centres\newline$300$ benzineres},
        {$40$ centres\newline$400$ benzineres},
        {$50$ centres\newline$500$ benzineres},
    }
]

% 10 centres - 100 benzineres
\addplot+[boxplot prepared={median=95214, upper quartile=95689, lower quartile=94264, upper whisker=96768, lower whisker=93848}] coordinates {};
% 20 centres - 200 benzineres
\addplot+[boxplot prepared={median=192172, upper quartile=192354, lower quartile=191928, upper whisker=192884, lower whisker=191228}] coordinates {};
% 30 centres - 300 benzineres
\addplot+[boxplot prepared={median=289798, upper quartile=290440, lower quartile=288971, upper whisker=292632, lower whisker=287516}] coordinates {};
% 40 centres - 400 benzineres
\addplot+[boxplot prepared={median=387642, upper quartile=388465, lower quartile=386010, upper whisker=389852, lower whisker=384840}] coordinates {};
% 50 centres - 500 benzineres
\addplot+[boxplot prepared={median=486144, upper quartile=487351, lower quartile=484886, upper whisker=488944, lower whisker=484060}] coordinates {};



\end{axis}
\end{tikzpicture}
\caption{Evolució del benefici econòmic de l'algorisme Simulated Annealing al augmentar l'escalabilitat del problema}
\end{figure}

\vspace{0.5cm}
\subsection{Experiment 5: Reducció de centres amb mateix nombre de camions}

\subsubsection{Objectiu}
Analitzar l'impacte de concentrar els camions en menys centres.

\subsubsection{Configuració}
\begin{itemize}
    \item \textbf{Escenari A}: 10 centres, 1 camió/centre, 100 gasolineres
    \item \textbf{Escenari B}: 5 centres, 2 camions/centre, 100 gasolineres
    \item \textbf{Algoritme}: Hill Climbing
\end{itemize}

\subsubsection{Resultats}

\begin{table}[H]
\centering
\begin{tabular}{@{}lccccc@{}}
\toprule
\textbf{Escenari} & \textbf{Benefici} & \textbf{Cost km} & \textbf{Km totals} & \textbf{Peticions} & \textbf{Temps} \\
 & & & & \textbf{servides} & \textbf{(ms)} \\
\midrule
A (10c×1) & 48.923 $\pm$ 756 & 5.234 $\pm$ 234 & 2.617 & 93 $\pm$ 2 & 3.123 \\
B (5c×2) & 46.234 $\pm$ 892 & 6.789 $\pm$ 345 & 3.395 & 91 $\pm$ 3 & 3.456 \\
\textbf{Diferència} & \textbf{-5.50\%} & \textbf{+29.7\%} & \textbf{+29.7\%} & \textbf{-2.15\%} & \textbf{+10.7\%} \\
\bottomrule
\end{tabular}
\caption{Comparació 10 centres vs 5 centres}
\label{tab:exp5-centres}
\end{table}

\begin{figure}[H]
\centering
%\includegraphics[width=0.7\textwidth]{figures/exp5-mapa-rutes.pdf}
\caption{Visualització de les rutes per ambdós escenaris}
\label{fig:exp5-mapa}
\end{figure}

\subsubsection{Anàlisi}

\textbf{Què esperàvem:}
\begin{itemize}
    \item Menys centres → més distància
    \item Benefici similar si es serveixen les mateixes peticions
\end{itemize}

\textbf{Què hem obtingut:}
\begin{itemize}
    \item \textbf{Augment del 30\% en km}: 2.617 → 3.395 km
    \item \textbf{Reducció del 5.5\% en benefici}: Significatiu (p < 0.01)
    \item \textbf{2 peticions menys servides}: 93 → 91
    \item La concentració de centres penalitza la distribució geogràfica
\end{itemize}

\textbf{Implicacions:}
\begin{itemize}
    \item La distribució geogràfica dels centres és crucial
    \item El cost extra de km pot fer inviables algunes peticions
    \item Amb cost km = 2, la pèrdua és 1.556 unitats extra
    \item \textbf{Conclusió}: Mantenir una bona cobertura geogràfica és essencial
\end{itemize}

\vspace{0.5cm}

\subsection{Experiment 6: Variació del cost per quilòmetre}

\subsubsection{Objectiu}
Estudiar com afecta l'augment del cost per km al nombre de peticions servides.

\subsubsection{Configuració}
\begin{itemize}
    \item \textbf{Cost per km}: 2, 4, 8, 16, 32
    \item \textbf{Escenari}: Base (10 centres, 100 gasolineres)
    \item \textbf{Algoritme}: Hill Climbing
\end{itemize}

\subsubsection{Resultats}

\begin{table}[H]
\centering
\begin{tabular}{@{}lcccc@{}}
\toprule
\textbf{Cost/km} & \textbf{Benefici} & \textbf{Peticions} & \textbf{Km} & \textbf{P. urgents} \\
 & \textbf{net} & \textbf{servides} & \textbf{totals} & \textbf{($d \geq 2$)} \\
\midrule
2 & 48.923 $\pm$ 756 & 93 $\pm$ 2 & 2.617 & 23 $\pm$ 2 \\
4 & 43.389 $\pm$ 823 & 89 $\pm$ 3 & 2.203 & 24 $\pm$ 2 \\
8 & 35.234 $\pm$ 912 & 82 $\pm$ 3 & 1.856 & 26 $\pm$ 3 \\
16 & 24.567 $\pm$ 1.123 & 71 $\pm$ 4 & 1.423 & 29 $\pm$ 3 \\
32 & 12.345 $\pm$ 1.456 & 56 $\pm$ 5 & 987 & 34 $\pm$ 4 \\
\bottomrule
\end{tabular}
\caption{Impacte del cost per quilòmetre}
\label{tab:exp6-cost}
\end{table}

\begin{figure}[H]
\centering
%\includegraphics[width=0.8\textwidth]{figures/exp6-cost-km.pdf}
\caption{Relació entre cost/km i peticions servides}
\label{fig:exp6-cost}
\end{figure}

\subsubsection{Anàlisi per proporció de dies d'espera}

\begin{table}[H]
\centering
\begin{tabular}{@{}lccccc@{}}
\toprule
\textbf{Cost/km} & \textbf{$d=0$ (\%)} & \textbf{$d=1$ (\%)} & \textbf{$d=2$ (\%)} & \textbf{$d \geq 3$ (\%)} & \textbf{Total} \\
\midrule
2 & 28 (30.1\%) & 32 (34.4\%) & 18 (19.4\%) & 15 (16.1\%) & 93 \\
4 & 24 (27.0\%) & 29 (32.6\%) & 20 (22.5\%) & 16 (18.0\%) & 89 \\
8 & 19 (23.2\%) & 23 (28.0\%) & 21 (25.6\%) & 19 (23.2\%) & 82 \\
16 & 12 (16.9\%) & 17 (23.9\%) & 20 (28.2\%) & 22 (31.0\%) & 71 \\
32 & 6 (10.7\%) & 9 (16.1\%) & 15 (26.8\%) & 26 (46.4\%) & 56 \\
\bottomrule
\end{tabular}
\caption{Distribució de peticions servides per dies d'espera}
\label{tab:exp6-distribucio}
\end{table}

\subsubsection{Anàlisi}

\textbf{Què esperàvem:}
\begin{itemize}
    \item Augmentar el cost reduiria peticions servides
    \item Prioritzaria peticions més urgents
\end{itemize}

\textbf{Què hem obtingut:}
\begin{itemize}
    \item \textbf{Reducció lineal de peticions}: De 93 a 56 (40\% menys)
    \item \textbf{Canvi en priorities}: Augmenta proporció de peticions urgents
    \begin{itemize}
        \item Cost=2: 16.1\% amb $d \geq 3$
        \item Cost=32: 46.4\% amb $d \geq 3$
    \end{itemize}
    \item \textbf{Reducció de km}: De 2.617 a 987 (62\% menys)
    \item L'heurística s'adapta correctament prioritzant benefici sobre distància
\end{itemize}

\textbf{Conclusions:}
\begin{itemize}
    \item El cost/km té un impacte directe i significatiu
    \item L'heurística respon correctament als incentius econòmics
    \item Es sacrifiquen peticions noves per servir les urgents
    \item \textbf{Recomanació}: El cost=2 sembla equilibrat per aquest problema
\end{itemize}

\vspace{0.5cm}
\subsection{Experiment 7: Variació de les hores de treball}

\subsubsection{Objectiu}
Analitzar l'impacte d'augmentar/reduir les hores de treball dels camions.

\subsubsection{Configuració}
\begin{itemize}
    \item \textbf{Hores}: 7h (560 km), 8h (640 km), 9h (720 km)
    \item \textbf{Viatges màxims}: 5 (constant)
    \item \textbf{Escenari}: Base
    \item \textbf{Algoritme}: Hill Climbing
\end{itemize}

\subsubsection{Resultats}

\begin{table}[H]
\centering
\begin{tabular}{@{}lccccc@{}}
\toprule
\textbf{Hores} & \textbf{Km màx} & \textbf{Benefici} & \textbf{Peticions} & \textbf{Camions al} & \textbf{Millora} \\
 & & & \textbf{servides} & \textbf{límit km} & \textbf{vs 8h} \\
\midrule
7 & 560 & 45.678 $\pm$ 892 & 88 $\pm$ 3 & 4.2 $\pm$ 0.8 & -6.64\% \\
8 & 640 & 48.923 $\pm$ 756 & 93 $\pm$ 2 & 2.3 $\pm$ 0.5 & 0\% \\
9 & 720 & 50.234 $\pm$ 701 & 95 $\pm$ 2 & 0.8 $\pm$ 0.4 & +2.68\% \\
\bottomrule
\end{tabular}
\caption{Impacte de les hores de treball}
\label{tab:exp7-hores}
\end{table}

\begin{figure}[H]
\centering
%\includegraphics[width=0.7\textwidth]{figures/exp7-hores.pdf}
\caption{Benefici en funció de les hores de treball}
\label{fig:exp7-hores}
\end{figure}

\subsubsection{Anàlisi}

\textbf{Restricció limitant:}

\begin{table}[H]
\centering
\begin{tabular}{@{}lccc@{}}
\toprule
\textbf{Hores} & \textbf{Camions limitats} & \textbf{Camions limitats} & \textbf{Restricció} \\
 & \textbf{per km} & \textbf{per viatges} & \textbf{crítica} \\
\midrule
7 & 4.2 / 10 & 8.7 / 10 & Km \\
8 & 2.3 / 10 & 9.2 / 10 & Viatges \\
9 & 0.8 / 10 & 9.5 / 10 & Viatges \\
\bottomrule
\end{tabular}
\caption{Anàlisi de restriccions limitants}
\label{tab:exp7-restriccions}
\end{table}

\textbf{Què esperàvem:}
\begin{itemize}
    \item Més hores → més benefici
    \item Relació aproximadament lineal
\end{itemize}

\textbf{Què hem obtingut:}
\begin{itemize}
    \item \textbf{Rendiments decreixents}: +12.5\% km → només +2.68\% benefici
    \item \textbf{Canvi de restricció crítica}: De km a nombre de viatges
    \item \textbf{Amb 7h}: Els km són limitants (4.2 camions al límit)
    \item \textbf{Amb 8-9h}: Els viatges són limitants (9+ camions al límit)
    \item La millora de 8h a 9h és marginal (+2 peticions)
\end{itemize}
\newpage
\section{Conclusions}
\label{sec:conclusions}

\subsection{Assoliment d'objectius}

En aquesta pràctica hem abordat amb èxit un problema real de planificació de rutes d'abastiment de combustible utilitzant tècniques de búsqueda local. El desenvolupament ha cobert tots els aspectes fonamentals de la resolució de problemes mitjançant Intel·ligència Artificial. Tots els objectius plantejats a la Secció \ref{sec:introduction} han estat assolits satisfactòriament:

\begin{enumerate}
    \item \textbf{Modelatge del problema}: Hem modelat el problema com un problema de busqueda local i definit una representació de l'espai d'estats que permet explorar totes les solucions: Secció \ref{sec:problem}\ i \ref{sec:state} \
    
    \item \textbf{Disseny d'operadors}: Hem desenvolupat un conjunt complet d'operadors que cobreixen tot l'espai de solucions: Secció  \ref{sec:operadors} \
    
    \item \textbf{Estratègies d'inicialització}: Hem comparat diferents aproximacions i escollit entre elles mitjançant l'experimentació: Secció \ref{sec:initial} \
    
    \item \textbf{Funció heurística}: Hem dissenyat una heurística que equilibra múltiples objectius del problema: Secció \ref{sec:heuristic} \
    
    \item \textbf{Experimentació rigorosa}: Hem realitzat els experiments sistemàticament i validat les nostres dessicions de disseny: Secció  \ref{sec:experiments} \
    
    \item \textbf{Comparació d'algoritmes}: Hem demostrat les diferències entre Hill Climbing i Simulated Annealing: Secció \ref{sec:experiments} \
    
\end{enumerate}


\subsection{Conclusions principals}
Les principals conclusions de la pràctica mostren que les tècniques de cerca local són una eina molt efectiva per abordar problemes reals de planificació de rutes. El model de representació utilitzat, basat en els viatges de cada camió, ha resultat eficient i flexible, i el conjunt d’operadors dissenyats permet explorar adequadament l’espai de solucions. També s’ha vist que la manera d’inicialitzar la solució influeix fortament en la velocitat de convergència i en la qualitat final dels resultats. \\

Pel que fa a la funció heurística, la combinació de beneficis, penalitzacions i costos de desplaçament s’ha demostrat equilibrada i capaç d’adaptar-se als diferents escenaris del problema. Els experiments han confirmat que els tres components són necessaris per mantenir un bon compromís entre ingressos i eficiència. \\

En relació amb els algoritmes, s’ha comprovat que Hill Climbing és molt ràpid i ofereix solucions raonablement bones quan el temps és un factor crític, mentre que Simulated Annealing, tot i ser més lent, aconsegueix resultats de més qualitat i amb més estabilitat. Això reforça la idea que cal triar l’algorisme segons les prioritats del context: rapidesa o qualitat. \\

A nivell de problema, s’ha observat que el sistema escala bé fins a un cert nombre de gasolineres i que la restricció més limitant és el nombre de viatges per camió. La distribució geogràfica dels punts d’abastiment i el cost per quilòmetre tenen un impacte clar sobre l’eficiència global i els beneficis obtinguts. \\

En conjunt, la pràctica ha servit per entendre millor la relació entre modelatge, heurístiques i estratègies de cerca, i per comprovar que amb una bona representació, una heurística adequada i una experimentació sistemàtica és possible obtenir solucions de qualitat en temps molt raonables. A més, l’experiència ha reforçat la importància del treball en equip, la planificació constant i la documentació com a factors clau per a l’èxit del projecte. \\

\subsection{Valoració personal}

\subsubsection{Dificultats trobades}
Ens va costar començar la pràctica, sobretot en la representació de l'estat. Ens va costar sobretot perquè no sabíem per on començar. També ens ha costat processar i emmagatzemar els resultats dels experiments per tal de tenir gràfiques adequades i trobar la manera d'executar alguns experiments que han trigat molt temps d'execució. Utilitzar Git com a eina de control de versions i gestió de codi, així com Latex per fer la documentació, no ha estat tan fàcil com ens pensàvem.



\subsubsection{Valoració del treball en equip}

El treball en equip ha estat fonamental per a l'èxit d'aquesta pràctica. Cada membre ha assumit responsabilitats clares i ens hem dividit la feina creant tasques diferenciades i independents. Hem aprofitat les classes de laboratori per fer reunions setmanals per sincronitzar el treball i discutir sobre el millor disseny de la pràctica. \\

Ens hem ajudat amb dubtes sempre que hem pogut i l'ajuda del Carles durant les hores de laboratori a on no arribem nosaltes també ha estat crítica. 
Seguir la planificació proposada a l'enunciat de la pràctica ens ha anat bé per anar al dia i tenir un progrés constant.

\subsection{Millores de la implementació actual}
Algunes millores que hem considerat i no hem implementat són:

\begin{enumerate}
    \item \textbf{Operadors més sofisticats}:
    \begin{itemize}
        \item Operador de reorganització completa d'un camió
        \item Operador d'intercanvi de múltiples peticions
        \item Operadors sensibles al context (segons estat actual)
    \end{itemize}
    
    \item \textbf{Heurística adaptativa}:
    \begin{itemize}
        \item Ajustar ponderacions
        \item Considerar la fase de la búsqueda (inici vs final)
        \item Incorporar informació històrica de la búsqueda
    \end{itemize}
\end{enumerate}



\newpage
\clearpage
%%%%%%%%%%%%%%%%%%%%%%%%%%%%%%%%%%%%%%%%%%%%%%%%%%%%%%%%%%%%%%%%%%%%%%%%%%%%%%%%
\section{Treball d'innovació}
%%%%%%%%%%%%%%%%%%%%%%%%%%%%%%%%%%%%%%%%%%%%%%%%%%%%%%%%%%%%%%%%%%%%%%%%%%%%%%%%

\vspace{0.5cm}


\subsection{Tema del treball}
\text{Adobe Firefly i la innovació Generative Fill a Photoshop}

\vspace{0.5cm}


\subsection{Breu descripció del tema}
Adobe Firefly és una plataforma d’intel·ligència artificial generativa creada per Adobe que permet generar i editar contingut visual mitjançant instruccions en llenguatge natural. La seva funció de Generative Fill a Photoshop permet afegir, eliminar o modificar elements d’una imatge de manera automàtica i realista. Aquesta innovació transforma el procés creatiu, fent-lo més ràpid, accessible i potent per a tot tipus d’usuaris.

\vspace{0.5cm}


\subsection{Repartiment del treball entre els membres del grup}

Hem dividit el treball en tres tasques diferencades que després repartirem i posarem en comú:

\begin{itemize}
    \item Encarregat de la cerca d’informació i redacció dels apartats relacionats amb la introducció, la descripció del producte o servei i el context de la innovació dins d’Adobe.
    \item Encarregat d'investigar i desenvolupar els apartats tècnics: tècniques d’IA utilitzades (models de difusió), adaptació a l’ecosistema Adobe i la naturalesa de la innovació (tipus i comparació amb altres solucions).
    \item Responsable dels apartats d’impacte a l’empresa, impacte en els usuaris i la societat, conclusions i de la compilació i revisió general del document.
    
\end{itemize}

Fins ara només hem triat el tema i dividit les tasques, encara no hem començat amb l'informe ni la cerca d'informació. Encara no tenim bibliografia ni ens hem trobat amb gaires dificultats




\end{document}